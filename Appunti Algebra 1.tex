\documentclass[11pt]{scrartcl}
\usepackage[italian]{babel}
\usepackage[sexy]{evan}

\begin{document}
\title{Appunti Algebra 1}
\subtitle{\large\normalfont\rmfamily\scshape APPUNTI DEL CORSO DI ALGEBRA 1 TENUTO\\ DALLA PROF. DEL CORSO E DAL PROF. LOMBARDO}
\author{Diego Monaco \\ \textnormal{\href{d.monaco2@studenti.unipi.it}{d.monaco2@studenti.unipi.it}}}
\date{Anno Accademico 2022-23}
\maketitle
\newpage

\tableofcontents
\eject
\newpage

\section*{Ringraziamenti}
Federico Allegri, Pietro Crovetto, Davide Ranieri, Francesco Sorce.

\newpage
\section{Automorfismi}
\subsection{Automorfismi di $G$}
Dato un gruppo $G$ possiamo definire l'insieme degli automorfismi di $G$ come segue:
    \[ \Aut(G) = \{\varphi : G \longrightarrow G |\, \varphi \, \text{isomorfismo}\}
        \]
si verifica facilmente che $(\Aut(G), \circ)$ è un gruppo, e in particolare $\Aut(G) \leqslant S(G)$,
ovvero il gruppo delle permutazioni di $G$. Si osserva che $id \in \Aut(G)$, $\varphi \in \Aut(G) \implies 
\varphi^{-1} \in \Aut(G)$ e $\varphi,\psi \in \Aut(G) \implies \varphi \circ \psi \in \Aut(G)$.

\begin{example}
    [Esempi di automorfismi]
    Esempi di insiemi di automorfismi:
        \begin{itemize}
            \item $\Aut(\ZZ) = \{\pm id\}$.
            \item $\Aut(\Zn) \cong \Zn^*$.
            \item $\Aut(\Z2 \times \Z2) \cong S_3$.
            \item $\Aut(\underbrace{\Zp \times \ldots \times \Zp}_{n \;\text{volte}}) \cong GL_n(\Fp)$
        \end{itemize}
\end{example}

\subsection{Automorfismi interni}
\begin{definition}
    Dato un gruppo $G$ possiamo definire l'omomorfismo di \vocab{coniugio}:
        \[ \varphi_g : G \longrightarrow G : x \longmapsto gxg^{-1}
            \]
    dove l'elemento $gxg^{-1}$ si dice \vocab{coniugato} di $g$.
\end{definition}

\begin{proposition}
    \label{prop1}
    Valgono i seguenti fatti:
    \begin{enumerate}[(1)]
        \item $\varphi_g \in \Aut(G)$, $\forall g \in G$.
        \item $\{\varphi_g | g \in G\} = \Inn(G) \trianglelefteqslant \Aut(G)$.\footnote{$\Inn(G)$ si definisce \vocab{gruppo degli automorfismi interni}.}
    \end{enumerate}
\end{proposition}

\begin{proof}
    Proviamo le due affermazioni:
        \begin{enumerate}[(1)]
            \item Per verificare che $\varphi_g$ è un automorfismo devo verificare che $\varphi_g$ è ben definita, ma ciò segue dalla chiusura di $g$ per l'operazione,
                verifichiamo allora che sia un omomorfismo:
                    \[ \varphi_g(xy) = gxyg^{-1} = gxg^{-1}gyg^{-1} = \varphi_g(x)\varphi_g(y)
                    \qquad \forall x,y \in G 
                        \]
                ci resta da verificare che sia una bigezione. Partiamo dalla surgettività, vogliamo verificare che $\forall y \in G$, $\exists g \in G :$
                    \[ \varphi_g(x) = y
                        \]
                in tal caso basta prendere $x = gyg^{-1} \in G$. Per l'iniettività si osserva:
                    \[ \ker \varphi_g = \{x \in G | \varphi_g(x) = e\} = \{x \in G | gxg^{-1} = e \iff x = e\} = \{e\}
                        \]
                pertanto $\varphi_g$ è iniettivo.
            \item Verifichiamo che $\Inn(G) \trianglelefteqslant \Aut(G)$, mostriamo prima che $\Inn(G)$ è un sottogruppo di $\Aut(G)$, infatti: 
            $id = \varphi_e \in \Inn(G)$, $\forall g_1,g_2 \in G$ vale che $\varphi_{g_{1}} \circ \varphi_{g_{2}} = \varphi_{g_1g_2} \in \Inn(G)$, infatti:
                    \[ \varphi_{g_{1}} \circ \varphi_{g_{2}}(x) = \varphi_{g_{1}}(g_2xg_2^{-1}) = g_1g_2xg_2^{-1}g_1^{-1} = \varphi_{g_1g_2}(x)
                        \]
                infine, $(\varphi_{g})^{-1} = \varphi_{g^{-1}} \in \Inn(G)$:
                    \[ (\varphi_{g})^{-1} \circ \varphi_{g}(x) = (\varphi_{g})^{-1} (gxg^{-1}) = x \iff (\varphi_{g})^{-1} = \varphi_{g^{-1}}
                        \]
                e analogamente per l'inversa a destra. Per verificare la normalità bisogna mostrare che:
                    \[ f \circ \Inn(G) \circ f^{-1} \subseteq \Inn(G)
                    \qquad \qquad \forall f \in \Aut(G)
                        \]
                ovvero:
                    \[ f \circ \varphi_g \circ f^{-1} \in \Inn(G)
                    \qquad \qquad \forall f \in \Aut(G), \forall \varphi_g \in \Inn(G)
                        \]
                si osserva che $f \circ \varphi_g \circ f^{-1} = \varphi_{f(g)} \in \Inn(G)$, infatti:
                    \[ f \circ \varphi_g \circ f^{-1} (x) = f(\varphi_g(f^{-1} (x))) = f(g(f^{-1}(x))g^{-1}) =
                        \]\[ = f(g) f(f^{-1}(x)) f(g^{-1}) = f(g) x (f(g))^{-1} = \varphi_{f(g)}
                            \]
        \end{enumerate}
\end{proof}

\begin{remark}
    Se $G$ è abeliano, allora $\Inn(G) = \{id\}$, infatti:
        \[ gxg^{-1} = gg^{-1}x = x \qquad \qquad \forall x \in G, \forall g \in G
            \]
\end{remark}

\begin{proposition}
    Dato un gruppo $G$ si ha:
    \[\Inn(G) \cong \faktor{G}{Z(G)}\]
\end{proposition}

\begin{proof}
    Per dimostrare il teorema ci basta trovare un omomorfismo surgettivo da $G$ in $\Inn(G)$ e poi sfruttare il Primo Teorema di Omomorfismo. Sia:
        \[ \phi : G \longrightarrow \Inn(G) : g \longmapsto \varphi_g
            \]
    tale applicazione è chiaramente ben definita, ed è surgettiva per come abbiamo definito $\Inn(G)$. Verifichiamo che è un omomorfismo:
        \[ \phi(g_1g_2) = \varphi_{g_1g_2} = \varphi_{g_1} \circ \varphi_{g_2} = \phi(g_1) \circ \phi(g_2)
        \qquad \qquad \forall g \in G
            \]
        dove la penultima uguaglianza è vera per quanto visto nella dimostrazione del (2) della proposizione precedente. A questo punto, per il 
        primo teorema di omomorfismo si ha che:
            \begin{center}
            \begin{tikzcd}
                G \arrow[d, "\pi_{\ker{\phi}}" left, twoheadrightarrow] \arrow[r, "\phi", twoheadrightarrow] &\Inn(G)\\	
                \faktor {G}{\ker{\phi}} \arrow[ur, "\widesim{}"' above, sloped, anchor=center] & 
            \end{tikzcd}
            \end{center}
        dunque:
            \[ \frac{G}{\ker{\phi}} \cong \Inn(G)
                \]
        non ci resta che osservare:
            \begin{multline*}
                \ker{\phi} = \{g \in G | \phi(g) = \varphi_g = id\} = \{g \in G | gxg^{-1} = x, \forall x \in G\} = \\ 
                = \{g \in G | gx = xg, \forall x \in G\} = Z(G)
            \end{multline*}
\end{proof}

\begin{remark}
    L'isomorfismo trovato è del tipo $gZ(G) \longmapsto \varphi_g$, ricordiamo che è ben definito per il Primo Teorema di Omomorfismo.
\end{remark}

\begin{remark}
    Si ricorda che se $\faktor{G}{Z(G)}$ è ciclico, allora $G$ è abeliano (e quindi $\faktor{G}{Z(G)}$ è banale), infatti, sia:
        \[ \faktor{G}{Z(g)} = \left<gZ(G)\right>
            \]
    Presi $g_1,g_2 \in G$, si ha che $g_1Z(G) = g^{k_1}Z(G)$ e $g_2Z(G) = g^{k_2}Z(G)$, da cui:
        \[ g^{-k_1}g_1Z(G) = Z(G) \iff  g^{-k_1}g_1 \in Z(G)
            \]
    ovvero $\exists z_1 \in Z(G): g_1 = g^{k_1}z_1$ e analogamente $g_2 = g^{k_2}z_2$, da cui:
        \[ g_1g_2 = g^{k_1}z_1g^{k_2}z_2 = g^{k_1}g^{k_2}z_1z_2 = g^{k_1+k_2}z_1z_2
            \]
    e contemporaneamente:
        \[ g_2g_1 = g^{k_2}z_2g^{k_1}z_1 = g^{k_2}g^{k_1}z_2z_1 = g^{k_2 + k_1}z_2z_1 = g^{k_1+k_2}z_1z_2
            \]
    dove nell'ultimo passaggio si è sfruttato il fatto che $k_1,k_2 \in \ZZ$ e $z_1,z_2 \in Z(G)$. Da ciò segue che $G$ è abeliano.
\end{remark}

\begin{remark}
    Dunque $\Inn(G)$ ciclico $\implies \faktor{G}{Z(G)}$ ciclico $\implies G$ abeliano da cui:  
        \[ \Inn(G) \cong \faktor{G}{Z(G)} \cong \{e\}
            \]
\end{remark}

\begin{remark}
    $N \trianglelefteqslant G \iff \forall \varphi_g \in \Inn(G)$ si ha $\varphi_g(N) = N$ (o anche $\varphi_g(N) \subseteq N$). Equivalentemente, i sottogruppi
    normali di $G$ sono i sottogruppi \textbf{invarianti} per automorfismi interni (ovvero sono tali che $gNg^{-1} = N$, $\forall g \in G$). Se $N \trianglelefteqslant G$, si può considerare:
        \[ \Inn(G) \longrightarrow \Aut(N) : \varphi_g \longmapsto \varphi_{g \mid N}
            \]
    con $\varphi_{g \mid N} : N \longrightarrow N$ che è un automorfismo, infatti rimane iniettivo, la surgettività segue dal fatto che $\varphi_g(N) = N$, e infine, essendo $\varphi_g$ 
    un omomorfismo su tutti gli elementi di $G$, lo sarà in particolare anche su tutti gli elementi di $N$. Dunque quando si ha un sottogruppo normale, ogni automorfismo interno si restringe
    a un automorfismo di $N$.
\end{remark}

Abbiamo visto che i sottogruppi normali sono invarianti per automorfismi interni, possiamo generalizzare quest'idea e considerare i sottogruppi invarianti per automorfismi:

\begin{definition}
    Dato un sottogruppo $H \leqslant G$, esso si dice \vocab{caratteristico} se è invariante per automorfismi:
        \[ f(H) = H
        \qquad \qquad \forall f \in \Aut(G)
            \]
\end{definition}

Anche in questo caso basta verificare che $f(H) \subseteq H$, $\forall f \in \Aut(G)$, perché si ha anche che:
    \[ f^{-1}(H) \subseteq H
        \]
da cui si ottiene:
    \[ f(f^{-1}(H)) \subseteq f(H)
        \]

\begin{remark}
    Si osserva che se $H$ è caratteristico in $G$, allora è invariante per tutti gli automorfismi di $G$ (e quindi in particolare quelli interni), dunque
    se $H$ è caratteristico in $G$, allora è anche normale. Il viceversa è falso.
\end{remark}

\begin{remark}
    Se $H$ è caratteristico in $G$ (dunque normale), si può scrivere un'applicazione:
        \[ \Aut(G) \longrightarrow \Aut(H) : f \longmapsto f_{\mid H}
            \]
    dove $f_{\mid H}$ è un automorfismo di $H$.
\end{remark}

\begin{remark}
    Si osserva che se $H$ è l'unico sottogruppo di $G$ di un certo ordine, allora $H$ è caratteristico in $G$ (segue immediatamente dal fatto che gli automorfismi
    preservano gli ordini degli elementi).
\end{remark}

\begin{exercise}
    Il centro di un gruppo, $Z(G)$ è un sottogruppo caratteristico.
\end{exercise}
        
\begin{soln}
    Per dimostrare che $Z(G)$ è caratteristico è sufficiente far vedere che:
        \[ f(Z(G)) \subseteq Z(G)
        \qquad \forall f \in \Aut(G)
            \]
    ovvero:
        \[ f(z) \in Z(G)
        \qquad \forall f \in \Aut(G), \forall z \in Z(G)
            \]
    dunque bisogna verificare che:
        \[ gf(z) = f(z)g \qquad \forall g \in G
            \]
    poiché $f$ è un automorfismo, allora $\exists h \in G : f(h) = g$, dunque:
        \[ gf(z) = f(h)f(z) = f(hz) = f(zh) = f(z)f(h) = f(z)g \qquad \forall g \in G
            \]
\end{soln}

\begin{example}
    Sia $G = \Z2 \times \Z2 = \{(\ol 0, \ol 0),(\ol 1, \ol 0),(\ol 0, \ol 1),(\ol 1, \ol 1)\}$, $G$ ha ordine $4$ ed ha tre sottogruppi ciclici di ordine 2:
        \[ H_1 = \left<(\ol 1, \ol 0)\right> \qquad H_2 = \left<(\ol 0, \ol 1)\right> \qquad H_3 = \left<(\ol 1, \ol 1)\right>
            \] 
    ed essendo $G$ abeliano si ha $H_1,H_2,H_3 \trianglelefteqslant G$ (e quindi i sottogruppi sono invarianti per automorfismi interni). Tuttavia nessuno dei sottogruppi è caratteristico,
    infatti possiamo prendere un automorfismo non banale (e quindi non uno interno) e vedere come i sottogruppi di questo tipo non siano invarianti:
        \[ f = \begin{cases}
            (\ol 1, \ol 0) \longmapsto (\ol 1, \ol 1)\\
            (\ol 0, \ol 1) \longmapsto (\ol 0, \ol 1)
        \end{cases}
            \]
    la definizione della mappa data tuttavia non è completa, perché abbiamo stabilito solo dove vengono mandati i generatori, dobbiamo definire cosa faccia un elemento generico:
        \[ f ((\ol a, \ol b)) = af((\ol 1, \ol 0)) + bf((\ol 0, \ol 1)) = (\ol a, \ol a) + (\ol 0, \ol b) = (\ol a, \ol{a + b})
            \]
    a questo punto abbiamo definito completamente l'applicazione (rimarrebbe da verificare che $f$ sia un omomorfismo),  e si verifica facilmente che
     $f(H_1) = H_3 $ quindi $H_1 \trianglelefteqslant G$, ma non caratteristico.
\end{example}

A questo punto è facile verificare che:
    \[ \Aut(\Z2 \times \Z2) \cong S_3
        \]
infatti, ogni automorfismo del gruppo si ottiene fissando l'elemento neutro $(\ol 0, \ol 0) \longmapsto (\ol 0, \ol 0)$, quindi il numero possibile di bigezioni è al più $3!$, occorre
verificare che tutte e $6$ le funzioni sono omomorfismi. Dimostriamo invece che:
    \[ \boxed{\Aut(S_3) \cong S_3}
        \]
Per farlo, poiché $S_3$ non è abeliano, possiamo osservare che:
    \[ \Inn(S_3) \cong \faktor{S_3}{Z(S_3)} \cong S_3
        \]
in quanto l'unico elemento che commuta con tutti gli altri in $S_3$ è l'identità, quindi $Z(S_3) = \{id\} \cong \{e\}$.
Per quanto detto si ha $\Inn(S_3) \trianglelefteqslant \Aut(S_3)$ e quindi $\Aut(S_3)$ contiene una copia isomorfa di $S_3$ come sottogruppo normale, pertanto,
se verifichiamo che $|\Aut(S_3)| \leq 6$ abbiamo concluso. Sia $f \in \Aut(S_3)$, $f$ può al più scambiare i $3$ elementi di ordine $2$, d'altra parte, fissate le 
immagini di $\tau_1,\tau_2,\tau_3$\footnote{Con $\tau_i$ si intendono le trasposizioni che lasciano fisso l'elemento $i$.}, i due 
$3$-ciclci\footnote{Come si vedrà $S_3 = \left<\tau_1,\tau_2,\tau_3\right>$} sono completamente determinati,
ciò significa che si hanno al più $3!$ automorfismi, dunque:
    \[ \Aut(S_3) = \Inn(S_3) \cong S_3 \implies \Aut(S_3) \cong S_3 
        \]

\newpage
\subsection{Azione di un gruppo su un insieme}

\begin{definition}
    Sia $G$ un gruppo e $X$ un insieme, un'\vocab{azione} di $G$ su $X$ è un omomorfismo:
        \[ \varphi : G \longrightarrow S(X) : g \longmapsto \varphi_g (= \varphi(g))
            \]
    dove $\varphi_g : X \longrightarrow X : x \longmapsto \varphi_g(x)$, con $\varphi_g$ bigettiva, $\forall g \in G$.
\end{definition}

\begin{example}
    Sia $X = G$, quindi $\varphi : G \longrightarrow S(G) : g \longmapsto \varphi_g$, con $\varphi_g$ coniugio, $\varphi$ è un'azione. Come si è visto nell'($1$)
    della \hyperref[prop1]{Proposizione 1.3} $\varphi_g$ è un automorfismo di $G$ (e quindi una bigezione), e $\varphi$ è un omomorfismo. In questo caso si ha che:
        \[ \varphi_g(x) = gxg^{-1}
            \]
\end{example}

\begin{example}
    Sia $V$ un $K$-spazio vettoriale, sia:
        \[ \varphi : K^* \longrightarrow S(V) : \lambda \longmapsto \varphi_\lambda
            \]
    con $\varphi_\lambda : V \longrightarrow V : \underline v \longmapsto \lambda \underline v$, $\varphi$ è un'azione di $K^*$ su $V$.
\end{example}

Sia $\varphi: G \longrightarrow S(X)$ un'azione, $\varphi$ definisce una relazione di equivalenza su $X$:
    \[ x \sim y \iff \exists g \in G : \varphi_g(x) = y
        \]
    ovvero due elementi sono in relazione se esiste un'applicazione $\varphi_g \in S(X)$, per cui un elemento è l'immagine dell'altro mediante 
    tale applicazione. La relazione è appunto di equivalenza, infatti: $x \sim x$, per $g = e$ si ha (essendo $\varphi$ un omomorfismo) $\varphi_e(x) = id(x) = x$, 
    $x \sim y \implies y \sim x$:
        \[ \varphi_g(x) = y \implies x = (\varphi_g(y))^{-1} = \varphi_{g^{-1}}(y)
            \]
    infine $x \sim y$, $y \sim z \implies x \sim z$, infatti si avrebbe: $\varphi_g(x) = y$, $\varphi_h(y) = z$ da cui:
        \[ z = \varphi_h(\varphi_g(x)) = \varphi_{hg}(x) \implies x \sim z
            \]

\begin{definition}
    Data la relazione di equivalenza $\sim$ si definiscono \vocab{orbite} le classi di equivalenza di $X$ rispetto alla relazione $\sim$:
        \[ \Orb(x) = \{\varphi_g(x) | g \in G\} (\subseteq X)
            \]
\end{definition}
Da cui:
    \[ X = \bigcupdot_{x \in \mathcal{R}} \Orb(x)
        \]
Con $\mathcal{R}$ insieme di rappresentanti. Un'orbita è quindi l'insieme di tutte le immagini di un elemento in un insieme, mediante tutte le possibili 
applicazioni (permutazioni) dell'insieme $\varphi(G)$.

\begin{definition}
    Per ogni $x \in X$ si dice \vocab{stabilizzatore} di $x$:
        \[ \St(x) = \{g \in G | \varphi_g(x) = x\}
            \]
\end{definition}
Cioè lo stabilizzatore è l'insieme degli elementi di $G$, che danno origine mediante $\varphi$ alle applicazioni $\varphi_g \in S(X)$, che lasciano fisso un determinato elemento.

\begin{example}
    Se $X = \RR^2$ e $G$ è il gruppo di traslazioni di vettore $\ul v = (0, l)$, allora:
        \[ \varphi : G \longrightarrow S(X) : \tau_{(0,l)} \longmapsto \tau_{(0,l)} \footnote{Si osserva che il primo $\lambda_{(0,l)}$ è un elemento del gruppo $G$, mentre il secondo è un'applicazione bigettiva di $X$.}
            \]
    con:
        \[ \Orb(x,y) = \{(x,y + l) | l \in \RR\}
        \quad \text e \quad
        \St(x,y) = \{\tau_{(0,l)} | (x, y + l) = (x,y)\}= \{id\}
            \]
\end{example}

\begin{example}
    Se $X = \RR^2$ e $G$ è il gruppo delle rotazioni di centro $O$, allora:
        \[ \varphi: G \longrightarrow S(\RR^2) : r_\theta \longmapsto r_\theta
            \]
    con:
        \[ \St(x,y) =
        \begin{cases}
            \{id\} & \text{se $(x,y) \ne (0,0)$}\\
            G & \text{se $(x,y) = (0,0)$}
        \end{cases}
            \]
    e, detta $\omega$ la circonferenza di centro $O$ raggio $\sqrt{x^2+y^2}$:
        \[ \Orb(x,y) = \{(x^{\prime},y^{\prime}) \in \RR^2 | (x^{\prime}, y^{\prime}) \in \omega\}
            \]
\end{example}

\begin{proposition}
    [$\St(x) \leqslant G$]
    Dato un gruppo $G$ e un'azione $\varphi : G \longrightarrow S(X)$, si ha che $\St(x) \leqslant G$.\footnote{In generale lo
     stabilizzatore non è un sottogruppo normale.}
\end{proposition}

\begin{proof}
    Si osserva che $e \in \St(x)$, in quanto $\varphi_e(x) = id(x) = x$, inoltre, presi $g,h \in \St(x)$, ovvero $\varphi_g (x) = \varphi_h(x) = x$, allora:
        \[ \varphi(gh) = \varphi_{gh}(x) = \varphi_g \circ \varphi_h (x) = \varphi_g(\varphi_h(x)) = \varphi_g(x) = x \implies gh \in \St(x)
            \]
    dove si ha che $ \varphi_{gh}(x) = \varphi_g \circ \varphi_h (x)$ in quanto $\varphi$ è un omomorfismo.
    Infine, preso $g \in \St(x)$, si ha $g^{-1} \in \St(x)$, infatti $\varphi_g$ è bigettiva e quindi ammette inversa:
        \[ (\varphi_g)^{-1} \circ \varphi_g (x) = x \implies (\varphi_g)^{-1}(\varphi_g(x)) = x \implies (\varphi_g)^{-1}(x) = x
            \]
    con $(\varphi_g)^{-1}(x) = (\varphi(g))^{-1}(x) = (\varphi(g^{-1}))(x) = \varphi_{g^{-1}}(x)$ e per quanto detto:
        \[ \varphi_{g^{-1}} (x) = x \implies g^{-1} \in \St(x)
            \]
\end{proof}

\begin{remark}
    Sia $x \in X$ e $g,h \in G$, allora:
        \[ \varphi_g(x) = \varphi_h(x) \iff \varphi_{h^{-1}}(\varphi_g(x)) = x
            \]
    e per le proprietà di omomorfismo dell'azione $\varphi$, si ha:
        \[ \varphi_{h^{-1}}(\varphi_g(x)) = x \iff \varphi_{h^{-1}g}(x) = x \iff h^{-1}g \in \St(x)
            \]
    ovvero $g \St(x) = h \St(x)$, in quanto $\St(x) \leqslant G$ e la condizione ottenuta è esattamente quella dell'equivalenza modulo $\St(x)$,
    quindi:
    \[ \Orb(x) \longleftrightarrow \text{classi laterali di $\St(x)$ in $G$}
        \]
    cioè due elementi danno la stessa immagine se e solo se stanno nella stessa classe laterale modulo $\St(x)$, e la corrispondenza biunivoca tra orbita e classi
    laterali è data da:
    \[ g\St(x) \longmapsto \varphi_g(x) \qquad \text e \qquad h\St(x) \longmapsto \varphi_h(x)
                \]
    che è ben definita per quanto detto all'inizio, è iniettiva:
        \[ \varphi_g(x) = \varphi_h(x) \iff g\St(x) = h\St(x)
            \]
    (quindi due elementi di un orbita sono uguali se e solo se lo sono le classi laterali dei rispettivi elementi che generano le applicazioni sono uguali modulo $\St(x)$) e surgettiva:
        \[ \forall y \in \Orb(x), y = \varphi_g(x) \implies g\St(x) \longmapsto y
            \]
\end{remark}

Per quanto detto si ha:
    \[ |G| = |\St(x)| [G : \St(x)]
        \]
ma $[G : \St(x)]$ è il numero di classi laterali di $\St(x)$ in $G$, che è proprio uguale a $|\Orb(x)|$ pertanto vale la seguente:

\begin{proposition}
    \label{p:1.25}
    Sia $G$ un gruppo finito e $X$ un insieme, allora:
        \[ |G| = |\Orb(x)||\St(x)|
            \qquad \forall x \in X
            \]
\end{proposition}

\begin{remark}
    Si osserva che essendo $\St(x) \leqslant G$, allora è ovvio (per Lagrange) che $|\St(x)| \mid |G|$, tuttavia, per la proposizione precedente, si ha che:
    $|\Orb(x)| \mid |G| $ con $\Orb(x) \subseteq X$.
\end{remark}

Ricordando che:
    \[ X = \bigcupdot_{x \in \mathcal R} \Orb(x)
        \]
se $|X|<+\infty$ si ha:
    \[ \colorboxed{red}{
        |X| = \sum_{x \in \mathcal{R}} |\Orb(x)| = \sum_{x \in \mathcal{R}} \frac{|G|}{|\St(x)|}
    }\]


\newpage
\subsection{Azione di coniugio}
\begin{definition}
    Si parla di \vocab{azione di coniugio}, quando si ha un'azione di $G$ su $G$ stesso:
        \[ \varphi : G \longrightarrow \Inn(G) (\trianglelefteqslant S(G)) : g \longrightarrow \varphi_g
            \]
\end{definition}

Abbiamo già osservato che è un'azione (ovvero che $\varphi$ è un omomorfismo). In questo caso:
    \[ \Orb(x) = \{\varphi_g(x) | g \in G\} = \{ gxg^{-1} | g \in G\} = C_x
        \]
dove $C_x$ prende il nome di \vocab{classe di coniugio} di $x$. Mentre:
    \[ \St(x) = \{g \in G | \varphi_g(x) = gxg^{-1} = x\} = Z_G(x)
        \]
dove $Z_G(x)$ si dice \vocab{centralizzatore} di $x$. Per quanto detto in precedenza si ha:
    \[ |G| = |C_x||Z_G(x)|
        \]
In particolare $|C_x| \mid |G|$ e :
    \[ |X| = \sum_{x \in \mathcal{R}} |C_x| = \sum_{x \in \mathcal{R}} \frac{|G|}{|Z_G(x)|}
        \]

\begin{remark}
    $C_x$ è un sottoinsieme, non un sottogruppo di $G$, poiché non c'è mai l'identità.
\end{remark}

\begin{remark}
    Osserviamo che $Z_G(x) = G \iff x \in Z(G)$, infatti la per un elemento del centro si ha che 
    $\forall g \in G$ l'elemento commuta, e dunque il suo centralizzatore è tutto il gruppo.
\end{remark}

\begin{remark}
    Per un'azione di coniugio ha che $x \in Z(G)$ se e solo se $\Orb(x) = \{x\}$ (ovvero $\varphi_g(x) = x$, $\forall g \in G$).
\end{remark}

    \[ |G| = \sum_{x \in Z(G)} \frac{|G|}{|Z_G(x)|} + \sum_{x \in \mathcal{R}\setminus Z(G)} \frac{|G|}{|Z_G(x)|}
        \]
ma, per quanto detto, se $x \in Z(G)$, allora $\displaystyle\frac{|G|}{|Z_G(x)|} = |C_x| = \{x\}$, segue dunque la relazione:
    \[ \colorboxed{red}{|G| = |Z(G)| + \sum_{x \in \mathcal{R}\setminus Z(G)} \frac{|G|}{|Z_G(x)|}}
        \]
che prende il nome di \vocab{formula delle classi} (di coniugio).

\newpage
\subsection{Applicazioni ai $p$-gruppi}
\begin{definition}
    Si definisce \vocab{$p$-gruppo} un gruppo di ordine $p^n$, con $p$ primo e $n \geq 1$.
\end{definition}

Se $G$ è un $p$-gruppo la formula delle classi diventa:
    \[ p^n = |G| = |Z(G)| + \sum_{x \in \mathcal{R}\setminus Z(G)} \frac{|G|}{|Z_G(x)|}
        \]
con $|Z(G)| = p^z$, $0 \leq z \leq n$, facciamo due osservazioni fondamentali:
    \begin{enumerate}[(1)]
        \ii Il centro di un $p$-gruppo non è mai banale, infatti, se osserviamo la formula delle classi, si ha:
            \[ p^n = |Z(G)| +  \sum_{x \in \mathcal{R}\setminus Z(G)} \frac{|G|}{|Z_G(x)|} \implies
             |Z(G)| +  \sum_{x \in \mathcal{R}\setminus Z(G)}\frac{|G|}{|Z_G(x)|} \equiv 0 \pmod p
                \]
            con $\displaystyle \frac{|G|}{|Z_G(x)|} > 1$, poiché se un elemento sta nel centro tutti gli addendi sono $1$ per quanto detto,
            viceversa deve essere che $\displaystyle \frac{|G|}{|Z_G(x)|} = p^kx$, $k>0$, poiché $G$ è un $p$-gruppo, dunque:
                \[ |Z(G)| \equiv 0 \pmod p \implies |Z(G)| \geq 2
                    \]
            e quindi il centro di un $p$-gruppo non è mai banale.
        \ii Un gruppo di ordine $p^2$ è abeliano, infatti, si ha:
            \[ |G| = p^2 \implies |Z(G)| = \begin{cases}
                                                1 &\text{non può accadere per (1)} \\
                                                p &\text{no perché allora $G/Z(G)$ ciclico, ma $G$ non è abeliano}\\
                                                p^2
                                            \end{cases}
                \]
            dunque l'unica possibilità è che $Z(G) = G \iff G$ abeliano.
    \end{enumerate}

\newpage
\subsection{Teorema di Cauchy}

\begin{theorem}
    [Teorema di Cauchy]
    Dato un gruppo $G$ e un primo $p$, se $p \mid |G|$, allora $\exists x \in G : \ord_G(x) = p$. \footnote{Si considera già noto il teorema per gruppi abeliani.}
\end{theorem}

\begin{proof}
    Sia $|G| = pn$, procediamo per induzione su $n$, nel caso $n = 1$ il teorema è ovvio. Supponiamo vera la tesi per i gruppi di ordine $pm$, con 
    $1 \leq m < n$ e proviamola per $n$. Distinguiamo due casi:
        \begin{itemize}
            \item Se esiste $H \lneq G$ con $p \mid H$, ovvero $|H| = pm \implies$ vale il teorema di Cauchy per ipotesi induttiva (essendo $m<n$), quindi 
            $\exists x \in H : \ord_H(x) = p$, ma essendo $H \subset G \implies x \in G$ e quindi la tesi è vera.
            \item Se $\forall H \lneq G$ si ha $p \nmid |H|$, allora si può applicare a $G$ la formula delle classi:
                \[ pn = |G| = |Z(G)| + \sum_{x \in \mathcal{R}\setminus{Z(G)}} \frac{|G|}{|Z_G(x)|}
                    \]
                ricordando il centralizzatore di $x$ è uno stabilizzatore (e quindi un sottogruppo di $G$), si ha $p \nmid |Z_G(x)|$, e quindi:
                    \[ p \Bigm| \sum_{x \in \mathcal{R}\setminus{Z(G)}} \frac{|G|}{|Z_G(x)|}
                        \]
                da cui segue che $p \mid |Z(G)| = |G| - \sum pl_x$, per quanto premesso ($\forall H \lneq G$ si ha $p \nmid |H|$), ed essendo $Z(G) \leqslant G$, l'unica 
                possibilità è che $Z(G) = G$ e vale il teorema poiché è già stato dimostrato per il caso in cui $G$ è abeliano.
        \end{itemize}
\end{proof}

\newpage
\subsection{Azione di coniugio su un sottogruppo}
Sia $X = \{H \leqslant G\}$ e $\varphi : G \longrightarrow S(X) : g \longmapsto \varphi_g(X)$, con $\varphi_g : X \longrightarrow X : H \longmapsto gHg^{-1}$. 
    Si verifica facilmente che $\varphi$ è un omomorfismo, per verificare l'iniettività si osserva che:
        \[ \varphi_g(H) = \varphi_g(K) \iff gHg^{-1} = gKg^{-1} \iff H = K
            \]
    mentre per la surgettività si ha che $\forall H \in X, \exists L \in X$:
        \[ \varphi_g(L) = H \iff gLg^{-1} = H \implies L = g^{-1}Hg
            \]
    inoltre si ha anche:
        \[ \Orb(H) = \{\varphi_g(H) | g \in G\} = \{gHg^{-1}| g \in G\} \quad \St(H) = \{g \in G | \varphi_g(H) = H\} = N_G(H)
            \]
    dove $\Orb(H)$ è l'insieme dei coniugati di $H$, mentre $\St(H) = N_G(H)$ prende il nome di \vocab{normalizzatore} di $H$.

\begin{remark}
    Si osserva che $N \trianglelefteqslant G$ se e solo se $\Orb(H) = \{H\} \iff N_G(H) = G$, ovvero se $H$ è sempre chiuso per coniugio in $G$.
\end{remark}

Per quanto affermato nella \hyperref[p:1.25]{Proposizione 1.25} si ha:
    \[ |G| = |\Orb(H)||N_G(H)| \implies |\Orb(H)| = \frac{|G|}{|N_G(H)|}
    \]
\begin{remark}
    Quindi in generale, dato $H \leqslant G$ si ha che $\#\{gH\} = [G:H]$ e $\#\{gHg^{-1}\} = [G : N_G(H)]$.
\end{remark}

\begin{remark}
    [Sulla definizione di sottogruppo normale]
    I sottogruppi normali possono essere ridefiniti nella maniera seguente, $H \trianglelefteqslant G$ se e solo se:
        \[ H = \bigcupdot_{h \in H} C_h
            \]
    cioè un sottogruppo è normale se e solo se è l'unione delle classi di coniugio dei suoi elementi. Infatti:
        \[ H \trianglelefteqslant G \iff ghg^{-1} \in H \qquad \forall h \in H, \forall g \in G
            \]
    che equivale a:
        \[ C_h = \{ghg^{-1} | h \in H\} \subseteq H  \quad \forall h \in H \implies \bigcupdot_{h \in H} C_h \subseteq H
            \]
    d'altra parte se $H$ è normale è chiuso per coniugio, ovvero il coniugio di ogni suo elemento è ancora in $H$ e in particolare
    ciò significa che:
        \[ H \subseteq \bigcupdot_{h \in H} C_h
            \] 
    
\end{remark}

\newpage
\subsection{Teorema di Cayley}

\begin{theorem}
    \label{p:Cauchy}
    Ogni gruppo è isomorfo ad un sottogruppo di un gruppo di permutazioni. In particolare, se $|G| = n$, allora 
    $G$ è isomorfo a un sottogruppo di $S_n$.
\end{theorem}

\begin{proof}
    Definiamo la mappa:
        \[ \lambda : G \longrightarrow S(G) : g \longmapsto \varphi_g
            \]
    con $\varphi_g : G \longrightarrow G : g \longmapsto gx$, l'applicazione $\lambda$ prende il nome di \vocab{rappresentazione regolare a sinistra} di $G$, si 
    vuole dimostrare che $\lambda$ è un omomorfismo iniettivo.
    Osserviamo innanzitutto che $\lambda$ è ben definita, cioè $\varphi_g \in S(G)$, infatti $\varphi_g$ è iniettiva (segue dalle leggi di cancellazione) e 
    surgettiva, perché $\forall y \in G$, $\exists g^{-1}y \in G : \varphi_g(g^{-1}y) = y$. Verifichiamo che $\lambda$ è un omomorfismo:
        \[ \lambda(g_1g_2) = \varphi_{g_1g_2}
            \]
    con $\varphi_{g_1g_2} (x) = \varphi_{g_1} \circ \varphi_{g_2} (x)$, $\forall x \in G$, e quindi:
        \[ \lambda(g_1g_2) = \lambda(g_1) \lambda(g_2)
        \qquad \forall g_1,g_2 \in G
            \]
    infine, per l'iniettività si ha che:
        \[ \ker \lambda = \{g \in G | \lambda(g) = \varphi_g = id = \varphi_e\} = \{e\}
            \]
\end{proof}

\begin{remark}
    In generale, dato $G = \left\{g_1 = e,g_2, \ldots, g_n\right\}$ e $\lambda : G \longrightarrow S(G) \cong S_n$, si ha che:
        \[ g_1 = e \longmapsto \lambda_{g_1} : G \longrightarrow G : g_i \longmapsto g_i
            \]
        \[ g_2 \longmapsto \lambda_{g_2} : G \longrightarrow G : x \longmapsto g_2x : g_2^2x \longmapsto \ldots \longmapsto g_2^{k-1}x
            \]
        con $k = \ord_G(g_2)$. $\lambda_{g_2}$ può essere rappresentata mediante la notazione dei cicli:
            \[ (x,g_2x,\ldots,g_2^{k-1}x)
                \]
        preso poi $y \not\in \lambda_{g_2}(G)$, si ha analogamente:
        \[ (y,g_2y,\ldots,g_2^{k-1}y)
            \]
\end{remark}

\begin{example}
    Nel caso in cui $G = \Z8$ e $X = \{0,\dots,7\}$, si ha l'azione:
        \[ \lambda : G \longrightarrow S_8 (= S(X)) : \ol a \longmapsto \lambda_a
            \]
    che genera ad esempio le applicazioni:\footnote{Per $+$ si intende la somma modulo $8$.}
            \begin{align*}
            \begin{array}{ccc}
            1 &\longmapsto \lambda_1 : X \longrightarrow X : a \longmapsto 1 + a \implies &(0,1,\ldots,7) \\
            2 &\longmapsto \lambda_2 : X \longrightarrow X : a \longmapsto 2 + a \implies &(0,2,4,6)(1,3,5,7)\\
            4 &\longmapsto \lambda_4 : X \longrightarrow X : a \longmapsto 4 + a \implies &(0,4)(1,5)(2,6)(3,7)\\		
            \end{array}
            \end{align*}
    che permutano gli elementi di $X$ secondo i cicli trovati.
\end{example}

\nopagebreak

\begin{definition}
    Un'azione $\lambda$ si dice \vocab{fedele} se è iniettiva.
\end{definition}
Ad esempio l'azione di rappresentazione regolare a sinistra è fedele:
    \[ \ker \lambda = \{g \in G | \lambda(g) = id\} = \{g \in G | \lambda_g(e) = e\} = \{g \in G | ge = e\} = \{e\}
        \]
da cui $\lambda$ fedele.

\begin{remark}
    Esiste anche un'applicazione $\rho : G \longrightarrow S(G) (\cong S_n)$, ($n = |G|$), detta azione di \vocab{rappresentazione regolare a destra}, con:
        \[ g \longmapsto \rho_g : x \longmapsto xg^{-1}
            \]
\end{remark}

\begin{lemma}
    \label{davide}
    Sia $G$ un gruppo abeliano di ordine $n$, allora $\forall d\mid n, \exists H \leqslant G : |H| = d$.
    \footnote{La dimostrazione non è stata fatta durante il corso, ma è stata comunque aggiunta per completezza.}
\end{lemma}

\begin{proof}
    Si consideri innanzitutto il caso $d=p^k$, $p$ primo, e mostriamolo per induzione:
    per $k=1$ la tesi è equivalente al \hyperref[p:Cauchy]{Teorema di Cauchy} (anche solo per i gruppi abeliani).
    Supponiamo la tesi per $k-1$. Poiché in particolare $p\mid |G|$ scegliamo un sottogruppo $H$ di $G$ di ordine $p$;
    tale sottogruppo è normale poiché $G$ è abeliano. $p^{k-1}\mid |G/H|\implies$ per ipotesi induttiva $\exists K\leqslant G,\ |K|=p^{k-1}$. \\
    Prendendo la controimmagine di $K$ tramite la proiezione al quoziente troviamo il sottogruppo di G cercato. A questo punto possiamo scrivere in generale
    $d=p_1^{k_1}\ldots p_s^{k_s}$; per ogni $i$ troviamo sottogruppi $H_i$ di ordini $p_i^{k_i}$ (tutti normali). Si ha quindi che $H_1H_2\leqslant G$ per normalità,
    inoltre $|H_1\cap H_2|=1$ poiché l'ordine di un elemento in tale intersezione deve dividere $(p_1^{k_1}, p_2^{k_2})=1$. Pertanto $|H_1H_2|=p_1^{k_1}p_2^{k_2}$.
    Ragionando per induzione otteniamo che il sottogruppo $H_1\ldots H_k$ ha ordine $d$ come voluto.
\end{proof}

\newpage

\begin{exercise}
    Sia $G$ un gruppo, se $|G| = p^n$, allora esiste:
        \[ \{e\} = H_n < H_{n-1} < \ldots < H_1 < G
            \]
    con $H_i \trianglelefteqslant G$ e $|H_i| = p^{n-i}$, $\forall i \in \{1,\ldots,n\}$.
\end{exercise}

\begin{soln}
    Procediamo per induzione su $n$, per $n = 1$ è ovvio, infatti si ha $H_1 = \{e\} \leqslant G$.
    Supponiamo la tesi vera $\forall 1 \leq k \leq n-1$, osserviamo che $G$ è un $p$-gruppo, pertanto il suo centro non è banale:
        \[ |Z(G)| = p^z \qquad z\geq 1
            \]
    sia $\mathcal{G} = \faktor{G}{Z(G)}$, essendo $|G/Z(G)| < p^n$ (perché deve essere $|Z(G)| \geq p$), allora vale l'ipotesi induttiva, dunque
    $|\mathcal{G}| = p^m$, con $m = n-z\ (<n)$, allora esiste:
        \[ \mathcal{H}_m = \{e_{\mathcal{G}}\} < \mathcal{H}_{m-1} < \ldots < \mathcal{H}_1 < \mathcal{G}
            \]
    con $|\mathcal{H}_i| = p^{m-i}$ e $\mathcal{H}_i \trianglelefteqslant \mathcal{G}$. Data la proiezione al quoziente:
        \[ \pi_{Z(G)} : G \longrightarrow \mathcal{G}
            \]
    per il Teorema di Corrispondenza dei sottogruppi, esiste una bigezione tra i sottogruppi di $\faktor{G}{Z(G)}$ e i sottogruppi di
    $G$ che contengono $Z(G)$, la quale preserva normalità e indice del sottogruppo, pertanto preso $\mathcal{H}_i \leqslant \faktor{G}{Z(G)}$ è sufficiente
    applicare $\pi_{Z(G)}^{-1}$ alla catena scritta sopra, e si trova:
        \[ Z(G) = \pi_{Z(G)}^{-1}(\mathcal{H}_{m}) < \ldots < \pi_{Z(G)}^{-1}(\mathcal{H}_{1}) < \pi_{Z(G)}^{-1}(\mathcal{G}) (= G)
            \]
    Segue per il teorema di corrispondenza che $\pi_{Z(G)}^{-1}(\mathcal{H}_i) = H_i \trianglelefteqslant G$, ovvero si preserva la normalità dei sottogruppi, inoltre,
    segue sempre dal teorema che:
        \[ p^i = [\mathcal{G} : \mathcal{H}_i] = [G : H] = p^i
            \]
    dunque la catena esiste  e $|H_i| = p^{n-i}$ per $1\leq i \leq m$, essendo $Z(G)$ abeliano, i sottogruppi di ogni suo ordine (che esistono sempre per il \hyperref[davide]{Lemma Di Ranieri}) sono
    normali in $Z(G)$, inoltre $|Z(G)| = p^z$ (dunque si hanno sottogruppi normali di ordine $p^l$ per $l \mid z$), pertanto esiste la catena:
        \[ \{e\} = H_n < \ldots < H_m = Z(G)
        \qquad \text{con $|H_j| = p^{n-j}$, $\forall m \leq j \leq n$} 
            \]
    bisogna infine verificare che $H_j \trianglelefteqslant G$, dunque:
        \[ gH_jg^{-1} = H_j \qquad \forall g \in G
            \]
    ma $H_j \subset Z(G)$ (sta nel centro, quindi è invariante per coniugio con tutti i $g \in G$, e in particolare quelli richiesti) dunque è sempre verificata l'ultima uguaglianza.
\end{soln}


\end{document}