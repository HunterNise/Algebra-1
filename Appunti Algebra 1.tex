\documentclass[11pt]{scrartcl}
\usepackage[italian]{babel}
\usepackage[sexy]{evan}

\begin{document}
\title{Appunti Algebra 1}
\subtitle{\large\normalfont\rmfamily\scshape APPUNTI DEL CORSO DI ALGEBRA 1 TENUTO\\ DALLA PROF. DEL CORSO E DAL PROF. LOMBARDO}
\author{Diego Monaco \\ \textnormal{\href{d.monaco2@studenti.unipi.it}{d.monaco2@studenti.unipi.it}}}
\date{Anno Accademico 2022-23}
\maketitle
\newpage

\tableofcontents
\eject
\newpage

\section*{Ringraziamenti}
Federico Allegri, Pietro Crovetto, Davide Ranieri, Francesco Sorce, Leonardo Migliorini, Matteo Gori, Daniele Lapadula, Alessandro Fenu,
Leonardo Alfani.

\newpage
\section{Gruppi}
\subsection{Automorfismi di $G$}
Dato un gruppo $G$ possiamo definire l'insieme degli automorfismi di $G$ come segue:
    \[ \Aut(G) = \{\varphi : G \longrightarrow G |\, \varphi \, \text{isomorfismo}\}
        \]
si verifica facilmente che $(\Aut(G), \circ)$ è un gruppo, e in particolare $\Aut(G) \leqslant S(G)$,
ovvero il gruppo delle permutazioni di $G$. Si osserva che $id \in \Aut(G)$, $\varphi \in \Aut(G) \implies 
\varphi^{-1} \in \Aut(G)$ e $\varphi,\psi \in \Aut(G) \implies \varphi \circ \psi \in \Aut(G)$.

\begin{example}
    [Esempi di automorfismi]
    Esempi di insiemi di automorfismi:
        \begin{itemize}
            \item $\Aut(\ZZ) = \{\pm id\}$.
            \item $\Aut(\Zn) \cong \Zn^*$.
            \item $\Aut(\Z2 \times \Z2) \cong S_3$.
            \item $\Aut(\underbrace{\Zp \times \ldots \times \Zp}_{n \;\text{volte}}) \cong GL_n(\Fp)$
        \end{itemize}
\end{example}

\subsection{Automorfismi interni}
\begin{definition}
    Dato un gruppo $G$ possiamo definire l'omomorfismo di \vocab{coniugio}:
        \[ \varphi_g : G \longrightarrow G : x \longmapsto gxg^{-1}
            \]
    dove l'elemento $gxg^{-1}$ si dice \vocab{coniugato} di $g$.
\end{definition}

\begin{proposition}
    \label{prop1}
    Valgono i seguenti fatti:
    \begin{enumerate}[(1)]
        \item $\varphi_g \in \Aut(G)$, $\forall g \in G$.
        \item $\{\varphi_g | g \in G\} = \Inn(G) \trianglelefteqslant \Aut(G)$.\footnote{$\Inn(G)$ si definisce \vocab{gruppo degli automorfismi interni}.}
    \end{enumerate}
\end{proposition}

\begin{proof}
    Proviamo le due affermazioni:
        \begin{enumerate}[(1)]
            \item Per verificare che $\varphi_g$ è un automorfismo bisogna verificare che $\varphi_g$ è ben definita, ma ciò segue dalla chiusura di $g$ per l'operazione.
                Verifichiamo che sia un omomorfismo:
                    \[ \varphi_g(xy) = gxyg^{-1} = gxg^{-1}gyg^{-1} = \varphi_g(x)\varphi_g(y)
                    \qquad \forall x,y \in G 
                        \]
                ci resta da verificare che sia una bigezione. Partiamo dalla surgettività, vogliamo verificare che $\forall y \in G$, $\exists g \in G :$
                    \[ \varphi_g(x) = y
                        \]
                in tal caso basta prendere $x = gyg^{-1} \in G$. Per l'iniettività si osserva:
                    \[ \ker \varphi_g = \{x \in G | \varphi_g(x) = e\} = \{x \in G | gxg^{-1} = e \iff x = e\} = \{e\}
                        \]
                pertanto $\varphi_g$ è iniettivo.
            \item Verifichiamo che $\Inn(G) \trianglelefteqslant \Aut(G)$; mostriamo prima che $\Inn(G)$ è un sottogruppo di $\Aut(G)$, infatti: 
            $id = \varphi_e \in \Inn(G)$, $\forall g_1,g_2 \in G$ vale che $\varphi_{g_{1}} \circ \varphi_{g_{2}} = \varphi_{g_1g_2} \in \Inn(G)$, infatti:
                    \[ \varphi_{g_{1}} \circ \varphi_{g_{2}}(x) = \varphi_{g_{1}}(g_2xg_2^{-1}) = g_1g_2xg_2^{-1}g_1^{-1} = \varphi_{g_1g_2}(x)
                        \]
                infine, $(\varphi_{g})^{-1} = \varphi_{g^{-1}} \in \Inn(G)$:
                    \[ (\varphi_{g})^{-1} \circ \varphi_{g}(x) = (\varphi_{g})^{-1} (gxg^{-1}) = x \iff (\varphi_{g})^{-1} = \varphi_{g^{-1}}
                        \]
                e analogamente per l'inversa a destra. Per verificare la normalità bisogna mostrare che:
                    \[ f \circ \Inn(G) \circ f^{-1} \subseteq \Inn(G)
                    \qquad \qquad \forall f \in \Aut(G)
                        \]
                ovvero:
                    \[ f \circ \varphi_g \circ f^{-1} \in \Inn(G)
                    \qquad \qquad \forall f \in \Aut(G), \forall \varphi_g \in \Inn(G)
                        \]
                si osserva che $f \circ \varphi_g \circ f^{-1} = \varphi_{f(g)} \in \Inn(G)$, infatti:
                    \[ f \circ \varphi_g \circ f^{-1} (x) = f(\varphi_g(f^{-1} (x))) = f(g(f^{-1}(x))g^{-1}) =
                        \]\[ = f(g) f(f^{-1}(x)) f(g^{-1}) = f(g) x (f(g))^{-1} = \varphi_{f(g)}
                            \]
        \end{enumerate}
\end{proof}

\begin{remark}
    Se $G$ è abeliano, allora $\Inn(G) = \{id\}$, infatti:
        \[ gxg^{-1} = gg^{-1}x = x \qquad \qquad \forall x \in G, \forall g \in G
            \]
\end{remark}

\begin{proposition}
    Dato un gruppo $G$ si ha:
    \[\Inn(G) \cong \faktor{G}{Z(G)}\]
\end{proposition}

\begin{proof}
    Per dimostrare il teorema ci basta trovare un omomorfismo surgettivo da $G$ in $\Inn(G)$ e poi sfruttare il Primo Teorema di Omomorfismo. Sia:
        \[ \phi : G \longrightarrow \Inn(G) : g \longmapsto \varphi_g
            \]
    tale applicazione è chiaramente ben definita, ed è surgettiva per come abbiamo definito $\Inn(G)$. Verifichiamo che è un omomorfismo:
        \[ \phi(g_1g_2) = \varphi_{g_1g_2} = \varphi_{g_1} \circ \varphi_{g_2} = \phi(g_1) \circ \phi(g_2)
        \qquad \qquad \forall g \in G
            \]
        dove la penultima uguaglianza è vera per quanto visto nella dimostrazione del (2) della proposizione precedente. A questo punto, per il 
        primo teorema di omomorfismo si ha che:
            \begin{center}
            \begin{tikzcd}
                G \arrow[d, "\pi_{\ker{\phi}}" left, twoheadrightarrow] \arrow[r, "\phi", twoheadrightarrow] &\Inn(G)\\	
                \faktor {G}{\ker{\phi}} \arrow[ur, "\widesim{}"' above, sloped, anchor=center] & 
            \end{tikzcd}
            \end{center}
        dunque:
            \[ \frac{G}{\ker{\phi}} \cong \Inn(G)
                \]
        non ci resta che osservare:
            \begin{multline*}
                \ker{\phi} = \{g \in G | \phi(g) = \varphi_g = id\} = \{g \in G | gxg^{-1} = x, \forall x \in G\} = \\ 
                = \{g \in G | gx = xg, \forall x \in G\} = Z(G)
            \end{multline*}
\end{proof}

\begin{remark}
    L'isomorfismo trovato è del tipo $gZ(G) \longmapsto \varphi_g$, ricordiamo che è ben definito per il Primo Teorema di Omomorfismo.
\end{remark}

\begin{remark}
    Si ricorda che se $\faktor{G}{Z(G)}$ è ciclico, allora $G$ è abeliano (e quindi $\faktor{G}{Z(G)}$ è banale), infatti, sia:
        \[ \faktor{G}{Z(g)} = \left<gZ(G)\right>
            \]
    Presi $g_1,g_2 \in G$, si ha che $g_1Z(G) = g^{k_1}Z(G)$ e $g_2Z(G) = g^{k_2}Z(G)$, da cui:
        \[ g^{-k_1}g_1Z(G) = Z(G) \iff  g^{-k_1}g_1 \in Z(G)
            \]
    ovvero $\exists z_1 \in Z(G): g_1 = g^{k_1}z_1$ e analogamente $g_2 = g^{k_2}z_2$, da cui:
        \[ g_1g_2 = g^{k_1}z_1g^{k_2}z_2 = g^{k_1}g^{k_2}z_1z_2 = g^{k_1+k_2}z_1z_2
            \]
    e contemporaneamente:
        \[ g_2g_1 = g^{k_2}z_2g^{k_1}z_1 = g^{k_2}g^{k_1}z_2z_1 = g^{k_2 + k_1}z_2z_1 = g^{k_1+k_2}z_1z_2
            \]
    dove nell'ultimo passaggio si è sfruttato il fatto che $k_1,k_2 \in \ZZ$ e $z_1,z_2 \in Z(G)$. Da ciò segue che $G$ è abeliano.
\end{remark}

\begin{remark}
    Dunque $\Inn(G)$ ciclico $\implies \faktor{G}{Z(G)}$ ciclico $\implies G$ abeliano da cui:  
        \[ \Inn(G) \cong \faktor{G}{Z(G)} \cong \{e\}
            \]
\end{remark}

\begin{remark}
    $N \trianglelefteqslant G \iff \forall \varphi_g \in \Inn(G)$ si ha $\varphi_g(N) = N$ (o anche $\varphi_g(N) \subseteq N$). Equivalentemente, i sottogruppi
    normali di $G$ sono i sottogruppi \textbf{invarianti} per automorfismi interni (ovvero sono tali che $gNg^{-1} = N$, $\forall g \in G$). Se $N \trianglelefteqslant G$, si può considerare:
        \[ \Inn(G) \longrightarrow \Aut(N) : \varphi_g \longmapsto \varphi_{g \mid N}
            \]
    con $\varphi_{g \mid N} : N \longrightarrow N$ che è un automorfismo, infatti rimane iniettivo, la surgettività segue dal fatto che $\varphi_g(N) = N$, e infine, essendo $\varphi_g$ 
    un omomorfismo su tutti gli elementi di $G$, lo sarà in particolare anche su tutti gli elementi di $N$. Dunque quando si ha un sottogruppo normale, ogni automorfismo interno si restringe
    a un automorfismo di $N$.
\end{remark}

Abbiamo visto che i sottogruppi normali sono invarianti per automorfismi interni, possiamo generalizzare quest'idea e considerare i sottogruppi invarianti per automorfismi:

\begin{definition}
    Dato un sottogruppo $H \leqslant G$, esso si dice \vocab{caratteristico} se è invariante per automorfismi:
        \[ f(H) = H
        \qquad \qquad \forall f \in \Aut(G)
            \]
\end{definition}

Anche in questo caso basta verificare che $f(H) \subseteq H$, $\forall f \in \Aut(G)$, perché si ha anche che:
    \[ f^{-1}(H) \subseteq H
        \]
da cui si ottiene:
    \[ f(f^{-1}(H)) \subseteq f(H)
        \]

\begin{remark}
    Si osserva che se $H$ è caratteristico in $G$, allora è invariante per tutti gli automorfismi di $G$ (e quindi in particolare quelli interni), dunque
    se $H$ è caratteristico in $G$, allora è anche normale. Il viceversa è falso.
\end{remark}

\begin{remark}
    Se $H$ è caratteristico in $G$ (dunque normale), si può scrivere un'applicazione:
        \[ \Aut(G) \longrightarrow \Aut(H) : f \longmapsto f_{\mid H}
            \]
    dove $f_{\mid H}$ è un automorfismo di $H$.
\end{remark}

\begin{remark}
    Si osserva che se $H$ è l'unico sottogruppo di $G$ di un certo ordine, allora $H$ è caratteristico in $G$ (segue immediatamente dal fatto che gli automorfismi
    preservano gli ordini degli elementi).
\end{remark}

\begin{exercise}
    Il centro di un gruppo, $Z(G)$ è un sottogruppo caratteristico.
\end{exercise}
        
\begin{soln}
    Per dimostrare che $Z(G)$ è caratteristico è sufficiente far vedere che:
        \[ f(Z(G)) \subseteq Z(G)
        \qquad \forall f \in \Aut(G)
            \]
    ovvero:
        \[ f(z) \in Z(G)
        \qquad \forall f \in \Aut(G), \forall z \in Z(G)
            \]
    dunque bisogna verificare che:
        \[ gf(z) = f(z)g \qquad \forall g \in G
            \]
    poiché $f$ è un automorfismo, allora $\exists h \in G : f(h) = g$, dunque:
        \[ gf(z) = f(h)f(z) = f(hz) = f(zh) = f(z)f(h) = f(z)g \qquad \forall g \in G
            \]
\end{soln}

\begin{example}
    Sia $G = \Z2 \times \Z2 = \{(\ol 0, \ol 0),(\ol 1, \ol 0),(\ol 0, \ol 1),(\ol 1, \ol 1)\}$, $G$ ha ordine $4$ ed ha tre sottogruppi ciclici di ordine 2:
        \[ H_1 = \left<(\ol 1, \ol 0)\right> \qquad H_2 = \left<(\ol 0, \ol 1)\right> \qquad H_3 = \left<(\ol 1, \ol 1)\right>
            \] 
    ed essendo $G$ abeliano si ha $H_1,H_2,H_3 \trianglelefteqslant G$ (e quindi i sottogruppi sono invarianti per automorfismi interni). Tuttavia nessuno dei sottogruppi è caratteristico,
    infatti possiamo prendere un automorfismo non banale (e quindi non uno interno) e vedere come i sottogruppi di questo tipo non siano invarianti:
        \[ f = \begin{cases}
            (\ol 1, \ol 0) \longmapsto (\ol 1, \ol 1)\\
            (\ol 0, \ol 1) \longmapsto (\ol 0, \ol 1)
        \end{cases}
            \]
    la definizione della mappa data tuttavia non è completa, perché abbiamo stabilito solo dove vengono mandati i generatori, dobbiamo definire cosa faccia un elemento generico:
        \[ f ((\ol a, \ol b)) = af((\ol 1, \ol 0)) + bf((\ol 0, \ol 1)) = (\ol a, \ol a) + (\ol 0, \ol b) = (\ol a, \ol{a + b})
            \]
    a questo punto abbiamo definito completamente l'applicazione (rimarrebbe da verificare che $f$ sia un omomorfismo),  e si verifica facilmente che
     $f(H_1) = H_3 $ quindi $H_1 \trianglelefteqslant G$, ma non caratteristico.
\end{example}

A questo punto è facile verificare che:
    \[ \Aut(\Z2 \times \Z2) \cong S_3
        \]
infatti, ogni automorfismo del gruppo si ottiene fissando l'elemento neutro $(\ol 0, \ol 0) \longmapsto (\ol 0, \ol 0)$, quindi il numero possibile di bigezioni è al più $3!$, occorre
verificare che tutte e $6$ le funzioni sono omomorfismi. Dimostriamo invece che:
    \[ \boxed{\Aut(S_3) \cong S_3}
        \]
Per farlo, poiché $S_3$ non è abeliano, possiamo osservare che:
    \[ \Inn(S_3) \cong \faktor{S_3}{Z(S_3)} \cong S_3
        \]
in quanto l'unico elemento che commuta con tutti gli altri in $S_3$ è l'identità, quindi $Z(S_3) = \{id\} \cong \{e\}$.
Per quanto detto si ha $\Inn(S_3) \trianglelefteqslant \Aut(S_3)$ e quindi $\Aut(S_3)$ contiene una copia isomorfa di $S_3$ come sottogruppo normale, pertanto,
se verifichiamo che $|\Aut(S_3)| \leq 6$ abbiamo concluso. Sia $f \in \Aut(S_3)$, $f$ può al più scambiare i $3$ elementi di ordine $2$, d'altra parte, fissate le 
immagini di $\tau_1,\tau_2,\tau_3$\footnote{Con $\tau_i$ si intendono le trasposizioni che lasciano fisso l'elemento $i$.}, i due 
$3$-ciclci\footnote{Come si vedrà $S_3 = \left<\tau_1,\tau_2,\tau_3\right>$} sono completamente determinati,
ciò significa che si hanno al più $3!$ automorfismi, dunque:
    \[ \Aut(S_3) = \Inn(S_3) \cong S_3 \implies \Aut(S_3) \cong S_3 
        \]

\newpage
\subsection{Azione di un gruppo su un insieme}

\begin{definition}
    Sia $G$ un gruppo e $X$ un insieme, un'\vocab{azione} di $G$ su $X$ è un omomorfismo:
        \[ \varphi : G \longrightarrow S(X) : g \longmapsto \varphi_g 
            \]
    dove $\varphi_g : X \longrightarrow X : x \longmapsto \varphi_g(x)$\footnote{Alternativamente si può indicare l'immagine con $\varphi_g : x \longmapsto g \ast x$ dove il simbolo $\ast$ indica l'azione di $g$ su $x$.},
     con $\varphi_g$ bigettiva, $\forall g \in G$. Si può definire un'azione anche come:
        \[ \varphi : G \times X \longrightarrow X : (g,x) \longmapsto \varphi_g(x)
            \]
    Un'azione di $G$ su $X$ si indica con $G \circlearrowleft X$.
\end{definition}

\begin{example}
    Sia $X = G$, quindi $\varphi : G \longrightarrow S(G) : g \longmapsto \varphi_g$, con $\varphi_g$ coniugio, $\varphi$ è un'azione. Come si è visto nell'($1$)
    della \hyperref[prop1]{Proposizione 1.3} $\varphi_g$ è un automorfismo di $G$ (e quindi una bigezione), e $\varphi$ è un omomorfismo. In questo caso si ha che:
        \[ \varphi_g(x) = gxg^{-1}
            \]
\end{example}

\begin{example}
    Sia $V$ un $K$-spazio vettoriale, sia:
        \[ \varphi : K^* \longrightarrow S(V) : \lambda \longmapsto \varphi_\lambda
            \]
    con $\varphi_\lambda : V \longrightarrow V : \underline v \longmapsto \lambda \underline v$, $\varphi$ è un'azione di $K^*$ su $V$.
\end{example}

Sia $\varphi: G \longrightarrow S(X)$ un'azione, $\varphi$ definisce una relazione di equivalenza su $X$:
    \[ x \sim y \iff \exists g \in G : \varphi_g(x) = y
        \]
    ovvero due elementi sono in relazione se esiste un'applicazione $\varphi_g \in S(X)$, per cui un elemento è l'immagine dell'altro mediante 
    tale applicazione. La relazione è appunto di equivalenza, infatti: $x \sim x$, per $g = e$ si ha (essendo $\varphi$ un omomorfismo) $\varphi_e(x) = id(x) = x$, 
    $x \sim y \implies y \sim x$:
        \[ \varphi_g(x) = y \implies x = (\varphi_g(y))^{-1} = \varphi_{g^{-1}}(y)
            \]
    infine $x \sim y$, $y \sim z \implies x \sim z$, infatti si avrebbe: $\varphi_g(x) = y$, $\varphi_h(y) = z$ da cui:
        \[ z = \varphi_h(\varphi_g(x)) = \varphi_{hg}(x) \implies x \sim z
            \]

\begin{definition}
    Data la relazione di equivalenza $\sim$ si definiscono \vocab{orbite} le classi di equivalenza di $X$ rispetto alla relazione $\sim$:
        \[ \Orb(x) = \{\varphi_g(x) | g \in G\} (\subseteq X)
            \]
\end{definition}
Da cui:
    \[ X = \bigcupdot_{x \in \mathcal{R}} \Orb(x)
        \]
Con $\mathcal{R}$ insieme di rappresentanti. Un'orbita è quindi l'insieme di tutte le immagini di un elemento in un insieme, mediante tutte le possibili 
applicazioni (permutazioni) dell'insieme $\varphi(G)$.

\begin{definition}
    Per ogni $x \in X$ si dice \vocab{stabilizzatore} di $x$:
        \[ \St(x) = \{g \in G | \varphi_g(x) = x\}
            \]
\end{definition}
Cioè lo stabilizzatore è l'insieme degli elementi di $G$, che danno origine mediante $\varphi$ alle applicazioni $\varphi_g \in S(X)$, che lasciano fisso un determinato elemento.

\begin{example}
    Se $X = \RR^2$ e $G$ è il gruppo di traslazioni di vettore $\ul v = (0, l)$, allora:
        \[ \varphi : G \longrightarrow S(X) : \tau_{(0,l)} \longmapsto \tau_{(0,l)} \footnote{Si osserva che il primo $\tau_{(0,l)}$ è un elemento del gruppo $G$, mentre il secondo è un'applicazione bigettiva di $X$.}
            \]
    con:
        \[ \Orb(x,y) = \{(x,y + l) | l \in \RR\}
        \quad \text e \quad
        \St(x,y) = \{\tau_{(0,l)} | (x, y + l) = (x,y)\}= \{id\}
            \]
\end{example}

\begin{example}
    Se $X = \RR^2$ e $G$ è il gruppo delle rotazioni di centro $O$, allora:
        \[ \varphi: G \longrightarrow S(\RR^2) : r_\theta \longmapsto r_\theta
            \]
    con:
        \[ \St(x,y) =
        \begin{cases}
            \{id\} & \text{se $(x,y) \ne (0,0)$}\\
            G & \text{se $(x,y) = (0,0)$}
        \end{cases}
            \]
    e, detta $\omega$ la circonferenza di centro $O$ e raggio $\sqrt{x^2+y^2}$:
        \[ \Orb(x,y) = \{(x^{\prime},y^{\prime}) \in \RR^2 | (x^{\prime}, y^{\prime}) \in \omega\}
            \]
\end{example}

\begin{proposition}
    [$\St(x) \leqslant G$]
    Dato un gruppo $G$ e un'azione $\varphi : G \longrightarrow S(X)$, si ha che $\St(x) \leqslant G$.\footnote{In generale lo
     stabilizzatore non è un sottogruppo normale.}
\end{proposition}

\begin{proof}
    Si osserva che $e \in \St(x)$, in quanto $\varphi_e(x) = id(x) = x$, inoltre, presi $g,h \in \St(x)$, ovvero $\varphi_g (x) = \varphi_h(x) = x$, allora:
        \[ \varphi(gh) = \varphi_{gh}(x) = \varphi_g \circ \varphi_h (x) = \varphi_g(\varphi_h(x)) = \varphi_g(x) = x \implies gh \in \St(x)
            \]
    dove si ha che $ \varphi_{gh}(x) = \varphi_g \circ \varphi_h (x)$ in quanto $\varphi$ è un omomorfismo.
    Infine, preso $g \in \St(x)$, si ha $g^{-1} \in \St(x)$, infatti $\varphi_g$ è bigettiva e quindi ammette inversa:
        \[ (\varphi_g)^{-1} \circ \varphi_g (x) = x \implies (\varphi_g)^{-1}(\varphi_g(x)) = x \implies (\varphi_g)^{-1}(x) = x
            \]
    con $(\varphi_g)^{-1}(x) = (\varphi(g))^{-1}(x) = (\varphi(g^{-1}))(x) = \varphi_{g^{-1}}(x)$ e per quanto detto:
        \[ \varphi_{g^{-1}} (x) = x \implies g^{-1} \in \St(x)
            \]
\end{proof}

\pagebreak
\begin{remark}
    Sia $x \in X$ e $g,h \in G$, allora:
        \[ \varphi_g(x) = \varphi_h(x) \iff \varphi_{h^{-1}}(\varphi_g(x)) = x
            \]
    e per le proprietà di omomorfismo dell'azione $\varphi$, si ha:
        \[ \varphi_{h^{-1}}(\varphi_g(x)) = x \iff \varphi_{h^{-1}g}(x) = x \iff h^{-1}g \in \St(x)
            \]
    ovvero $g \St(x) = h \St(x)$, in quanto $\St(x) \leqslant G$ e la condizione ottenuta è esattamente quella dell'equivalenza modulo $\St(x)$,
    quindi:
    \[ \Orb(x) \longleftrightarrow \text{classi laterali di $\St(x)$ in $G$}
        \]
    cioè due elementi danno la stessa immagine se e solo se stanno nella stessa classe laterale modulo $\St(x)$, e la corrispondenza biunivoca tra orbita e classi
    laterali è data da:
    \[ g\St(x) \longmapsto \varphi_g(x) \qquad \text e \qquad h\St(x) \longmapsto \varphi_h(x)
                \]
    che è ben definita e per quanto detto all'inizio è iniettiva:
        \[ \varphi_g(x) = \varphi_h(x) \iff g\St(x) = h\St(x)
            \]
    (quindi due elementi di un'orbita sono uguali se e solo se lo sono le classi laterali dei rispettivi elementi che generano le applicazioni sono uguali modulo $\St(x)$, duqnue per ogni elemento
    dell'orbita c'è una classe laterale di $\St(x)$) e surgettiva:
        \[ \forall y \in \Orb(x), y = \varphi_g(x) \implies g\St(x) \longmapsto y
            \]
    e quindi concludiamo che il numero di classi laterali di $\St(x)$ in $G$ è lo stesso della cardinalità di $\Orb(x)$.
\end{remark}

Per quanto detto si ha:
    \[ |G| = |\St(x)| [G : \St(x)]
        \]
ma $[G : \St(x)]$ è il numero di classi laterali di $\St(x)$ in $G$, che è proprio uguale a $|\Orb(x)|$ pertanto vale la seguente:

\begin{proposition}
    \label{p:1.25}
    Sia $G$ un gruppo finito e $X$ un insieme, allora:
        \[ |G| = |\Orb(x)||\St(x)|
            \qquad \forall x \in X
            \]
\end{proposition}

\begin{remark}
    Si osserva che essendo $\St(x) \leqslant G$, allora è ovvio (per Lagrange) che $|\St(x)| \mid |G|$, tuttavia, per la proposizione precedente, si ha che:
    $|\Orb(x)| \mid |G| $ con $\Orb(x) \subseteq X$.
\end{remark}

\pagebreak
Ricordando che:
    \[ X = \bigcupdot_{x \in \mathcal R} \Orb(x)
        \]
se $|X|<+\infty$ si ha:
    \[ \colorboxed{red}{
        |X| = \sum_{x \in \mathcal{R}} |\Orb(x)| = \sum_{x \in \mathcal{R}} \frac{|G|}{|\St(x)|}
    }\]


\newpage
\subsection{Azione di coniugio}
\begin{definition}
    Si parla di \vocab{azione di coniugio}, quando si ha un'azione di $G$ su $G$ stesso:
        \[ \varphi : G \longrightarrow \Inn(G) (\trianglelefteqslant S(G)) : g \longrightarrow \varphi_g
            \]
\end{definition}

Abbiamo già osservato che è un'azione (ovvero che $\varphi$ è un omomorfismo). In questo caso:
    \[ \Orb(x) = \{\varphi_g(x) | g \in G\} = \{ gxg^{-1} | g \in G\} = C_x
        \]
dove $C_x$ prende il nome di \vocab{classe di coniugio} di $x$. Mentre:
    \[ \St(x) = \{g \in G | \varphi_g(x) = gxg^{-1} = x\} = Z_G(x)
        \]
dove $Z_G(x)$ si dice \vocab{centralizzatore} di $x$. Per quanto detto in precedenza si ha:
    \[ |G| = |C_x||Z_G(x)|
        \]
In particolare $|C_x| \mid |G|$ e :
    \[ |G| = \sum_{x \in \mathcal{R}} |C_x| = \sum_{x \in \mathcal{R}} \frac{|G|}{|Z_G(x)|}
        \]

\begin{remark}
    $C_x$ è un sottoinsieme, non un sottogruppo di $G$, poiché non c'è mai l'identità.
\end{remark}

\begin{remark}
    Osserviamo che $Z_G(x) = G \iff x \in Z(G)$, infatti la per un elemento del centro si ha che 
    $\forall g \in G$ l'elemento commuta, e dunque il suo centralizzatore è tutto il gruppo.
\end{remark}

\begin{remark}
    Per un'azione di coniugio ha che $x \in Z(G)$ se e solo se $\Orb(x) = \{x\}$ (ovvero $\varphi_g(x) = x$, $\forall g \in G$).
\end{remark}

    \[ |G| = \sum_{x \in Z(G)} \frac{|G|}{|Z_G(x)|} + \sum_{x \in \mathcal{R}\setminus Z(G)} \frac{|G|}{|Z_G(x)|}
        \]
ma, per quanto detto, se $x \in Z(G)$, allora $\displaystyle\frac{|G|}{|Z_G(x)|} = |C_x| = \{x\}$, segue dunque la relazione:
    \[ \colorboxed{red}{|G| = |Z(G)| + \sum_{x \in \mathcal{R}\setminus Z(G)} \frac{|G|}{|Z_G(x)|}}
        \]
che prende il nome di \vocab{formula delle classi} (di coniugio).

\newpage
\subsection{Applicazioni ai $p$-gruppi}
\begin{definition}
    Si definisce \vocab{$p$-gruppo} un gruppo di ordine $p^n$, con $p$ primo e $n \geq 1$.
\end{definition}

Se $G$ è un $p$-gruppo la formula delle classi diventa:
    \[ p^n = |G| = |Z(G)| + \sum_{x \in \mathcal{R}\setminus Z(G)} \frac{|G|}{|Z_G(x)|}
        \]
con $|Z(G)| = p^z$, $0 \leq z \leq n$, facciamo due osservazioni fondamentali:
    \begin{enumerate}[(1)]
        \ii Il centro di un $p$-gruppo non è mai banale, infatti, se osserviamo la formula delle classi, si ha:
            \[ p^n = |Z(G)| +  \sum_{x \in \mathcal{R}\setminus Z(G)} \frac{|G|}{|Z_G(x)|} \implies
             |Z(G)| +  \sum_{x \in \mathcal{R}\setminus Z(G)}\frac{|G|}{|Z_G(x)|} \equiv 0 \pmod p
                \]
            con $\displaystyle \frac{|G|}{|Z_G(x)|} > 1$, poiché se un elemento sta nel centro tutti gli addendi sono $1$ per quanto detto,
            viceversa deve essere che $\displaystyle \frac{|G|}{|Z_G(x)|} = p^{k_x}$, $k>0$, poiché $G$ è un $p$-gruppo, dunque:
                \[ |Z(G)| \equiv 0 \pmod p \implies |Z(G)| \geq 2
                    \]
            e quindi il centro di un $p$-gruppo non è mai banale.
        \ii Un gruppo di ordine $p^2$ è abeliano, infatti, si ha:
            \[ |G| = p^2 \implies |Z(G)| = \begin{cases}
                                                1 &\text{non può accadere per (1)} \\
                                                p &\text{no perché allora $G/Z(G)$ ciclico, ma $G$ non è abeliano}\\
                                                p^2
                                            \end{cases}
                \]
            dunque l'unica possibilità è che $Z(G) = G \iff G$ abeliano.
    \end{enumerate}

\newpage
\subsection{Teorema di Cauchy}

\begin{theorem}
    [Teorema di Cauchy]
    Dato un gruppo $G$ e un primo $p$, se $p \mid |G|$, allora $\exists x \in G : \ord_G(x) = p$. \footnote{Si considera già noto il teorema per gruppi abeliani.}
\end{theorem}

\begin{proof}
    Sia $|G| = pn$, procediamo per induzione su $n$, nel caso $n = 1$ il teorema è ovvio. Supponiamo vera la tesi per i gruppi di ordine $pm$, con 
    $1 \leq m < n$ e proviamola per $n$. Distinguiamo due casi:
        \begin{itemize}
            \item Se esiste $H \lneq G$ con $p \mid H$, ovvero $|H| = pm \implies$ vale il teorema di Cauchy per ipotesi induttiva (essendo $m<n$), quindi 
            $\exists x \in H : \ord_H(x) = p$, ma essendo $H \subset G \implies x \in G$ e quindi la tesi è vera.
            \item Se $\forall H \lneq G$ si ha $p \nmid |H|$, allora si può applicare a $G$ la formula delle classi:
                \[ pn = |G| = |Z(G)| + \sum_{x \in \mathcal{R}\setminus{Z(G)}} \frac{|G|}{|Z_G(x)|}
                    \]
                ricordando il centralizzatore di $x$ è uno stabilizzatore (e quindi un sottogruppo di $G$), si ha $p \nmid |Z_G(x)|$, e quindi:
                    \[ p \Bigm| \sum_{x \in \mathcal{R}\setminus{Z(G)}} \frac{|G|}{|Z_G(x)|}
                        \]
                da cui segue che $p \mid |Z(G)| = |G| - \sum pl_x$, per quanto premesso ($\forall H \lneq G$ si ha $p \nmid |H|$), ed essendo $Z(G) \leqslant G$, l'unica 
                possibilità è che $Z(G) = G$ e vale il teorema poiché è già stato dimostrato per il caso in cui $G$ è abeliano.
        \end{itemize}
\end{proof}

\newpage
\subsection{Azione di coniugio su un sottogruppo}
Sia $X = \{H \leqslant G\}$ e $\varphi : G \longrightarrow S(X) : g \longmapsto \varphi_g(X)$, con $\varphi_g : X \longrightarrow X : H \longmapsto gHg^{-1}$. 
    Si verifica facilmente che $\varphi$ è un omomorfismo; mostriamo invece che $\varphi_g$ è una permutazione, per l'iniettività si osserva che:
        \[ \varphi_g(H) = \varphi_g(K) \iff gHg^{-1} = gKg^{-1} \iff H = K
            \]
    mentre per la surgettività si ha che $\forall H \in X, \exists L \in X$:
        \[ \varphi_g(L) = H \iff gLg^{-1} = H \implies L = g^{-1}Hg
            \]
    inoltre si ha anche:
        \[ \Orb(H) = \{\varphi_g(H) | g \in G\} = \{gHg^{-1}| g \in G\} \quad \St(H) = \{g \in G | \varphi_g(H) = H\} = N_G(H)
            \]
    dove $\Orb(H)$ è l'insieme dei coniugati di $H$, mentre $\St(H) = N_G(H)$ prende il nome di \vocab{normalizzatore} di $H$.

\begin{remark}
    Si osserva che $N \trianglelefteqslant G$ se e solo se $\Orb(H) = \{H\} \iff N_G(H) = G$, ovvero se $H$ è sempre chiuso per coniugio in $G$.
\end{remark}

Per quanto affermato nella \hyperref[p:1.25]{Proposizione 1.25} si ha:
    \[ |G| = |\Orb(H)||N_G(H)| \implies |\Orb(H)| = \frac{|G|}{|N_G(H)|}
    \]
\begin{remark}
    Quindi in generale, dato $H \leqslant G$ si ha che $\#\{gH\} = [G:H]$ e $\#\{gHg^{-1}\} = [G : N_G(H)]$.
\end{remark}

\begin{remark}
    [Sulla definizione di sottogruppo normale]
    I sottogruppi normali possono essere ridefiniti nella maniera seguente, $H \trianglelefteqslant G$ se e solo se:
        \[ H = \bigcupdot_{h \in H} C_h
            \]
    cioè un sottogruppo è normale se e solo se è l'unione delle classi di coniugio dei suoi elementi. Infatti:
        \[ H \trianglelefteqslant G \iff ghg^{-1} \in H \qquad \forall h \in H, \forall g \in G
            \]
    che equivale a:
        \[ C_h = \{ghg^{-1} | h \in H\} \subseteq H  \quad \forall h \in H \implies \bigcupdot_{h \in H} C_h \subseteq H
            \]
    d'altra parte se $H$ è normale è chiuso per coniugio, ovvero il coniugio di ogni suo elemento è ancora in $H$
    ($ghg^{-1} = h^{\prime}$, $\forall h \in H$) e in particolare ciò significa che:
        \[ H \subseteq \bigcupdot_{h \in H} C_h
            \] 
    
\end{remark}

\newpage
\subsection{Teorema di Cayley}

\begin{theorem}
    \label{p:Cauchy}
    Ogni gruppo è isomorfo ad un sottogruppo di un gruppo di permutazioni. In particolare, se $|G| = n$, allora 
    $G$ è isomorfo a un sottogruppo di $S_n$.
\end{theorem}

\begin{proof}
    Definiamo la mappa:
        \[ \lambda : G \longrightarrow S(G) : g \longmapsto \varphi_g
            \]
    con $\varphi_g : G \longrightarrow G : x \longmapsto gx$, l'applicazione $\lambda$ prende il nome di \vocab{rappresentazione regolare a sinistra} di $G$, si 
    vuole dimostrare che $\lambda$ è un omomorfismo iniettivo.
    Osserviamo innanzitutto che $\lambda$ è ben definita, cioè $\varphi_g \in S(G)$, infatti $\varphi_g$ è iniettiva (segue dalle leggi di cancellazione) e 
    surgettiva, perché $\forall y \in G$, $\exists g^{-1}y \in G : \varphi_g(g^{-1}y) = y$. Verifichiamo che $\lambda$ è un omomorfismo:
        \[ \lambda(g_1g_2) = \varphi_{g_1g_2}
            \]
    con $\varphi_{g_1g_2} (x) = \varphi_{g_1} \circ \varphi_{g_2} (x)$, $\forall x \in G$, e quindi:
        \[ \lambda(g_1g_2) = \lambda(g_1) \lambda(g_2)
        \qquad \forall g_1,g_2 \in G
            \]
    infine, per l'iniettività si ha che:
        \[ \ker \lambda = \{g \in G | \lambda(g) = \varphi_g = id = \varphi_e\} = \{e\}
            \]
    da ciò segue che $G \cong \Imm(G) \leqslant S(G)$, e se $|G| = n$ si ha che $\Imm(G) \leqslant S_n$.
\end{proof}

\begin{remark}
    In generale, dato $G = \left\{g_1 = e,g_2, \ldots, g_n\right\}$ e $\lambda : G \longrightarrow S(G) \cong S_n$, si ha che:
        \[ g_1 = e \longmapsto \lambda_{g_1} \qquad \text{con} \qquad \lambda_{g_1}: G \longrightarrow G : g_i \longmapsto g_i
            \]
        \[ g_2 \longmapsto \lambda_{g_2} \qquad \text{con} \qquad \lambda_{g_2} : G \longrightarrow G : x \longmapsto g_2x \longmapsto g_2^2x \longmapsto \ldots \longmapsto g_2^{k-1}x
            \]
        con $k = \ord_G(g_2)$. $\lambda_{g_2}$ può essere rappresentata mediante la notazione dei cicli:
            \[ (x,g_2x,\ldots,g_2^{k-1}x)
                \]
        preso poi $y \not\in \lambda_{g_2}(G)$, si ha analogamente:
        \[ (y,g_2y,\ldots,g_2^{k-1}y)
            \]
\end{remark}

\begin{example}
    Nel caso in cui $G = \Z8$ consideriamo l'azione:
        \[ \lambda : G \longrightarrow S(\Z8) \cong S_8\footnote{Perché appunto $S(\Z8)$ è l'insieme di permutazioni di un insieme di $8$ elementi.} : \ol a \longmapsto \lambda_a
            \]
    che, per quanto visto  genera ad esempio le applicazioni:\footnote{Per $+$ si intende la somma modulo $8$.}
            \begin{align*}
            \begin{array}{ccc}
            1 &\longmapsto \lambda_1 : X \longrightarrow X : a \longmapsto 1 + a \implies &(0,1,\ldots,7) \\
            2 &\longmapsto \lambda_2 : X \longrightarrow X : a \longmapsto 2 + a \implies &(0,2,4,6)(1,3,5,7)\\
            4 &\longmapsto \lambda_4 : X \longrightarrow X : a \longmapsto 4 + a \implies &(0,4)(1,5)(2,6)(3,7)\\		
            \end{array}
            \end{align*}
    che permutano gli elementi di $X$ secondo i cicli trovati.
\end{example}

\nopagebreak

\begin{definition}
    Un'azione $\lambda$ si dice \vocab{fedele} se è iniettiva.
\end{definition}
Ad esempio l'azione di rappresentazione regolare a sinistra è fedele:
    \[ \ker \lambda = \{g \in G | \lambda(g) = id\} = \{g \in G | \lambda_g(e) = e\} = \{g \in G | ge = e\} = \{e\}
        \]
da cui $\lambda$ fedele.

\begin{remark}
    Esiste anche un'applicazione $\rho : G \longrightarrow S(G) (\cong S_n)$, ($n = |G|$), detta azione di \vocab{rappresentazione regolare a destra}, con:
        \[ g \longmapsto \rho_g : x \longmapsto xg^{-1}
            \]
\end{remark}

\begin{lemma}
    \label{davide}
    Sia $G$ un gruppo abeliano di ordine $n$, allora $\forall d\mid n, \exists H \leqslant G : |H| = d$.
    \footnote{La dimostrazione non è stata fatta durante il corso, ma è stata comunque aggiunta per completezza.}
\end{lemma}

\begin{proof}
    Si consideri innanzitutto il caso $d=p^k$, $p$ primo, e mostriamolo per induzione:
    per $k=1$ la tesi è equivalente al \hyperref[p:Cauchy]{Teorema di Cauchy} (anche solo per i gruppi abeliani).
    Supponiamo la tesi per $k-1$. Poiché in particolare $p\mid |G|$ scegliamo un sottogruppo $H$ di $G$ di ordine $p$;
    tale sottogruppo è normale poiché $G$ è abeliano. $p^{k-1}\mid |G/H|\implies$ per ipotesi induttiva $\exists K\leqslant G,\ |K|=p^{k-1}$. \\
    Prendendo la controimmagine di $K$ tramite la proiezione al quoziente troviamo il sottogruppo di G cercato. A questo punto possiamo scrivere in generale
    $d=p_1^{k_1}\ldots p_s^{k_s}$; per ogni $i$ troviamo sottogruppi $H_i$ di ordini $p_i^{k_i}$ (tutti normali). Si ha quindi che $H_1H_2\leqslant G$ per normalità,
    inoltre $|H_1\cap H_2|=1$ poiché l'ordine di un elemento in tale intersezione deve dividere $(p_1^{k_1}, p_2^{k_2})=1$. Pertanto $|H_1H_2|=p_1^{k_1}p_2^{k_2}$.
    Ragionando per induzione otteniamo che il sottogruppo $H_1\ldots H_k$ ha ordine $d$ come voluto.
\end{proof}

\newpage

\begin{exercise}
    Sia $G$ un gruppo, se $|G| = p^n$, allora esiste:
        \[ \{e\} = H_n < H_{n-1} < \ldots < H_1 < G
            \]
    con $H_i \trianglelefteqslant G$ e $|H_i| = p^{n-i}$, $\forall i \in \{1,\ldots,n\}$.
\end{exercise}

\begin{soln}
    Procediamo per induzione su $n$, per $n = 1$ è ovvio, infatti si ha $H_1 = \{e\} \trianglelefteqslant G$.
    Supponiamo la tesi vera $\forall 1 \leq k \leq n-1$, osserviamo che $G$ è un $p$-gruppo, pertanto il suo centro non è banale:
        \[ |Z(G)| = p^z \qquad z\geq 1
            \]
    sia $\mathcal{G} = \faktor{G}{Z(G)}$, essendo $|G/Z(G)| < p^n$ (perché deve essere $|Z(G)| \geq p$), allora vale l'ipotesi induttiva, dunque
    $|\mathcal{G}| = p^m$, con $m = n-z\ (<n)$, allora esiste:
        \[ \mathcal{H}_m = \{e_{\mathcal{G}}\} < \mathcal{H}_{m-1} < \ldots < \mathcal{H}_1 < \mathcal{G}
            \]
    con $|\mathcal{H}_i| = p^{m-i}$ e $\mathcal{H}_i \trianglelefteqslant \mathcal{G}$. Data la proiezione al quoziente:
        \[ \pi_{Z(G)} : G \longrightarrow \mathcal{G}
            \]
    per il Teorema di Corrispondenza dei sottogruppi, esiste una bigezione tra i sottogruppi di $\faktor{G}{Z(G)}$ e i sottogruppi di
    $G$ che contengono $Z(G)$, la quale preserva normalità e indice del sottogruppo, pertanto preso $\mathcal{H}_i \leqslant \faktor{G}{Z(G)}$ è sufficiente
    applicare $\pi_{Z(G)}^{-1}$ alla catena scritta sopra, e si trova:
        \[ Z(G) = \pi_{Z(G)}^{-1}(\mathcal{H}_{m}) < \ldots < \pi_{Z(G)}^{-1}(\mathcal{H}_{1}) < \pi_{Z(G)}^{-1}(\mathcal{G}) (= G)
            \]
    Segue per il teorema di corrispondenza che $\pi_{Z(G)}^{-1}(\mathcal{H}_i) = H_i \trianglelefteqslant G$, ovvero si preserva la normalità dei sottogruppi, inoltre,
    segue sempre dal teorema che:
        \[ p^i = [\mathcal{G} : \mathcal{H}_i] = [G : H] = p^i
            \]
    dunque la catena esiste  e $|H_i| = p^{n-i}$ per $1\leq i \leq m$, essendo $Z(G)$ abeliano, i sottogruppi di ogni suo ordine (che esistono sempre per il \hyperref[davide]{Lemma Di Ranieri}) sono
    normali in $Z(G)$, inoltre $|Z(G)| = p^z$ (dunque si hanno sottogruppi normali di ordine $p^l$ per $l \mid z$), pertanto esiste la catena:
        \[ \{e\} = H_n < \ldots < H_m = Z(G)
        \qquad \text{con $|H_j| = p^{n-j}$, $\forall m \leq j \leq n$} 
            \]
    bisogna infine verificare che $H_j \trianglelefteqslant G$, dunque:
        \[ gH_jg^{-1} = H_j \qquad \forall g \in G
            \]
    ma $H_j \subset Z(G)$ (sta nel centro, quindi è invariante per coniugio con tutti i $g \in G$, e in particolare quelli richiesti) dunque è sempre verificata l'ultima uguaglianza.
\end{soln}

\newpage
\subsection{Permutazioni}
Ricordiamo brevemente che:

\begin{definition}
    Dato un insieme $X$ si definsce \vocab{permutazione} un'applicazione bigettiva di $X$ in se stesso.
\end{definition}

Indichiamo con $S(X)$ il gruppo delle permutazioni di $X$ e con $S_n$ il gruppo delle permutazioni di un insieme di cardinalità $n$, che 
per semplicità indichiamo con $\{1,\ldots,n\}$.
Le permutazioni si possono indicare in vari modi, ad esempio, preso $\sigma \in S_{12}$ si può rappresentare mediante la matrice di permutazione:
    \[ \sigma = \left(\begin{array}{cccccccccccc}
        1 & 2 & 3 & 4 & 5 & 6 & 7 & 8 & 9 & 10 & 11 & 12 \\
        3 & 2 & 4 & 5 & 1 & 9 & 8 & 7 & 6 & 12 & 11 & 10 
        \end{array}\right)
        \]
o anche con la notazione dei cicli:
    \[ \sigma = \cycle{1,3,4,5} \cycle{6,9} \cycle{7,8} \cycle{10,12}
            \]
ogni ciclo prende il nome di \vocab{$k$-ciclo} (dove $k$ indica la sua lunghezza), come si osserva i cicli di lunghezza $1$ sono stati omessi,
 in quanto lasciano fissi gli elementi, inoltre, i $2$-cicli prendono il nome di \vocab{trasposizioni}. Formalmente, sia $\sigma \in S_n$ una permutazione di un insieme di $n$ elementi,
 possiamo considerare l'insieme $X$, con $|X| = n$, il gruppo $G = \left<\sigma\right>$ e definire l'azione:
    \[ \varphi : G = \left<\sigma\right> \longrightarrow S(X) \cong S_n : \sigma \longmapsto \sigma
        \]
con $\sigma \in S_n$ e $\sigma : i \longmapsto \sigma(i)$. Osserviamo quindi che:
    \[ \Orb(x) = \{\sigma(x) | \sigma \in \left<\sigma\right>\} = \{\sigma^l(x) | l \in \NN\} = \{x, \sigma(x), \sigma^2(x),\ldots,\sigma^{m-1}(x)\}
        \]
con $|\Orb(x)| = m_x$, con $m_x = \min\{k > 0 | \sigma^k(x) = x\}$, perché se $\sigma^k(x) = x$, allora $\sigma^{k+1}(x) = \sigma(x)$, pertanto, sia $k \in \NN$
tale che $\sigma^k(x) \in \{x,\ldots,\sigma^{k-1}(x)\}$, allora $\exists h:$
    \[ \sigma^k(x) = \sigma^h(x)
    \qquad \text{con $0 \leq h < k$}
        \]
Dunque vale che $\sigma^{k-h}(x) = x \in \{x,\ldots,\sigma^{k-1}(x)\}$ e per la minimalità di $k$ si ha che $h = 0$.
L'azione di $\left<\sigma\right>$ su $X$ divide $X$ in orbite e su ogni orbita $\sigma$ agisce ciclicamente (ovvero $\sigma(\Orb(x)) = \Orb(x)$).

\begin{definition}
    Si dice \vocab{ciclo} di $\sigma \in S_n$ l'orbita di un elemento $x \in \{1,\ldots,n\}$ vista come insieme ordinato:
        \[ (x,\sigma(x),\ldots,\sigma^{m_x-1}(x))
            \]
\end{definition}

\begin{remark}
    Un ciclo di lunghezza $k$ (un $k$-ciclo) ha $k$ scritture distinte, in quanto possiamo scegliere arbitrariamente il primo elemento.
\end{remark}

\begin{remark}
    Data $\sigma \in S_n$, essa è determinata dalle immagini di $\{1,\ldots,n\}$, dunque è determinata dai suoi cicli.
\end{remark}

\begin{example}
    Presa ad esempio $\sigma \in S_{10}$:
        \[ \sigma = \cycle{1,2,3}\cycle{4,5}\cycle{6,7,8,9}
            \]
    chiamiamo i suoi cicli:
        \[ \sigma_1 = \cycle{1,2,3} \qquad \sigma_2 = \cycle{4,5} \qquad \sigma_3 = \cycle{6,7,8,9}
            \]
    dove appunto $\sigma_1,\sigma_2,\sigma_3 \in S_{10}$ e:
        \[ \sigma = \sigma_1 \circ \sigma_2 \circ \sigma_3
            \]
\end{example}

\begin{definition}
    Una permutazione si dice \vocab{ciclica} se ha un unico ciclo (orbita) non banale.
\end{definition}

\begin{remark}
    Si osserva che:
        \begin{itemize}
            \item Cicli disgiunti commutano.
            \item L'ordine di una permutazione ciclica è la lunghezza del suo ciclo:
                    \[ \sigma = (x_1,\ldots,x_k) \implies \ord \sigma = k
                        \]
                quindi $\sigma^k = id$ e se $d < k$, allora $\sigma^d(x_1) = x_{d+1} \ne x$.
        \end{itemize}
\end{remark}

\begin{proposition}
    [Struttura Delle Permutazioni]
    \label{perm}
    Ogni permutazione si scrive in modo unico (a meno dell'ordine e della scrittura di cicli) come prodotto di cicli disgiunti,
     ovvero come composizione di permutazioni cicliche che agiscono su insiemi disgiunti.
\end{proposition}

\begin{proof}
    I cicli della permutazione sono univocamente determinati in quanto orbite della permutazione, sappiamo che ogni permutazione si scrive come prodotto dei suoi cicli,
    e per concludere basta osservare che i cicli disgiunti commutano.
\end{proof}

\begin{remark}
    Si osserva che l'unicità della scrittura di una permutazione vista nella \hyperref[perm]{Proposizione 1.50} è effettivamente valida solo nel caso di cicli disgiunti, infatti, prendendo ad esempio:
        \[ \sigma = \cycle{1,2}\cycle{2,4} \in S_4 \qquad \text{con}\qquad \sigma_1 = \cycle{2,4} \quad \text e \quad \sigma_2 = \cycle{1,2}
            \]
    non essendo $\sigma_1,\sigma_2$ cicli disgiunti, si osserva che $\sigma_2 \circ \sigma_1 = \cycle{2,4,1}$ e quindi $\sigma$ era in realtà un $3$-ciclo, e la sua fattorizzazione è unica come tale 
    (mentre non era unica come prodotto di cicli non disgiunti). 
\end{remark}

\pagebreak

\begin{corollary}
    $S_n$ è generato dalle permutazioni cicliche.
\end{corollary}

\begin{proof}
    Segue immediatamente dal fatto che ogni permutazione si ottiene mediante composizione di permutazioni cicliche. 
\end{proof}

\begin{example}
    Per esempio, preso $S_4$, le permutazioni possibili sono cicli del tipo:
        \[ id \qquad \cycle{a,b} \qquad \cycle{a,b,c} \qquad \cycle{a,b,c,d} \qquad \cycle{a,b}\cycle{c,d}
            \]
    per contare il numero di $2$-cicli, ci basta scegliere $2$ elementi dell'insieme in $\binom{4}{2}$ modi e poi considerare tutti i possibili 
    riordinamenti ciclici (dove la scelta del primo elemento è arbitraria), e ciò può essere fatto in $\frac{2!}{2}$ modi, per un totale di:
        \[ \binom{4}{2}\frac{2!}{2} = 6
            \]
    e ragionando analogamente per i $3$-cicli e i $4$-cicli si ottiene:
        \[ \binom{4}{3}\frac{3!}{3} = 8 \qquad \text e \qquad \binom{4}{4}\frac{4!}{4} = 6
            \]
    infine, per quanto riguarda le permutazioni ottenute dalla composizione di due $2$-cilci, possiamo scegliere e permutare due coppie di elementi, come 
    nei casi precedenti, tuttavia, essendo i cicli disgiunti commutanto (banalmente perché lasciano fissi gli altri elementi del dominio), quindi bisogna anche 
    dividere per il numero di scambi per i cicli della stessa lunghezza, ovvero $2!$ dunque:
        \[ \binom{4}{2}\frac{2!}{2}\binom{2}{2}\frac{2!}{2} \cdot \frac{1}{2!} = 3
            \]
    e dal conteggio delle permutazioni di $S_4$ divise per cicli di diversa lunghezza si ottiene: $6+8+6+3+1 = 24 = |S_4|$.
\end{example}

\begin{remark}
    Quanto visto nell'esempio precedente può essere generalizzato ottenendo:
        \[ \#\{\sigma \in S_n | \sigma\,\text{è un $k$-ciclo}\} = \binom{n}{k}\frac{k!}{k} = \binom{n}{k}(k-1)!
            \]
\end{remark}

\begin{example}
    Per quanto detto risulta semplice ad esempio calcolare:
        \[ \#\{\sigma \in S_{20} | \text{$\sigma$ si fattorizza in cicli del tipo $2+2+2+4+5+5$}\}
            \]
    applicando quanto detto nell'osservazione pretendente si trovano:
        \[ \frac{\binom{20}{2}\binom{18}{2}\binom{16}{2}1!1!1!}{3!} \cdot \binom{14}{4}3! \cdot \frac{\binom{10}{5}\binom{5}{5}4!4!}{2!}
            \]
\end{example}

\begin{proposition}
    [Ordine Di Una Permutazione]
    Data $\sigma \in S_n$ con $\sigma = \sigma_1\ldots\sigma_k$, con $\sigma_i$ cicli disgiunti, allora:
        \[ \ord \sigma = [\ord \sigma_1,\ldots, \ord\sigma_k]
            \]
\end{proposition}

\begin{proof}
    Sia $\sigma_i$ un $l_i$-ciclo, ovvero $\ord\sigma_i = l_i$, vogliamo dimostrare che:
        \[ \ord\sigma = [l_1,\ldots,l_k] = d
            \]
    osserviamo che $\sigma^d = (\sigma_1\ldots\sigma_k)^d = \sigma_1^d\ldots\sigma_k^d$, in quanto i cicli $\sigma_i$ sono disgiunti (pertanto commutano),
    essendo $d = [l_1,\ldots,l_k] \implies d \mid l_i$, $\forall \in \{1,\ldots,k\}$, pertanto:
        \[ \sigma^d = \sigma_1^d\ldots\sigma_k^d = id \implies \ord \sigma = m \mid d
            \]
    d'altra parte, si ha che:
        \[ \sigma^m = \sigma_1^m\ldots\sigma_k^m = id \iff \sigma_i = id, \forall i \in\{1,\ldots,k\}
            \]
    dunque $\ord \sigma_i = l_i \mid m$, $\forall i \in\{1,\ldots,k\}$, ovvero $[l_1,\ldots,l_k] \mid m$ da cui si conclude che $m = [l_1,\ldots,l_k]$.
\end{proof}

\begin{proposition}
    \label{trasp}
    Le trasposizioni generano $S_n$, $\forall n \geq 3$.
\end{proposition}

\begin{proof}
    Per dimostrare l'affermazione bisogna mostrare che ogni permutazione è prodotto di trasposizioni (in generale non disgiunte).
    Poiché ogni permutazione, per quanto affermato nella \hyperref[perm]{Proposizione 1.50}, è il prodotto di cicli (permutazioni cicliche) disgiunti,
    è sufficiente mostrare che i cicli sono tutti prodotto di trasposizioni, infatti si può osservare che:
        \[ \cycle{1,\ldots,k} = \cycle{1,k}\cycle{1,k-1}\ldots\cycle{1,2}
            \]
    dove l'uguaglianza è tra funzioni, quindi ci basta mostrare che danno la stessa immagine. Se $i>k$, allora entrambe le funzioni mandano $i \longmapsto i$, se 
    $i \leq k$, allora la funzione a sinistra manda $i \longmapsto i+1$ e $k \longmapsto 1$, quella a destra lascia fisso $i$ fino al ciclo $\cycle{1,i}$ che manda $i \longmapsto 1 \longmapsto i+1$
    che rimane fisso in $i+1$, mentre $k \longmapsto \ldots \longmapsto 1$.
\end{proof}

\begin{remark}
    La scrittura di una permutazione come prodotto di trasposizioni non è unica. Ad esempio in $S_4$:
        \[ \sigma = \cycle{1,2}\cycle{2,4} = \cycle{1,2}\cycle{3,4}\cycle{3,4}\cycle{2,4}
            \]
\end{remark}

La seguente proposizione ci mostra invece che è fissata la parità della decomposizione in trasposizioni, cioè se $\sigma$ si compone come prodotto di $m$ trasposizioni,
ogni altra decomposizione come prodotto di trasposizioni ha un numero di trasposizioni con la stessa parità.

\begin{proposition}
    L'applicazione:
        \[ sgn : S_n \longrightarrow \{\pm1\} : \sigma \longmapsto sgn(\sigma) = \prod_{1 \leq i < j \leq k} \frac{\sigma(i) - \sigma(j)}{i - j}
            \]
    è un omomorfismo di gruppi. Inoltre, se $\sigma$ è una trasposizione, allora $sgn(\sigma) = -1$.
\end{proposition}

\begin{proof}
    Osserviamo inizialmente che $sgn$ è ben definita cioè:
        \[ sgn(\sigma) = \prod_{1 \leq i < j \leq k} \frac{\sigma(i) - \sigma(j)}{i - j} \in \{\pm 1\}
            \]
    al denominatore del prodotto vi sono tutte le possibili coppie $i - j$ e anche al numeratore poiché $\sigma$ è bigettiva, l'unica cosa che 
    può cambiare è l'ordine (ovvero potrebbe comparire $i - j$ al numeratore e $j - i$ al denominatore), quindi $sgn(\sigma) \in \{\pm 1\}$. Mostriamo che $sgn$ 
    è un omomorfismo:
        \[ sgn(\sigma \circ \tau) = \prod_{i < j}\frac{\sigma(\tau(i)) - \sigma(\tau(j))}{\tau(i) - \tau(j)} \frac{\tau(i) - \tau(j)}{ i - j}
            \]
    da cui:
        \[ \prod_{i < j}\frac{\sigma(\tau(i)) - \sigma(\tau(j))}{\tau(i) - \tau(j)}\frac{\tau(i) - \tau(j)}{ i - j} =
        \underbrace{\prod_{i < j} \frac{\sigma(\tau(i)) - \sigma(\tau(j))}{\tau(i) - \tau(j)}}_{sgn(\sigma)} \underbrace{\prod_{i < j}\frac{\tau(i) - \tau(j)}{ i - j}}_{sgn(\tau)}
        \qquad \forall \sigma,\tau \in S_n
            \]
    Ci resta da verificare che il segno di una trasposizione è $-1$. Sia $\sigma = \cycle{a,b}$, analizziamo il segno delle varie coppie, distinguiamo le seguenti possibilità:
    \begin{itemize}
        \item $\{i,j\} \cap \{a,b\} = \emptyset $, in tal caso $\sigma(i) = i, \sigma(j) = j \implies \frac{\sigma(i) - \sigma(j)}{i - j} = 1$.
        \item $\{i,a\}$ (o $\{i,b\}$), in tal caso $\frac{\sigma(i) - \sigma(a)}{i - a} = \frac{i - b}{i - a}$, però vi è anche $\frac{\sigma(i) - \sigma(b)}{i - b} = \frac{i - a}{i - b}$ e quindi 
            il fattore dà $1$.
        \item Infine, nel caso in cui $\{i,j\} = \{a,b\}$ si ha:
            \[ \frac{\sigma(a) - \sigma(b)}{a - b} = \frac{b - a}{a - b} = -1
                \]
    \end{itemize}
    Dunque si conclude che $sgn(\cycle{a,b}) = -1$.
\end{proof}

\begin{remark}
    La proposizione appena vista dimostra quanto detto sopra, ovvero:
        \[ \sigma = \tau_1 \ldots \tau_m \qquad \text{con $\tau_i$ trasposizione}
            \]
    allora $sgn(\sigma) = \prod_{1\leq i \leq m}sgn(\tau_i) = (-1)^m$.
\end{remark}

\begin{definition}
    Una permutazione $\sigma \in S_n$ si dice \vocab{pari} se $sgn(\sigma) = 1$, \vocab{dispari} se $sgn(\sigma) = -1$.
\end{definition}

\begin{definition}
    Dato l'omomorfismo $sgn : S_n \longrightarrow \{\pm 1\}$, si definisce \vocab{gruppo alterno}:
        \[ \mathcal{A}_n = \ker sgn = \{\sigma \in S_n |\, \text{$\sigma$ è pari}\}
            \]
\end{definition}

\begin{remark}
    Si osserva che $\mathcal{A}_n \trianglelefteqslant S_n$, $|\mathcal{A}_n| = \frac{n!}{2}$ e $\faktor{S_n}{\mathcal{A}_n} \cong \{\pm 1\}$.
\end{remark}

\begin{remark}
    Per quanto detto nella \hyperref[trasp]{Proposizione 1.57}, un $k$-ciclo si può scrivere nella forma:
        \[ \cycle{1,\ldots,k} = \underbrace{\cycle{1,k}\cycle{1,k-1}\ldots\cycle{1,2}}_{k-1 \, \text{trasposizioni}}
            \]
    dunque un $k$-ciclo è pari se $k \equiv 0 \pmod 2$, dispari se $k \equiv 1 \pmod 2$.
\end{remark}

\pagebreak

\subsection{Classi di coniugio in $S_n$}
\begin{theorem}
    Due permutazioni in $S_n$ sono coniugate se e solo se hanno la stessa decomposizione in cicli disgiunti.
\end{theorem}

\begin{proof}
    Mostriamo le due implicazioni:
        \begin{itemize}
            \item Presa $\sigma = \cycle{a_1,\ldots,a_k}$ e $\tau \in S_n$, vogliamo dimostrare che $\tau\circ\sigma\circ\tau^{-1}$ è ancora un $k$-ciclo.
                Sia $\tau(a_i) = b_i$, allora $\tau\sigma\tau^{-1} = \cycle{b_1,\ldots,b_k}$, con $b_i \ne b_j$, $\forall i \ne j$, poiché $\tau$ è bigettiva; verifichiamo
                l'uguaglianza mostrando che le due funzioni coincidono per tutti gli elementi. Si osserva che nel ciclo a destra accade semplicemente che $b_i \longmapsto b_{i+1}$, a sinistra invece:
                    \[ b_i \xmapsto{\tau^{-1}} a_i \xmapsto{\sigma} a_{i+1} \xmapsto{\tau} b_{i+1} \qquad \forall i \in \{1,\ldots,k\}
                        \]
                Se, invece, $x \ne b_i$, a sinistra si ha $\tau\sigma\underbrace{\tau^{-1}(x)}_{\ne a_1,\ldots,a_k}$ (ciò poiché non si parte da alcun $b_i$), quindi $\sigma(\tau^{-1}(x)) = \tau^{-1}(x)$, e quindi
                $\tau \circ \tau^{-1} (x) = x$; a destra invece, non essendo $x$ alcun $b_i$ viene lasciato fisso, ciò conclude che le due funzioni sono uguali e che quella a sinistra è quindi un $k$-ciclo.
            \item Mostriamo ora che due permutazioni con la stessa fattorizzazione in cicli disgiunti sono coniugate. Siano:
                \[ \sigma = \cycle{a_1, \ldots, a_l}\cycle{b_1, \ldots, b_s} \ldots \cycle{z_1, \ldots, z_t}
                    \]\[ \rho = \cycle{a_1^{\prime}, \ldots, a_l^{\prime}}\cycle{b_1^{\prime}, \ldots, b_s^{\prime}} \ldots \cycle{z_1^{\prime}, \ldots, z_t^{\prime}}
                        \]
                per dimostrare la tesi è sufficiente trovare $\tau \in S_n$ tale che $\tau\circ\sigma\circ\tau^{-1} = \rho$. Scegliamo $\tau$ definita da:
                    \[ \tau(a_i) = a_i^{\prime}, \tau(b_i) = b_i^{\prime}, \ldots, \tau(z_i) = z_i^{\prime}
                        \]
                ed eventualmente si aggiungono altri elementi. Verifichiamo allora che $\tau\circ\sigma\circ\tau^{-1} = \rho$, consideriamo (WLOG) il primo ciclo:
                    \[ a_i^{\prime} \xmapsto{\tau^{-1}} a_i \xmapsto{\sigma} a_{i+1} \xmapsto{\tau} a_{i+1}^{\prime}
                        \]
                e quindi $a_i^{\prime} \longmapsto a_{i+1}^{\prime}$, pertanto $\tau\circ\sigma\circ\tau^{-1}$ e $\rho$ coincidono sempre.
        \end{itemize}
\end{proof}

\begin{example}
    In $S_5$ le classi di coniugio di $\sigma = \cycle{1,2}\cycle{3,4}$ sono $C_\sigma = \{\cycle{a,b}\cycle{c,d} \in S_5\}$, con:
        \[ \#C_\sigma = \frac{\binom{5}{2}\binom{3}{2}1!1!}{2!} = 15
            \]
    e da ciò si ricava anche che:
        \[ \#Z_{S_5}(\sigma) = \frac{|S_5|}{|C_\sigma|} = \frac{5!}{15} = 8
            \]
\end{example}

\begin{example}
    Sia $\sigma = \cycle{3,5}\cycle{14} \in S_5$ e sia $\rho = \cycle{1,2}\cycle{3,4}$, cerchiamo $\tau \in S_5$ tale che:
        \[ \tau\circ\sigma\circ\tau^{-1} = \rho
            \]
    si può scegliere $\tau = \cycle{1,3}\cycle{2,5}$, da cui:
        \[ \cycle{1,3}\cycle{2,5}\cycle{3,5}\cycle{14}\cycle{1,3}\cycle{2,5} = \cycle{1,2}\cycle{3,4} = \rho
            \]
\end{example}

\begin{corollary}
    \label{c:1.68}
    Valgono i seguenti fatti:
        \begin{enumerate}[(1)]
            \item Il numero di classi di coniugio in $S_n$ è uguale al numero di partizioni di $n$.
            \item Se $H \leqslant S_n$, allora $H \trianglelefteqslant S_n$ se e solo se contiene tutte le permutazioni di un certo tipo o nessuna.
        \end{enumerate}
\end{corollary}

\pagebreak

\subsection{Prodotto diretto}
Ricordiamo brevemente che se $G_1$ e $G_2$ sono gruppi, allora l'insieme $G_1 \times G_2$ con l'operazione fatta componente per componente prende il nome di \vocab{prodotto diretto}.

\begin{example}
    Presi ad esempio $\Z7$ e $S_4$, si ha $\Z7 \times S_4$, con $\sigma = (\ol 1, \cycle{1,2,3})$ e $\rho = (\ol 4, \cycle{1,4,2,4})$ in $\Z7 \times S_4$ e l'operazione:
        \[ \sigma \cdot \rho = (\ol 1 + \ol 4, \cycle{1,2,3} \circ \cycle{1,4,2,3}) = (\ol 5, \cycle{1,4,3,2})
            \]
\end{example}

\begin{remark}
    Si ricordano i seguenti fatti:
    \begin{itemize}
        \item Se $H,K \leqslant G$ in generale $HK$ non è un sottogruppo, ma $HK \leqslant G \iff HK = KH$. Ovviamente se uno tra $H$ e $K$ è normale in $G$, allora questo
        è sempre vero.
        \item $H \times K \leqslant G \times G$.
    \end{itemize}
\end{remark}

\begin{lemma}
    \label{l:1.71}
    Siano $H,K \trianglelefteqslant G$ e $H \cap K = \{e\}$, allora $hk = kh$, $\forall h \in H$, $\forall k \in K$.
\end{lemma}

\begin{proof}
    Preso $hkh^{-1}k^{-1}$, si ha:
        \[ hkh^{-1}k^{-1} = \underbrace{\underbrace{(hkh^{-1})}_{= k^{\prime}}k^{-1}}_{\in K} = \underbrace{h\underbrace{(kh^{-1}k^{-1})}_{= h^{\prime}}}_{\in H}
            \]
    dunque $hkh^{-1}k^{-1} \in H \cap K \implies hkh^{-1}k^{-1} = e$, da cui segue la tesi.
\end{proof}

\begin{theorem}
    \label{t:1.72}
    Sia $G$ un gruppo e siano $H,K \trianglelefteqslant G$ tali che:
        \begin{enumerate}[(1)]
            \item HK = G.
            \item $H \cap K = \{e\}$.
        \end{enumerate}
    Allora $G \cong H \times K$.
\end{theorem}

\begin{proof}
    Definiamo l'applicazione:
        \[ \varphi : H \times K \longrightarrow G : (h,k) \mapsto hk
            \]
    Si verifica che è un omomorfismo:
        \[\varphi((h_1,k_1)(h_2,k_2)) = \varphi((h_1h_2,k_1k_2)) = h_1h_2k_1k_2
            \]
    per il \hyperref[l:1.71]{Lemma 1.71} si ha che $h_1h_2k_1k_2 = h_1k_1h_2k_2 = \varphi((h_1,k_1))\varphi((h_2,k_2))$, $\forall h_1,h_2 \in H$, $\forall k_1,k_2 \in K$.
    Si osserva ora che $\varphi$ è surgettiva, per l'ipotesi ($1$); infine, è iniettiva in quanto:
        \[ \ker \varphi = \{(h,k) \in H \times K | hk = e\} = \{(h,k) \in H \times K | h = k^{-1}\} = \{e\}
            \]
    dove nell'ultima uguaglianza si è usato il fatto che $H \cap K = \{e\}$.
\end{proof}

\begin{remark}
    Se abbiamo due sottogruppi $G_1$ e $G_2$ e costruiamo $G = G_1 \times G_2$, allora presi:
        \[ H = G_1 \times \{e_2\} \trianglelefteqslant G \qquad \text e \qquad K = \{e_1\} \times G_2 \trianglelefteqslant G
            \]
    $H,K$ sono normali, hanno intersezione banale e sono tali che $HK = G$, quindi verifichiamo le ipotesi del teorema, pertanto $G \cong H \times K$.
\end{remark}

\begin{example}
    Sia $G$ un gruppo con $|G| = p^2$, dalla formula delle classi avevamo ottenuto che $G$ è necessariamente abeliano, quindi $G$ è 
    isomorfo a $\ZZ/p^2\ZZ$ o $\Zp \times \Zp$. Se $G$ è ciclico, allora $G \cong \ZZ/p^2\ZZ$. Mostriamo che se non lo è, allora $G \cong \Zp \times \Zp$ e in 
    questo caso tutti gli elementi di $G$ hanno ordine $p$. \\
    Consideriamo $(e \ne) x \in G$ e $H = \left<x\right> \trianglelefteqslant G$ (in quanto $G$ abeliano); prendiamo $y \in G \setminus\left<x\right>$
    e analogamente $K = \left<y\right> \trianglelefteqslant G$, da ciò segue che $H \cap K = \{e\}$, infatti $H$ e $K$ sono sottogruppi ciclici di $G$ di 
    ordine $p$ e quindi hanno in comune solo l'elemento neutro.
    Osservando infine che $HK = G$, per cardinalità:
        \[ |HK| = \frac{|H||K|}{|H \cap K|} = \frac{p\cdot p}{1} = p^2
            \]
    le ipotesi del \hyperref[t:1.72]{Teorema 1.72} sono verificate, dunque:
        \[ G \cong H \times K \cong \Zp \times \Zp
            \]
\end{example}

\newpage
\subsection{Prodotto semidiretto}

\begin{definition}
    Dati due gruppi $H,K$ e l'azione:
        \[ \varphi : K \longrightarrow \Aut(H) (\leqslant S(H)) : k \longmapsto \varphi_k
            \]
    si dice \vocab{prodotto semidiretto} di $H$ e $K$ via $\varphi$:
        \[ H \rtimes_{\varphi} K
            \]
    (o anche $K {_\varphi}\ltimes H$) l'insieme ottenuto come prodotto cartesiano $H \times K$ con l'operazione definita da:
        \[ (h,k) (h^{\prime},k^{\prime}) = (h \cdot_H \varphi_k(h^{\prime}), k \cdot_K k^{\prime})
            \]
\end{definition}

\begin{proposition}
    [Il Prodotto Semidiretto è un gruppo]
    Dati due gruppi $H,K$, allora $H \rtimes_{\varphi} K$ è un gruppo.
\end{proposition}

\begin{proof}
    Come si verifica facilmente l'operazione indotta dal prodotto semidiretto è associativa, verifichiamo che $(e_H,e_K)$ è 
    l'elemento neutro:
        \[ (h,k)(e_H,e_K) = (h \cdot \varphi_k(e_H), k e_K) = (h e_H, k) = (h,k)
            \]
    dove $\varphi_k(e_H) = e_H$ poiché $\varphi_k$ è un automorfismo (e quindi in particolare un omomorfismo), a sinistra, invece, si ha:
        \[(e_H,e_K)(h,k) = (e_H \cdot \varphi_{e_K}(h), e_Kk) = (e_H \cdot id(h), k) = (e_H h, k) = (h,k)
            \]
    Per l'inverso si osserva:
        \[ (h,k)^{-1} = ((\varphi_k)^{-1}(h^{-1}), k^{-1}) = (\varphi_{k^{-1}}(h^{-1}), k^{-1})\,\footnote{L'uguaglianza $(\varphi_k)^{-1} = \varphi_{k^{-1}}$
         segue dal fatto che $\varphi$ è un omomorfismo e quindi manda inversi in inversi.}
            \]
    dunque si verifica a destra:
        \begin{multline*}
            (h,k)(\varphi_{k^{-1}}(h^{-1}), k^{-1}) = (h \cdot \varphi_k(\varphi_{k^{-1}}(h^{-1})), kk^{-1}) = \\
            =(h \cdot id(h^{-1}), e_K) = (hh^{-1}, e_K) = (e_H, e_K)
        \end{multline*}
    e analogamente a sinistra:
        \begin{multline*}
            (\varphi_{k^{-1}}(h^{-1}), k^{-1})(h,k) = (\varphi_{k^{-1}}(h^{-1}) \cdot \varphi_{k^{-1}}(h), k^{-1}k) = \\
            = (\varphi_{k^{-1}}(h^{-1}h), e_K) = (\varphi_{k^{-1}}(e_H), e_K) = (e_H, e_K)
        \end{multline*}
\end{proof}

\pagebreak

\begin{remark}
    Si osserva che $H \rtimes_{\varphi} K$ è il prodotto diretto se e solo se $\varphi_k = e$, $\forall k \in K$.
    Infatti:
        \[ (h,k)(h^{\prime},k^{\prime}) = (h\cdot \varphi_k(h^{\prime}), kk^{\prime}) = (hh^{\prime}, kk^{\prime}) \iff \varphi_k(h^{\prime}) = h^{\prime}
        \qquad \forall k \in K
            \]
        e dunque $\varphi_k = id_H$.
\end{remark}

\begin{theorem}
    \label{t:1.79}
    Sia $G$ un gruppo e siano $H,K \leqslant G$, con $H \trianglelefteqslant G$, tali che:
        \begin{enumerate}[(1)]
            \item $HK = G$.
            \item $H \cap K = \{e\}$. 
        \end{enumerate}
    Allora $G \cong H \rtimes_{\varphi} K$, dove $\varphi : K \longrightarrow \Aut(H) : k \longmapsto \varphi_k$, con $\varphi_k : h \longmapsto khk^{-1}$.
\end{theorem}

\begin{proof}
    Costruiamo esplicitamente un isomorfismo tra i due gruppi:
        \[ \mathcal{F} : H \rtimes_{\varphi} K \longrightarrow G : (h,k) \longmapsto hk
            \]
    Verifichiamo che è un omomorfismo:
        \[
            \mathcal{F}((h,k)(h^{\prime},k^{\prime})) = \mathcal{F}(h\cdot \varphi_k(h^{\prime}),kk^{\prime}) = \mathcal{F}(h\underbrace{kh^{\prime}k^{-1}}_{= \varphi_k(h^{\prime})},kk^{\prime})
            = hkh^{\prime}k^{-1}kk^{\prime} = \underbrace{hk}_{= \mathcal{F}(h,k)}\underbrace{h^{\prime}k^{\prime}}_{= \mathcal{F}(h^{\prime},k^{\prime})}
        \]
    Si vede inoltre che $\mathcal{F}$ è surgettiva per l'ipotesi ($1$) e iniettiva per la ($2$), infatti:
        \[ \ker\mathcal{F} = \{(h,k) \in  H \rtimes_{\varphi} K | \mathcal{F}(h,k) = hk = e\} = \{e\}
            \]
\end{proof}

\begin{remark}
    Si osserva che $\varphi_k$ è la restrizione al sottogruppo $H$ dell'automorfismo interno $g \longmapsto kgk^{-1}$, poiché
    $H \trianglelefteqslant G$, allora la restrizione a $H$ di ogni elemento di $\Inn(G)$ è un automorfismo di $H$.
\end{remark}

\begin{remark}
    Sapendo che $G \cong H \rtimes_{\varphi} K$ e seguendo i passaggi della verifica di omomorfismo al contrario, si ricava che necessariamente $\varphi$ è esattamente
    l'azione di coniugio su $H$.
\end{remark}

\pagebreak

\begin{remark}
    Siano $\ol H = H \times \{e_K\}$ e $\ol K = \{e_H\} \times K$, si osserva che $\ol H, \ol K \leqslant G = H \rtimes_{\varphi} K$, infatti sono chiusi per prodotto (ristretto):
        \[ (h,e_K)(h^{\prime},e_K) = (h \cdot \varphi_{e_K}(h^{\prime}), e_K) = (h \cdot id(h^{\prime}), e_K) = (hh^{\prime},e_K)
        \]\[ (e_H,k)(e_H,k^{\prime}) = (e_H \cdot \varphi_k(e_H), kk^{\prime}) = (e_H,kk^{\prime})
            \]
    e si verifica facilmente anche per inverso. Si osserva che $\ol H \trianglelefteqslant G$\footnote{$\ol K$ in generale non è normale, lo è solo se il prodotto è diretto, infatti in quel caso vale il \hyperref[t:1.72]{Teorema 1.72}.},
    in quanto $H = \ker \pi$, con:
        \[ \pi : H \rtimes_{\varphi} K \longrightarrow : (h,k) \longmapsto k
            \]
    con $\pi$ omomorfismo come si vede:
        \[ \pi((h,k)(h^{\prime},k^{\prime})) = \pi(h \cdot \varphi_k(h^{\prime}),kk^{\prime}) = kk^{\prime} = \pi((h,k))\pi((h^{\prime},k^{\prime}))
            \]
    Per come li abbiamo presi si nota subito che $\ol H \ol K = G$ e $\ol H \cap \ol K = \{e\}$, quindi valgono le ipotesi del \hyperref[t:1.79]{Teorema 1.79}, pertanto:
        \[ \ol H \times \ol K \cong G = H \rtimes_{\varphi} K
            \]
\end{remark}

\begin{example}
    [$S_n \cong \mathcal{A}_n \rtimes \left<\cycle{1,2}\right>$]
    Verifichiamo che $S_n$ è prodotto semidiretto di $H = \mathcal{A}_n$ e $ K = \left<\cycle{1,2}\right>$ \footnote{In generale va bene qualsiasi trasposizione (che esiste sempre in $S_n$ per $n \geq 2$).}
    usando il \hyperref[t:1.72]{Teorema 1.72}, per quanto detto nel ($1$) del \hyperref[c:1.68]{Corollario 1.68} sappiamo che $\mathcal{A}_n \triangleleft S_n$, inoltre, sempre per il punto ($1$), essendo $|\mathcal{A}_n| = \frac{n!}{2}$, segue
    per cardinalità che $HK = S_n$. Essendo $\mathcal{A}_n = \ker sgn$ e $\left<\cycle{1,2}\right>$ una trasposizione $H \cap K = \{e\}$ (in quanto il nucleo dell'omomorfismo segno contiene solo permutazioni pari), pertanto segue la tesi:
        \[ S_n \cong \mathcal{A}_n \rtimes_{\varphi} \left<\cycle{1,2}\right>
            \]
    Osserviamo inoltre che:
        \[ \varphi : \left<\cycle{1,2}\right> \longrightarrow \Aut(\mathcal{A}_n) : \cycle{1,2} \longmapsto \varphi_{\cycle{1,2}}, id \longmapsto id
            \]
    con $\varphi_{\cycle{1,2}} : \mathcal{A}_n \longrightarrow \mathcal{A}_n : \rho \longmapsto \cycle{1,2}\rho\cycle{1,2}$.
\end{example}

\begin{example}
    [$D_n \cong \Zn \rtimes_{\varphi}\Z2$]
    Ricordando che $D_n = \left<r,s | r^n = s^2 = id, srs^{-1} = r^{-1}\right>$, possiamo osservare ancora una volta che le ipotesi del \hyperref[t:1.72]{Teorema 1.72} sono soddisfatte. Poiché $\ord r = n$, allora $|\left<r\right>| = n$, 
    e in particolare $[D_n : \left<r\right>] = 2 \implies \left<r\right> \triangleleft D_n$; inoltre, $\left<r\right> \cap \left<s\right> = \{id\}$ perché $\det(r_i) = 1$, mentre $\det(sr_i) = -1$, $\forall i \in \{1,\ldots,n\}$. Infine, essendo $\ord s = 2$, allora il prodotto di sottogruppi avrà cardinalità:
        \[ \left<r\right>\left<s\right> = \frac{|\left<r\right>||\left<s\right>|}{|\left<r\right> \cap \left<s\right>|} = \frac{2n}{1} = 2n
            \]
    pertanto $\left<r\right>\left<s\right> = D_n$. Pertanto $D_n \cong \left<r\right> \rtimes_{\varphi} \left<s\right>$, dove $\left<r\right> \cong \Zn$ e $\left<s\right> \cong \Z2$, pertanto:
        \[ D_n \cong \Zn \rtimes_{\varphi} \Z2
            \]
    con:
        \[ \varphi : \left<s\right> \longrightarrow \Aut(\left<r\right>) : s \longmapsto \varphi_s
            \]
    dove $\varphi_s : \left<r\right> \longrightarrow \left<r\right> : r \longmapsto srs^{-1} (= r^{-1})$. Si osserva che deve essere $\ord \varphi_s | \ord s = 2$, quindi ci sono soltanto due possibilità:
        \[ \varphi_s = \begin{cases}
            id \\
            r \longmapsto r^{-1}
            \end{cases}
        \]
    nel caso in cui $\varphi_s = id$ si ottiene il prodotto diretto, nell'altro caso si ottiene il prodotto semidiretto che definisce $D_n$.Se in $\Aut(\Zn)$ ci sono altri elementi di ordine due (ad esempio se $\Aut(\Z8) \cong \Z8^* \cong \Z2 \times \Z2$) si possono definire anche altri prodotti semidiretti:
    \[ \Zn \rtimes_{\varphi} \Z2
        \]
    Rimane il problema di verificare se danno o meno due gruppi isomorfi.
\end{example}

\begin{example}
    [Gruppi di ordine $pq$]
    Sia $|G| = pq$, per il \hyperref[p:Cauchy]{Teorema Di Cauchy} esistono $x,y \in G$ tali che $\ord x = q$, $\ord y = p$, assumiamo (WLOG) $q>p$, allora si ha che:
        \[ H = \left<x\right> \triangleleft G
            \]
    poiché $[G : H] = p$, con $p$ più piccolo primo che divide $|G|$. Alternativamente si può vedere che $H$ è caratteristico in $G$ poiché è l'unico sottogruppo di quell'ordine;
    se $H^{\prime} < G$ e $|H^{\prime}| = q$, se fosse $H \ne H^{\prime}$, allora $H \cap H^{\prime} = \{e\}$ e quindi:
    \[ |HH^{\prime}| = \frac{|H||H^{\prime}|}{|H \cap H^{\prime}|} = \frac{q \cdot q}{1} = q^2 > pq
        \]
    quindi $H^{\prime}$ non può essere un sottogruppo di $G$. Si verifica che, detto $K = \left<y\right>$, le ipotesi del \hyperref[t:1.72]{Teorema 1.72} sono verificate:
        \[ HK = G \qquad H \cap K = \{e\} \qquad H \triangleleft G
            \]
da ciò segue che ogni gruppo di ordine $pq$ è prodotto semidiretto: $G \cong H \rtimes_{\varphi} K$.
\end{example}

Per classificare tutti i gruppi di ordine $pq$ bisogna classificare tutti i possibili prodotti semidiretti $\ZZ/q\ZZ \rtimes_{\varphi} \Zp$ a meno di isomorfismo.
Osserviamo che un prodotto semidiretto deve avere un'operazione definita da:
\[ \varphi : \Zp \longrightarrow \Aut(\ZZ/q\ZZ) \cong \ZZ/q\ZZ^* \cong \ZZ/(q-1)\ZZ
            \]
Essendo $\Zp = \left<x\right>$ e $\ZZ/q\ZZ = \left<y\right>$ possiamo scrivere:
    \[ \varphi : \left<y\right> \longrightarrow \Aut(\left<x\right>) (\cong \ZZ/q\ZZ^* \cong \ZZ/(q-1)\ZZ) : y \longmapsto \varphi_y
            \]
con $\varphi_y : \left<x\right> \longrightarrow \left<x\right> : x \longmapsto x^l$. Per definire $\varphi$ su $\left<y\right>$ (un dominio ciclico)
basta assegnare $\varphi_y$ con la condizione $\ord \varphi_y \mid \ord y = p$, inoltre, $\varphi_y \in \Aut(\Z{q}) \cong \Z{(q-1)} \implies \ord \varphi_y \mid q-1$, 
quindi $\ord \varphi_y \mid (p,q-1)$. Distinguiamo due casi:
    \begin{itemize}
        \item Se $p \nmid q-1$, si ha che $\ord \varphi_y \mid 1 \implies \varphi_y = id$, dunque l'unico automorfismo possibile di $\Z{q}$ è l'identità, pertanto si ha un prodotto diretto tra $\Z{p}$ e
            $\Z{q}$ e quindi esiste ed è unico il gruppo di ordine $pq$, $\Z{pq}$.  
        \item Se $p \mid q-1$, allora o $\ord \varphi_y = 1$ e quindi ancora $\varphi_y = id$; oppure $\varphi_y = p$, e poiché ci sono $p-1$ elementi di ordine $p$ in $\Z{(q-1)}$, abbiamo $p-1$
            scelte per $\varphi_y$ che danno un prodotto semidiretto.
    \end{itemize}
Si osserva che $\ord \varphi_y = \ord_{\Z{q}^*}(\ol l)$ e:
    \[ \varphi_y(x) \longmapsto x^l \implies (\varphi_y(x))^k = x^{l^k}
        \]
quindi $\ord \varphi_y = p \iff l^p \equiv 1 \pmod q \iff \ord l = p$.
Le $p-1$ scelte per $\varphi_y$ danno tutte gruppi isomorfi, quindi se $p \mid q-1$ ci sono esattamente due gruppi di ordine $pq$ a meno di isomorfismo. Infatti, detti:
    \[ G_1 = \left<x\right> \rtimes_{\varphi} \left<y\right>
    \qquad \text e \qquad 
    G_2 = \left<x\right> \rtimes_{\psi} \left<y\right>
        \]
con $\varphi_y(x) = x^l$, $\ord l = p$ e $\psi_y(x) = x^\lambda$, $\ord \lambda = p$, pertanto $\left<l\right> = \left<\lambda\right>$ se e solo se 
$l = \lambda^r$, con $0 < r < p$. Possiamo scrivere l'applicazione:
    \[ \mathcal{F} : G_1 \longrightarrow G_2 : x \longmapsto x, y \longmapsto y^r
        \]
che definisce un isomorfismo tra i due gruppi:
    \[ G_1 = \left<x,y | x^q = y^p = 1, yxy^{-1} = x^l\right> \quad \text e \quad G_2 = \left<x,y | x^q = y^p = 1, yxy^{-1} = x^\lambda\right>
        \]
Per mostrare che è un isomorfismo basta osservare che:
    \[ \mathcal{F}(x^q) = (\mathcal{F}(x))^q = id \qquad \text{in quanto $x^q = id$}
        \]
e anche:
    \[ \mathcal{F}(y^p) = (\mathcal{F}(y))^p = id \qquad \text{in quanto $y^q = id$}
        \]
ed infine:
    \[ \mathcal{F}(yxy^{-1}) = \mathcal{F}(x^l)
        \]
in quanto:
    \[ \mathcal{F}(yxy^{-1}) = \mathcal{F}(y) \mathcal{F}(x) \mathcal{F}(y^{-1}) = \underbrace{y^rxy^{-r}}_{\in G_2} = x^{\lambda^{r}} = x^e = \mathcal{F}(x)
        \]
ciò garantisce che $\mathcal{F}$ ottenuto estendendo l'assegnamento $x \longmapsto x, y \longmapsto y^r$ è un omomorfismo, segue banalmente che è anche una bigezione e quindi è un isomorfismo. \footnote{Quest'ultima pagina non è in versione 
definitiva e necessita di ulteriori revisioni.}

\end{document}