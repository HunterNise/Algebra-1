\documentclass[11pt]{scrartcl}
\usepackage[italian]{babel}
\usepackage[sexy]{evan}

\begin{document}
\title{Complementi di Algebra 1}
\subtitle{\large\normalfont\rmfamily\scshape APPUNTI DEL CORSO DI ALGEBRA 1 TENUTO\\ DALLA PROF. DEL CORSO E DAL PROF. LOMBARDO}
\author{Leonardo Migliorini \\ \textnormal{\href{l.migliorini@studenti.unipi.it}{l.migliorini@studenti.unipi.it}}}
\date{Anno Accademico 2022-23}
\maketitle
\newpage

\tableofcontents
\eject
\newpage

\section*{Ringraziamenti}

Diego Monaco, Niccolò Nannicini, Pietro Crovetto, Leonardo Alfani, Daniele
Lapadula.

\newpage

\section{Gruppi}

\subsection{Insiemi di generatori}

\begin{definition}
    Dati un gruppo $G$ e $x_1, \ldots, x_n$ elementi di $G$, chiamiamo \vocab{sottogruppo 
    generato} da $x_1, \ldots, x_n$ il più piccolo sottogruppo $\langle x_1, \ldots x_n
    \rangle$ di $G$ contenente $x_1, \ldots, x_n$, cioè \[\langle x_1, \ldots, x_n\rangle =
    \bigcap_{\substack{H\leqslant G\\ \{x_1, \ldots, x_n\} \subseteq H}} H\] 
\end{definition}

\begin{remark}
    La definizione è ben posta, infatti l'intersezione avviene su una 
    famiglia non vuota di insiemi dal momento che $G$ è un sottogruppo di 
    se stesso contenente $x_1, \ldots, x_n$. Inoltre l'intersezione non è vuota in 
    quanto contiene almeno l'identità e gli elementi $x_1, \ldots, x_n$.
\end{remark}

La definizione data non dà informazioni su come sono fatti gli elementi di 
$\langle x_1, \ldots, x_n\rangle$, cerchiamo quindi di caratterizzare in modo
diverso tale sottogruppo. Poiché chiuso per l'operazione indotta da $G$, $\langle x_1, \ldots, x_n\rangle$
deve contenere tutti i prodotti finiti, in qualsiasi ordine, delle potenze di
$x_1, \ldots, x_n$, cioè deve contenere l'insieme 
\[\{g_1^{\pm 1} \ldots g_r^{\pm 1}\mid r \in \NN, g_i \in \{x_1, \ldots, x_n\}
~\forall i \in \{1, \ldots, r\}\}\]

\begin{proposition}
Dati un gruppo $G$ e $x_1, \ldots, x_n$ elementi di $G$, allora \[
    \langle x_1 \ldots x_n\rangle = \{g_1^{\pm 1} \ldots g_r^{\pm 1}\mid r 
    \in \NN, g_i \in \{x_1, \ldots, x_n\}~\forall i \in \{1, \ldots, r\}\}
    \]
\end{proposition}

\begin{proof}
Poniamo $S = \{g_1^{\pm 1} \ldots g_r^{\pm 1}\mid r \in \NN, g_i \in \{x_1, \ldots, x_n\}
~\forall i \in \{1, \ldots, r\}\}$, mostriamo che $S$ è un sottogruppo di $G$. 
Effettivamente $e \in S$ in quanto è prodotto di nessuna potenza di $x_1, \ldots, x_n$, 
il prodotto di due elementi di $S$ è ancora un elemento di $S$ in quanto
prodotto finito di potenze di $x_1, \ldots, x_n$ e l'inverso di un elemento
$g_1^{\pm 1}\ldots g_r^{\pm 1} \in\nolinebreak S$ è $(g_1^{\pm 1}\ldots 
g_r^{\pm 1})^{-1} = g_r^{\mp 1}\ldots g_1^{\mp 1}$, che è un elemento di $S$.
Abbiamo quindi che $S$ è un sottogruppo di $G$ contenente $x_1, \ldots, x_n$,
pertanto $\langle x_1, \ldots, x_n\rangle\subseteq S$ per minimalità di $\langle x_1,
\ldots, x_n\rangle$. D'altra parte, per quanto osservato sopra abbiamo che
tutti gli elementi della forma $g_1^{\pm 1}\ldots g_r^{\pm 1}$ con $r \in \NN$, 
$g_i \in \{x_1, \ldots, x_n\}$ per ogni $i \in \{1, \ldots, r\}$ devono essere
contenuti in $\langle x_1, \ldots, x_n\rangle$, pertanto i due sottogruppi
coincidono.
\end{proof}

\begin{remark}
    Se $G$ è un gruppo ciclico abbiamo che esiste $x \in G$ tale che 
    $\langle x\rangle = G$, cioè tutti gli elementi di $G$ sono potenze di $x$.
\end{remark}

Diciamo che $x_1, \ldots, x_n \in G$ sono \vocab{generatori} per $G$, o che 
l'insieme $\{x_1, \ldots, x_n\}$ \vocab{genera} $G$ se $\langle x_1, \ldots, x_n\rangle = G$.

\newpage

\subsection{Automorfismi di $(\Zp)^n$}
\label{aut sp vet}

Dato $p$ un primo, vogliamo determinare quanti sono gli automorfismi di 
$(\Zp)^n$, per fare ciò è conveniente definire una struttura di spazio vettoriale,
quindi un prodotto per scalari\[
    \cdot:\Zp\times (\Zp)^n \longrightarrow (\Zp)^n : 
    (\overline{\lambda}, v)\longmapsto \overline{\lambda}v
\]
con $\overline{\lambda}v = \underset{\tilde{\lambda}\text{ volte}}{\underbrace{v + \ldots + v}}$
e $\tilde{\lambda}$ un qualsiasi rappresentante di $\overline{\lambda}$.
Tale prodotto è ben definito, infatti se $\lambda, \lambda' \in \ZZ$ sono tali che
$\overline{\lambda} = \overline{\lambda'}$, cioè esiste $k \in \ZZ$ tale che
$\lambda = \lambda' + kp$, allora \[
    \overline{\lambda'} v = \underset{\lambda'\text{ volte}}{\underbrace{v + \ldots + v}} = 
    \underset{\lambda + kp\text{ volte}}{\underbrace{v + \ldots + v}} = 
    \underset{\lambda\text{ volte}}{\underbrace{v + \ldots + v}}
\]in quanto $\underset{kp\text{ volte}}{\underbrace{v + \ldots + v}} = 0$. 
Si verifica che $((\Zp)^n, +, \cdot)$ è effettivamente uno spazio vettoriale
sul campo $\FF_p = \Zp$ (dove $\cdot$ è il prodotto per scalari appena definito).
Per come abbiamo definito il prodotto per scalari, abbiamo che per ogni
$\varphi \in \Aut((\Zp)^n)$ vale $\varphi(\lambda v) = \lambda\varphi(v)$ 
per ogni $\lambda \in \FF_p$, pertanto
\[
    \Aut((\Zp)^n) = GL((\FF_p)^n) = \{\varphi: (\FF_p)^n\longrightarrow(\FF_p)^n
    \mid\varphi\text{ isomorfismo di spazi vettoriali}\}
\]

Poiché $GL((\FF_p)^n) \cong GL_n(\FF_p) = \{M \in M_{n\times n}(\FF_p)\mid \det M \neq 0\}$
possiamo rappresentare ogni automorfismo di $(\Zp)^n$ con una matrice invertibile
di taglia $n\times n$ a coefficienti in $\FF_p$.

\begin{proposition}
Dato $p$ un primo, allora \[
    |\Aut((\Zp)^n)| = \prod_{i = 0}^{n - 1} (p^n - p^i)
    \]
\end{proposition}

\begin{proof}
    Osserviamo che un elemento di $\Aut((\Zp)^n)$ deve necessariamente mandare 
    una base di $(\Zp)^n$ in un'altra base, e si dermina univocamente in questo 
    modo. Sia $\{v_1, \ldots, v_n\}$ una base di $(\Zp)^n$ e $\varphi \in 
    \Aut((\Zp)^n)$, consideriamo $\varphi(v_1)$: $\varphi(1)$ può assumere
    qualsiasi valore non nullo, pertanto abbiamo $(p^n - 1)$ possibilità per 
    l'immagine del primo vettore. Per quanto riguarda $v_2$, $\varphi(v_2)$
    può assumere qualsiasi valore non nullo che non sia multiplo di $\varphi(v_1)$,
    che sono $p^n - p$, analogamente $\varphi(v_3)$ può assumere qualsiasi
    valore non nullo che non sia combinazione lineare di $v_1$ e $v_2$, che sono
    $p^n - p^2$, e così via. Reiteriamo questo ragionamento fino a $\varphi(v_n)$,
    che può essere scelto in $p^n - p^{n - 1}$ modi, da cui \[
        |\Aut((\Zp)^n)| = \prod_{i = 0}^{n - 1}(p^n - p^i)
    \]
\end{proof}

\newpage

\subsection{Gruppo diedrale}

\subsubsection{Elementi del gruppo}

\begin{definition}
    Dato $n \geqslant 2$ un numero naturale consideriamo un poligono regolare di $n$ vertici
    centrato nell'origine del piano $\RR^2$,
    chiamiamo \vocab{gruppo diedrale} su $n$ vertici l'insieme $D_n$
    delle isometrie di $\RR^2$ che fissano il poligono, cioè che mandano i 
    vertici in se stessi (per $n = 2$ consideriamo le isometrie che mandano un 
    segmento in se stesso).
\end{definition}

\begin{remark}
    $D_n$ è un gruppo, in quanto l'applicazione identità che 
    fissa tutti i vertici è un'isometria dal poligono in se stesso, la 
    composizione di isometrie è un'isometria e un'isometria ammette sempre 
    un'inversa, che è anch'essa un'isometria.
\end{remark}

\begin{remark}
    Una rotazione di angolo $\displaystyle\frac{2\pi}{n}$ è un elemento di $D_n$,
    così come una simmetria rispetto a un asse.
\end{remark}

Proseguendo con questa intuizione geometrica, indicheremo con $r$ una rotazione
di angolo $\displaystyle \frac{2\pi}{n}$ e con $s$ una simmetria rispetto a
un qualsiasi asse. Notiamo che $\ord(r) = n$ e $\ord(s) = 2$ (per convenzione, 
indichiamo con un angolo positivo una rotazione in senso antiorario e con un 
angolo negativo una rotazione in senso orario).

\begin{definition}
    Data $r \in D_n$ una rotazione di ordine $n$, indichiamo con $\mathcal{R}$ il
    \vocab{sottogruppo delle rotazioni} $\langle r\rangle$.
\end{definition}

\begin{remark}
    Il sottogruppo $\mathcal{R}$ contiene tutte le rotazioni di $D_n$, infatti
    se $r'$ è una rotazione di angolo $\displaystyle\frac{2k\pi}{n}$, $k \in \ZZ$,
    allora $r^k = r'$ in quanto anche $r^k$ è una rotazione di angolo 
    $\displaystyle\frac{2k\pi}{n}$.
\end{remark}

Per determinare come sono fatti gli elementi di $D_n$, studiamo il sottogruppo
$\langle r, s\rangle$. Sicuramente $\langle r, s\rangle$ contiene il sottogruppo $\mathcal{R}$
e tutti gli elementi della forma $sr^k$, $sr^ks$, $sr^ksr^h$ e così via, vogliamo
mostrare che in effetti $D_n$ è generato da $r$ e $s$.

\begin{remark}
    \label{obs1.0}
    Gli elementi della forma $r^k$ e $sr^h$ sono distinti per ogni $h, k \in \ZZ$. 
    Infatti sappiamo dall'algebra lineare che il determinante di una simmetria
    è $-1$ e che il determinante di una rotazione è $1$, per la moltiplicatività
    del determinante quindi $\det (r^k) = (\det r)^k = 1$ e
    $\det (sr^h) = (\det s)(\det r)^h = -1$, da cui $r^k \neq sr^h$.
\end{remark}

\begin{lemma}
    Per ogni rotazione $r \in D_n$ e per ogni simmetria $s \in D_n$ vale
    \[srs^{-1} = r^{-1}\]
\end{lemma}

\begin{proof}
    Senza perdita di generalità possiamo supporre che $r$ sia la rotazione 
    di angolo $\displaystyle\frac{2\pi}{n}$ e che $s$ sia la simmetria
    (rispetto all'asse $y$) che 
    a ogni punto $x$ del piano associa il punto $-x$. Possiamo rappresentare
    rispettivamente $r$ e $s$ tramite le matrici
    \begingroup
    \renewcommand{\arraystretch}{1.2}
    \[
        \begin{pmatrix}
            \cos\left(\frac{2\pi}{n}\right) & -\sin\left(\frac{2\pi}{n}\right)\\
            \sin\left(\frac{2\pi}{n}\right) & \cos\left(\frac{2\pi}{n}\right)
        \end{pmatrix},
        \begin{pmatrix}
            -1 & 0\\
            0 & 1
        \end{pmatrix}
    \]
    \endgroup

    svolgendo esplicitamente il prodotto quindi
    
    \begingroup
    \renewcommand{\arraystretch}{1.2}
    \begin{multline*}
        \begin{pmatrix}
            -1 & 0\\
            0 & 1
        \end{pmatrix}
        \begin{pmatrix}
            \cos\left(\frac{2\pi}{n}\right) & -\sin\left(\frac{2\pi}{n}\right)\\
            \sin\left(\frac{2\pi}{n}\right) & \cos\left(\frac{2\pi}{n}\right)
        \end{pmatrix}
        \begin{pmatrix}
            -1 & 0\\
            0 & 1
        \end{pmatrix} = 
        \begin{pmatrix}
            -1 & 0\\
            0 & 1
        \end{pmatrix}
        \begin{pmatrix}
            -\cos\left(\frac{2\pi}{n}\right) & -\sin\left(\frac{2\pi}{n}\right)\\
            -\sin\left(\frac{2\pi}{n}\right) & \cos\left(\frac{2\pi}{n}\right)
        \end{pmatrix} = \\ =
        \begin{pmatrix}
            \cos\left(\frac{2\pi}{n}\right) & \sin\left(\frac{2\pi}{n}\right)\\
            -\sin\left(\frac{2\pi}{n}\right) & \cos\left(\frac{2\pi}{n}\right)
        \end{pmatrix} = 
        \begin{pmatrix}
            \cos\left(-\frac{2\pi}{n}\right) & -\sin\left(-\frac{2\pi}{n}\right)\\
            \sin\left(-\frac{2\pi}{n}\right) & \cos\left(-\frac{2\pi}{n}\right)
        \end{pmatrix}
    \end{multline*}
    \endgroup
    
    che è la matrice associata alla rotazione di angolo $-\displaystyle
    \frac{2\pi}{n}$, cioè $r^{-1}$.
\end{proof}

\begin{proposition}
    Se $n \geqslant 3$ allora $|D_n| = 2n$.
\end{proposition}

\begin{proof}
    Indicando con $1, \ldots, n$ gli $n$ vertici di un poligono regolare di $n$ lati, notiamo
    che un elemento $g \in D_n$ è univocamente determinato da $g(1), \ldots, g(n)$.
    In particolare, fissato $g(1)$, per il quale abbiamo $n$ possibili scelte,
    abbiamo al massimo due valori per $g(2)$, cioè $g(2) \in \{g(1) + 1, g(1) - 1\}$
    (a meno di sommare $n$ se uno dei due elementi è negativo). Poiché $g(1)$
    e $g(2)$ individuano due vettori nel piano non allineati, cioè
    linearmente indipendenti, ne costituiscono una base: fissati i valori di 
    $g(1)$ e $g(2)$ abbiamo quindi determinato ogni
    elemento di $D_n$ in modo unico e, poiché possiamo farlo in al più $2n$ modi, 
    $|D_n| \leq 2n$. Ricordiamo adesso che $D_n$ contiene gli elementi
    della forma $r^k$, $sr^h$ al variare di $h, k \in \ZZ$, mostriamo che questi sono 
    infatti $2n$. Gli elementi $r^k$ appartengono al gruppo ciclico $\mathcal{R}$
    di ordine $n$, pertanto sono $n$ elementi distinti, inoltre 
    \[
        sr^i = sr^j \iff r^i = r^j\iff i \equiv j \pmod n
    \] pertanto anche questi sono $n$
    elementi distinti. Poiché gli insiemi $\mathcal{R}$ e 
    $\{sr^h\mid h \in \ZZ\}$ sono disgiunti (\hyperref[obs1.0]{Osservazione 3.6}) 
    abbiamo $|D_n| = 2n$.
\end{proof}

\begin{remark}
    Abbiamo mostrato che effettivamente $D_n = \langle r, s\rangle$, quindi i
    suoi elementi sono tutti della forma $r^k$, $sr^h$ al variare di $h, k \in \ZZ$. 
\end{remark}

\begin{remark}
    Il risultato è valido anche per $D_2$, ma con motivazioni diverse. 
    Se consideriamo un segmento nel piano $\RR^2$ giacente sulla retta $y = 0$, 
    le isometrie che possiamo applicare sono l'identità, la rotazione di 
    angolo $\pi$, la simmetria lungo la retta $y = 0$ e la simmetria lungo l'asse
    passante per il suo punto medio. $D_2$ contiene quindi quattro elementi,
    l'identità e tre elementi di ordine 2, pertanto è isomorfo a $\Z2\times\Z2$.
\end{remark}


\subsubsection{Sottogruppi}

Consideriamo un sottogruppo $H\leqslant D_n$, distinguiamo due possibilità: 
$H \subseteq \mathcal{R}$ oppure $H \nsubseteq \mathcal{R}$. Nel primo caso
abbiamo che $|H|\mid n$, ed è l'unico sottogruppo di $\mathcal{R}$ con questa 
proprietà in quanto $\mathcal{R}$ è ciclico, in particolare $H$ è ciclico 
della forma $\langle r^{\frac n d}\rangle$, con $\mathcal{R} = 
\langle r\rangle$ e $d \mid n$. \newline
Studiamo quindi il caso $H \nsubseteq \mathcal{R}$: notiamo che 
$\mathcal{R}\trianglelefteqslant D_n$ in quanto $[D_n : \mathcal{R}] = 2$,
pertanto $\faktor{D_n}{\mathcal{R}}$ è un gruppo con l'operazione indotta da $D_n$
e risulta essere isomorfo a $\Z2$. \newline
Consideriamo la proiezione al quoziente 
\[
    \pi_{\mathcal{R}}: D_n \longrightarrow \faktor{D_n}{\mathcal{R}} : g \mapsto g\mathcal{R}
\]
poiché $H \nsubseteq \mathcal{R}$ abbiamo che esiste $h \in H$ tale che 
$h \notin \mathcal{R}$, pertanto $\pi_{\mathcal{R}}(h) \neq \mathcal{R}$ e
in particolare $\pi_{\mathcal{R}}(H) \nsubseteq \{\mathcal{R}\}$. Dato che i 
sottogruppi di $\faktor{D_n}{\mathcal{R}}$ sono solo $\{\mathcal{R}\}$ e
$\faktor{D_n}{\mathcal{R}}$ abbiamo $\pi_{\mathcal{R}}(H) = 
\faktor{D_n}{\mathcal{R}}$. Osserviamo inoltre che $\ker \pi_{\mathcal{R}\mid H} = 
\ker \pi_{\mathcal{R}} \cap H = \mathcal{R}\cap H$, per il Primo Teorema di Omomorfismo
allora $\displaystyle\frac{H}{H\cap \mathcal{R}} \cong \Z2$, quindi 
$|H\cap\mathcal{R}| = \displaystyle\frac 1 2 |H|$. Dato che $\mathcal{R}\cap H \subseteq
\mathcal{R}$, esiste $k \in \ZZ$ tale che $H\cap\mathcal{R} = \langle r^k\rangle$
in particolare $\langle r^k\rangle$ e $\langle sr^h\rangle$, $h \in \ZZ$, sono
contenuti in $H$. 

\begin{proposition}
    Dati $H\leqslant D_n$ un sottogruppo tale che $H\nsubseteq \mathcal{R}$, se
    $r$ è un generatore di $\mathcal{R}$ tale che $H\cap\mathcal{R} = \langle r^k\rangle$ 
    e $s$ è una simmetria allora \[
    H = \langle r^k\rangle\cdot\langle sr^h\rangle = \{xy \mid x \in \langle r^k\rangle,
    y \in \langle sr^h\rangle\}\qquad h, k \in \ZZ
    \]
\end{proposition}

\begin{proof}
    Per quanto visto sopra vale $|\langle r^k\rangle| = 
    \displaystyle\frac 1 2|H|$, inoltre $\ord(sr^h) = 2$:
    \[
        (sr^h)^2 = sr^hsr^h = (srs)^hr^h = (srs^{-1})^hr^h = r^{-h}r^h = e
    \]
    pertanto $\langle sr^h\rangle \cong \Z2$. Da questo ricaviamo $\langle sr^h\rangle
    \subseteq N_{D_n}(\langle r^k\rangle)$, infatti per ogni $m \in \ZZ$ 
    \[
        (sr^h)r^{mk}(sr^h)^{-1} = sr^{h + mk} sr^h = r^{-h-mk}r^h = r^{-mk}
        \in \langle r^k \rangle
    \]
    cioè $\langle sr^h\rangle \subseteq N_{D_n}(\langle r^k\rangle)$ e quindi
    $\langle r^k\rangle\cdot\langle sr^h\rangle$ è un sottogruppo di $D_n$\footnote
    {Dati $K, N$ sottogruppi
    di un gruppo $G$, se vale almeno una delle inclusioni $K \subseteq N_G(N)$,
    $N \subseteq N_G(K)$ allora $HK = KH$, quindi $HK$ è un sottogruppo di $G$.}.
    Poiché $\langle r^k\rangle$ e $\langle sr^h\rangle$ sono contenuti in $H$
    abbiamo che $\langle r^k\rangle\cdot\langle sr^h\rangle \subseteq H$, inoltre
    \[|\langle r^k\rangle\cdot\langle sr^h\rangle| = \displaystyle\frac 1 2 |H|\cdot 2 = |H|\]
    in quanto $\langle r^k\rangle\cap\langle sr^h\rangle = \{e\}$\footnote{
        Se $H, K$ sono sottogruppi finiti di un gruppo $G$ e $HK\leqslant G$ allora
        vale $|HK| = \displaystyle\frac{|H|\cdot|K|}{|H\cap K|}$.
    }, pertanto 
    i due sottogruppi coincidono.
\end{proof}

\begin{remark}
    Per $k \mid n$ e $0\leq h < k$, i sottogruppi $H_{k, h} = \langle r^k, sr^h\rangle$
    e $H = \langle r^k\rangle\cdot\langle sr^h\rangle$ coincidono. Infatti 
    $H_{k, h}\subseteq H$ in quanto $r^k, sr^h$ sono elementi di $H$, 
    d'altra parte $H \subseteq H_{k, h}$ in quanto $H_{h, k}$ contiene tutti i 
    prodotti finiti delle potenze di $r^k$ e $sr^h$, in particolare gli elementi di $H$.
\end{remark}

\begin{remark}
    Per $k \mid n$ e $0\leq h < k$, $\langle r^k, sr^h\rangle = 
    \langle r^k, sr^{h + k}\rangle$. Infatti $\langle r^k, sr^h\rangle \subseteq
    \langle r^k, sr^{h + k}\rangle$ in quanto $sr^h = (sr^{h + k})r^{-k}$ è
    un elemento del secondo gruppo, simmetricamente $\langle r^k, sr^{h + k}\rangle
    \subseteq \langle r^k, sr^h\rangle$ in quanto $sr^{h + k} = (sr^h)r^k$ è un
    elemento del primo gruppo.
\end{remark}

\begin{theorem}
    [Classificazione dei sottogruppi di $D_n$]
I sottogruppi di $D_n$ sono della forma \begin{enumerate}[(1)]
    \item $\langle r^k\rangle$ con $k\mid n$;
    \item $\langle r^k, sr^h\rangle$ con $k \mid n$, $0\leq h < k$, 
\end{enumerate}
con $r \in \mathcal{R}$ e $s$ una simmetria. Inoltre tali sottogruppi sono
tutti distinti.
\end{theorem}

\begin{proof}
    Abbiamo già visto che i sottogruppi di $D_n$ sono di questo tipo, 
    mostriamo quindi che sono tutti distinti. A meno di cambiare $k$, possiamo
    supporre $\mathcal{R} = \langle r\rangle$, cioè $\ord(r) = n$. 
    Consideriamo $H, K\leqslant D_n$ due sottogruppi, abbiamo tre casi:
    \begin{itemize}
        \item se $H = \langle r^k\rangle$ e $K = \langle r^m\rangle$, $m \in \ZZ$,
        allora $H = K\iff k = m$ in quanto entrambi sottogruppi di $\mathcal{R}$, 
        pertanto esiste un unico sottogruppo della forma $\langle r^k\rangle$
        per $k \mid n$;
        \item se $H = \langle r^k\rangle$ e $K = \langle r^m, sr^h\rangle$, $m \mid n$,
        allora $H \neq K$ in quanto $H$ è ciclico e $K$ no;
        \item se $H = \langle r^k, sr^h\rangle$ e $K = \langle r^m, sr^l\rangle$, 
        con $m \mid n$ e $0\leq l < m$, considerando le intersezioni
        $H \cap \mathcal{R} = \langle r^k\rangle$ e $K \cap \mathcal{R} = \langle r^m\rangle$ 
        abbiamo \[
        H \cap \mathcal{R} = K\cap\mathcal{R} \iff \langle r^k\rangle = \langle r^m\rangle
        \iff k = m
        \] Inoltre, se $sr^h \in \langle r^m, sr^l\rangle = \langle r^m\rangle
        \cdot \langle sr^l\rangle$, allora esiste $t \in \ZZ$ tale che \[
        sr^h = (r^m)^t sr^l \iff sr^h = s^2r^{mt}sr^l \iff r^h = r^{-mt + l}
        \iff h \equiv l - mt \pmod n
        \]da cui ricaviamo $h \equiv l \pmod m$ in quanto $m \mid n$. Ma allora 
        $h =l$ dato che $0 \leq h < k$ e $0\leq l < m$.
    \end{itemize}
\end{proof}

\begin{lemma}
    \label{lemma1.0}
    Dati un gruppo $G$ e $A, B$ due sottogruppi tali che $A \leqslant B \leqslant G$,
    se $B\trianglelefteqslant G$ e $A$ è caratteristico in $B$ allora 
    $A \trianglelefteqslant G$.
\end{lemma}

\begin{proof}
    Fissato $g \in G$, consideriamo l'omomorfismo di coniugio 
    \[
        \varphi_g : G\longrightarrow G : x\longmapsto gxg^{-1}
    \] poiché 
    $B\trianglelefteqslant G$ è ben definita la restrizione $\varphi_{g\mid B} \in \Aut(B)$\footnote{
        Notiamo che $\varphi_{g\mid B}$ in generale non è un 
        coniugio di $B$, poiché $g$ non appartiene necessariamente a $B$.
    }. 
    Dal momento che $A$ è
    un sottogruppo caratteristico di $B$ abbiamo che $\varphi_{g\mid B}(A) =
    \varphi_g(A) = A$,
    pertanto $A \trianglelefteqslant G$.
\end{proof}


\begin{corollary}
    Ogni sottogruppo di $\mathcal{R}$ è normale in $D_n$.
\end{corollary}

\begin{proof}
    Siano $\langle r^k\rangle$ un sottogruppo di $\mathcal{R}$ e $\varphi
    \in \Aut(\mathcal{R})$, allora $\varphi(\langle r^k\rangle) = \langle r^k\rangle$
    in quanto $\varphi$ preserva l'ordine del sottogruppo e $\langle r^k\rangle$
    è l'unico sottogruppo di $\mathcal{R}$ di tale ordine ($\mathcal{R}$ è ciclico),
    pertanto $\langle r^k\rangle$
    è caratteristico in $\mathcal{R}$. Poiché $\mathcal{R}$ è un sottogruppo
    normale di $D_n$, per il \hyperref[lemma1.0]{Lemma 2.15}
    abbiamo $\langle r^k\rangle\trianglelefteqslant D_n$.
\end{proof}

\begin{remark}
    $\mathcal{R} = \langle r \rangle$ è caratteristico in $D_n$ per $n \geqslant 3$.
    Infatti per ogni $\varphi \in \Aut(D_n)$ allora
    $\ord(r) = \ord(\varphi(r))$, da cui $|\langle\varphi(r)\rangle| = n$.
    Se fosse $\varphi(r) \notin \mathcal{R}$ avremmo $\ord(\varphi(r)) = 2$, 
    quindi $|\langle \varphi(r)\rangle| = n = 2$, che è assurdo in quanto $|D_n| \geqslant 6$.
    Questo non è vero per $D_2$, che contiene una rotazione e due
    simmetrie: poiché $\Aut(D_2) \cong S_3$ esiste un $\psi \in \Aut(D_2)$ che manda 
    la rotazione in una riflessione.
\end{remark}

\begin{corollary}
    Per $k\mid n$ e $0\leq h < k$, il sottogruppo $H_{k, h} = \langle r^k, sr^h\rangle$
    è normale in $D_n$ se e solo se $r, s \in N_{D_n}(H_{k, h})$.
\end{corollary}

\begin{proof}~
    \begin{itemize}
        \item Se $H_{k, h}\trianglelefteqslant D_n$ allora $N_{D_n}(H_{k, h}) = D_n$, 
        in particolare $r, s \in N_{D_n}(H_{k, h})$;
        \item se $r, s \in N_{D_n}(H_{k, h})$, poiché il normalizzatore è un
        sottogruppo di $D_n$ abbiamo che $D_n = \langle r, s\rangle \subseteq
        N_{D_n}(H_{k, h})$, pertanto $H_{k, h} \trianglelefteqslant D_n$.
    \end{itemize}
\end{proof}

Vediamo quali sono i sottogruppi normali della forma 
$\langle r^k, sr^h\rangle$, consideriamo i coniugi \[
    \varphi_s: D_n \longrightarrow D_n :x \longmapsto sxs^{-1}\qquad
    \varphi_r: D_n \longrightarrow D_n :x \longmapsto rxr^{-1}
\]e sia $x_1^{\pm 1}\ldots x_m^{\pm 1} \in H_{k, h} = \langle r^k, sr^h\rangle$, allora
\[
    \varphi_s(x_1^{\pm 1}\ldots x_m^{\pm 1}) = \varphi_s(x_1)^{\pm 1}\ldots \varphi_s(x_m)^{\pm 1}
    \in \langle srs, r^hs^{-1}\rangle = \langle sr^ks, r^hs^{-1}\rangle = \langle
    r^k, sr^{-h}\rangle
\]
\[
    \varphi_r(x_1^{\pm 1}\ldots x_m^{\pm 1}) = \varphi_r(x_1)^{\pm 1}\ldots \varphi_r(x_m)^{\pm 1}
    \in \langle r^k, rsr^{h - 1}\rangle = \langle r^k, sr^{h - 2}\rangle
\]

Pertanto $H_{k, h}\trianglelefteqslant D_n$ se e solo se $\langle r^k, sr^{h - 2}\rangle
= \langle r^k, sr^{-h}\rangle = \langle r^k, sr^h\rangle$, se e solo se 
$h \equiv h - 2 \pmod k$, cioè $k \in \{1, 2\}$.\begin{itemize}
    \item Se $k = 1$ allora $H_{k, h} = \langle r, s\rangle = D_n$;
    \item se $k = 2$ (e $n$ pari) allora $H_{k, h} = \langle r^2, sr\rangle$ oppure 
    $H_{k, h} = \langle r^2, s\rangle$.
\end{itemize}

\begin{remark}
    Il secondo caso si presenta solo se $n$ è pari, questo corrisponde al fatto 
    che in un poligono con un numero pari di lati gli assi di simmetria sono 
    per metà passanti per i lati e metà passanti per i vertici opposti. In un poligono con un numero dispari
    di lati gli assi di simmetria sono tutti passanti per i lati.
\end{remark}


\subsubsection{Classi di coniugio}

Abbiamo visto che possiamo scrivere ogni elemento di $D_n$ nella forma
 $s^hr^k$, dove $s$ è una simmetria e $r$ è una rotazione che genera 
 $\mathcal{R}$, con $h \in \{0, 1\}$ e $k \in \{0, \ldots, n - 1\}$
in quanto $\ord(s)= 2$ e $\ord(r) = n$. Inoltre tutti gli elementi della
forma $sr^k$ hanno ordine $2$.

Consideriamo la classe di coniugio di $r$, $\Cl(r) = \{grg^{-1}\mid g \in D_n\}$,
fissato $g \in D_n$ abbiamo due possibili valori per $grg^{-1}$:
\begin{itemize}
    \item se $g \in \mathcal{R}$ allora $g$ è una potenza di $r$, pertanto i
    due elementi commutano e si ha $grg^{-1} = r$;
    \item se $g\notin\mathcal{R}$ allora $g = sr^h$ con $h \in \ZZ$, quindi
    \[
        (sr^h)r(sr^h)^{-1} = (sr^h)r(sr^h) = sr^{h + 1}sr^h = s^2r^{-1-h}r^h = r^{-1}
    \]
\end{itemize}
cioè $\Cl(r) = \{r, r^{-1}\}$. In modo analogo si mostra che $\Cl(r^k) = \{r^k, r^{-k}\}$
per ogni $k \in \ZZ$.

\begin{remark}
    Se $n$ è pari, scriviamo $n = 2m$ e consideriamo la classe di coniugio
    di $r^m$. Poiché $r^m \neq e$ e $r^{2m} = (r^m)^2 = e$ abbiamo che
    $\ord(r^m) = 2$, cioè $(r^m)^{-1} = r^m$. Allora $\Cl(r^m) = \{r^m\}$,
    pertanto abbiamo trovato un elemento del centro di $D_n$ (infatti se $G$
    è un gruppo e $x \in G$, allora $x \in Z(G)$ se e solo se $\Cl(x) = \{x\}$).
\end{remark}

Consideriamo adesso la classe di coniugio di $sr^h$, $\Cl(sr^h) = 
\{g(sr^h)g^{-1}\mid g\in D_n\}$, fissato $g \in D_n$ abbiamo due possibili
valori per $g(sr^h)g^{-1}$:
\begin{itemize}
    \item se $g \in \mathcal{R}$ allora $g = r^k$ con $k \in \ZZ$, pertanto
    \[
        r^k(sr^h)r^{-k} = sr^{-k}r^h r^{-k} = sr^{h - 2k}
    \]
    \item se $g \notin \mathcal{R}$ allora $g = sr^k$ con $k \in \ZZ$, pertanto
    \[
        (sr^k)(sr^h)(sr^k)^{-1} = (sr^k)(sr^h)(sr^k) = sr^{2k - h}
    \]
\end{itemize}
cioè $\Cl(sr^h) = \{sr^{h - 2k}, sr^{2k - h}\mid k \in \ZZ\}$. 

\begin{remark}
    La classe di coniugio di $sr^h$ contiene tutte le simmetrie in cui 
    l'esponente di $r$ ha la stessa parità di $h$. Se $n$ è 
    dispari tutte le simmetrie appartengono alla stessa classe,
    mentre se $n$ è pari abbiamo due classi distinte: quella 
    delle simmetrie rispetto agli assi passanti per i vertici opposti e quella 
    delle simmetrie rispetto agli assi passanti per i lati.
\end{remark}


\subsubsection{Legge di gruppo e omomorfismi}

Se $g$ è un elemento di $D_n$ possiamo scrivere $g$ in modo unico come $s^ar^b$ 
con $a \in \{0, 1\}$ e $b \in \{0, \ldots, n - 1\}$, utilizziamo questa 
proprietà per esplicitare la legge di gruppo di $D_n$. \newline
Fissati $g_1, g_2 \in D_n$, scriviamo $g_1 = s^{a_1}r^{b_1}$ e $g_2 = s^{a_2}r^{b_2}$
con $a_1, a_2 \in \{0, 1\}$ e $b \in \{0, \ldots, n - 1\}$, 
\[
    g_1g_2 = (s^{a_1}r^{b_1})(s^{a_2}r^{b_2}) = s^{a_1}s^{a_2}(s^{a_2}r^{b_1}s^{-a_2})r^{b_2} = 
    s^{a_1}s^{a_2}\varphi_{s^{a_2}}(r^{b_1})r^{b_2}
\]
dove $\varphi_{s^{a_2}}$ è l'automorfismo di coniugio per $s^{a_2}$
(ricordiamo che $s^{a_2} = s^{-a_2}$). Poiché
$\varphi_{s^{a_2}}$ è un omomorfismo e $\varphi_x\circ\varphi_y = \varphi_{xy}$ per ogni $x, y \in G$,
abbiamo $(\varphi_{s^{a_2}}(r^{b_1})) = (\varphi_s^{a_2}(r))^{b_1}$, quindi
\[
    g_1g_2 = s^{a_1}s^{a_2}(\varphi_s^{a_2}(r))^{b_1}r^{b_2} = 
    s^{a_1 + a_2}r^{(-1)^{a_2}b_1 + b_2}
\]
Per l'unicità della scrittura che stiamo usando (scegliendo 
$a \in \{0, 1\}$ e $b \in \{0, \ldots, n - 1\}$)\footnote{
    Ricordiamo che $\varphi_s^m = \underset{m\text{ volte}}{\underbrace{\varphi_s\circ\ldots\circ\varphi_s}}$
    in quanto l'operazione del gruppo degli automorfismi è la composizione di 
    funzioni.} possiamo identificare 
ogni elemento $g = s^ar^b \in D_n$ con la coppia $(a, b)$, la legge di gruppo
è quindi tale che \[
    (a_1, b_1)(a_2, b_2) = (a_1 + a_2, (-1)^{a_2}b_1 + b_2)
\]
\newline 
Usiamo il risultato appena ottenuto per descrivere gli omomorfismi da $D_n$ in 
un qualsiasi gruppo $G$. Poiché ogni elemento $g \in D_n$ si scrive come
$s^ar^b$, con $a, b \in \ZZ$, un omomorfismo $\varphi \in \Hom(D_n, G)$ è univocamente
determinato da $\varphi(r)$ e $\varphi(s)$: infatti \[
    \varphi(g) = \varphi(s^ar^b) = \varphi(s)^a\varphi(r)^b
\]Poniamo $x = \varphi(s)$, $y = \varphi(r)$, necessariamente $\ord(x) \mid 2$
e $\ord(y)\mid n$, cioè $x^2 = e_G$ e $y^n = e_G$, inoltre \[
    xyx^{-1} = \varphi(s)\varphi(r)\varphi(s)^{-1} = \varphi(srs^{-1}) = 
    \varphi(r^{-1}) = \varphi(r)^{-1} = y^{-1}
\]Mostriamo che effettivamente queste condizioni sono anche sufficienti:

\begin{proposition}
    Dati un gruppo $G$ e un'applicazione
    \[
        \varphi:D_n\longrightarrow G :s^ar^b \longmapsto x^ay^b
    \]dove $x = \varphi(s)$ e $y = \varphi(r)$, allora $\varphi$ è un omomorfismo
    se e solo se $x^2 = e_G$, $y^n = e_G$ e $xyx^{-1} = y^{-1}$.
\end{proposition}

\begin{proof}
    Mostriamo che tali condizioni sono sufficienti affinché $\varphi$ sia un 
    omomorfismo. Poiché $x^m = x^{-m}$ per ogni $m \in \ZZ$, fissati 
    $a_1, a_2, b_1, b_2 \in \ZZ$ abbiamo 
    \begin{multline*}
        (x^{a_1}y^{b_1})(x^{a_2}y^{b_2}) = x^{a_1}x^{a_2}(x^{a_2}y^{b_1}x^{-a_2})y^{b_2} = 
        x^{a_1 + a_2}\varphi_{x^{a_2}}(y^{b_1})y^{b_2} = \\
        = x^{a_1 + a_2} (\varphi_x^{a_2}(y))^{b_1}y^{b_2} =
        x^{a_1 + a_2}y^{(-1)^{a_2}b_1}y^{b_2} = x^{a_1 + a_2}y^{(-1)^{a_2}b_1 + b_2}
    \end{multline*}dove $\varphi_g$ è l'automorfismo di coniugio per $g \in G$.
    Allora abbiamo che $\varphi$ è un omomorfismo, infatti per ogni $h_1, h_2, k_1, k_2 \in \ZZ$
    \begin{multline*}
        \varphi((s^{h_1}r^{k_1})(s^{h_2}r^{k_2})) = \varphi(s^{h_1 + h_2}r^{(-1)^{h_2}k_1 + k_2}) =\\
        = x^{h_1 + h_2}y^{(-1)^{h_2}k_1 + k_2} = (x^{h_1}y^{k_1})(x^{h_2}y^{k_2}) = 
        \varphi(s^{h_1}r^{h_2})\varphi(s^{h_2}r^{h_2})
    \end{multline*}
\end{proof}

\begin{remark}
    Abbiamo visto che le condizioni $D_n = \langle r, s\rangle$ con $\ord(r) = n$,
    $\ord(s) = 2$ e $srs^{-1} = r^{-1}$ determinano in modo univoco 
    la struttura astratta di $D_n$, racchiudiamo queste proprietà fondamentali
    nella scrittura
    \[
        \langle r, s\mid r^n = s^2 = e, srs^{-1} = r^{-1}\rangle
    \]
    Tale scrittura si chiama \vocab{presentazione di un gruppo} e ne determina 
    in modo univoco la classe di isomorfismo. Senza scendere troppo nei dettagli,
    nella presentazione indichiamo un insieme di generatori minimale e il 
    minor numero di proprietà che i generatori devono rispettare affinché il 
    gruppo abbia la struttura desiderata. Altri esempi di presentazioni sono
    \[
        \langle x \mid x^n = e\rangle
    \]
    \[
        \langle x\rangle
    \]
    \[
        \langle x, y \mid x^2 = y^2 = e, xy = yx\rangle
    \]
    rispettivamente dei gruppi $\Zn$, $\ZZ$, $\Z2\times\Z2$
    (notiamo che $\Z2\times\Z2$ e $D_2$ hanno la stessa presentazione,
    e questo ha senso in quanto i due gruppi sono isomorfi).
\end{remark}


\subsubsection{Automorfismi}

Studiamo separatamente gli automorfismi di $D_n$ per $n \geqslant 3$ e di $D_2$.\newline
Per $n\geqslant 3$ consideriamo $\varphi \in \Aut(D_n)$, poiché $D_n = \langle
r, s\rangle$ è sufficiente studiare le immagini di $r, s$ per determinare $\varphi$.
Osserviamo che necessariamente $\varphi(r) = r^k$ con $(n, k) = 1$, infatti 
$\varphi$ deve preservare l'ordine di $r$ e la sua immagine deve essere un 
generatore di $\mathcal{R}$, in quanto $\mathcal{R}$ è caratteristico in $D_n$
è isomorfo a $\Zn$. Per quanto riguarda $\varphi(s)$, se $n$ è dispari allora le simmetrie
sono gli unici elementi di ordine 2, pertanto $\varphi(s) = sr^h$ con 
$0\leq h < n$. Se $n$ è pari abbiamo apparentemente due possibilità:
\begin{enumerate}[(1)]
    \item $\varphi(s) = sr^h$, con $0\leq h < n$;
    \item $\varphi(s) = r^{\frac n 2}$, se $n$ è pari.
\end{enumerate}

D'altra parte, se fosse $\varphi(s) = r^{\frac n 2}$ allora $\varphi$ non
sarebbe né iniettiva né surgettiva, pertanto $\varphi(s) = sr^h$ con 
$0\leq h \leq n$. Verifichiamo che $\varphi$ è un omomorfismo, per la 
caratterizzazione che abbiamo dato sopra è sufficiente verificare che
$\varphi(s)\varphi(r)\varphi(s)^{-1} = \nolinebreak\varphi(r)^{-1}$:
\[
    \varphi(s)\varphi(r)\varphi(s)^{-1} = (sr^h)r^k(sr^h)^{-1} = sr^{h + k}r^{-h}s =
    sr^k s^{-1} = r^{-k} = \varphi(r)^{-1}
\]
Inoltre $\varphi$ è surgettiva, infatti $r^k, sr^h \in \mathrm{Im}\varphi$,
cioè 
\[
    \langle r^k, sr^h\rangle = \langle r, sr^h\rangle = \langle s, r\rangle =
    D_n\subseteq \mathrm{Im}\varphi
\]da cui $\mathrm{Im}\varphi = D_n$. Poiché $D_n$ è finito abbiamo che $\varphi$
è un automorfismo. Gli automorfismi di $D_n = \langle r, s\rangle$ quindi sono
tutti e soli gli omomorfismi da $D_n$ in $D_n$ che mandano $r$ in un generatore
di $\mathcal{R}$, che sono $\phi(n)$, e $s$ in un'altra simmetria, che sono 
$n$, pertanto $|\Aut(D_n)| = n\phi(n)$.\newline

Per $n = 2$, sappiamo che $D_2 \cong (\Z2)^2$, pertanto 
\[
    \Aut(D_2) \cong \Aut((\Z2)^2) \cong S_3
\]
Alternativamente possiamo considerare $(\Z2)^2$ come spazio vettoriale su $\FF_2$,
pertanto abbiamo 
\[
    \Aut(D_2) \cong GL_2(\FF_2)
\]Per quanto visto nella sezione \hyperref[aut sp vet]{(2)}, $GL_2(\FF_2)$ 
contiene $(4 - 1)(4 - 2) = 6$ elementi, inoltre $GL_2$ non è un gruppo 
commutativo (con l'operazione di prodotto tra matrici), pertanto $GL_2(\FF_2) \cong S_3$.
In particolare, gli elementi di $GL_2(\FF_2)$ sono:
\begin{itemize}
    \item $\begin{pmatrix}
    1 & 0\\
    0 & 1
    \end{pmatrix}$, che è l'identità del gruppo;
    \item $\begin{pmatrix}
    0 & 1\\
    1 & 0
    \end{pmatrix}, \begin{pmatrix}
        1 & 0\\
        1 & 1
    \end{pmatrix}, \begin{pmatrix}
        1 & 1\\
        0 & 1
    \end{pmatrix}$, che sono gli elementi di ordine 2 corrispondenti alle 
    trasposizioni;
    \item $\begin{pmatrix}
    1 & 1\\
    1 & 0
    \end{pmatrix}, \begin{pmatrix}
        0 & 1\\
        1 & 1
    \end{pmatrix}$ che sono gli elementi di ordine $3$ corrispondenti ai $3$-cicli.
\end{itemize}

\newpage

\subsection{Automorfismi di un prodotto diretto}

Consideriamo due gruppi finiti $H, K$, studiamo il gruppo degli automorfismi 
di $H\times K$. Chiaramente esiste un'inclusione di $\Aut(H)\times \Aut(K)$ in 
$\Aut(H\times K)$ data dall'omomorfismo 
\[
    \iota: \Aut(H)\times \Aut(K)\longhookrightarrow \Aut(H\times K) :
    (\varphi_1, \varphi_2)\longmapsto \varphi_1\times \varphi_2
\]con 
\[
    \varphi_1\times\varphi_2: H\times K \longrightarrow H\times K:
    (g_1, g_2)\longmapsto (\varphi_1(g_1), \varphi_2(g_2))
\]
Mostriamo che $\iota$ è ben definita e che è effettivamente un omomorfismo iniettivo:
\begin{itemize}
    \item per ogni $(\varphi_1, \varphi_2)\in \Aut(H)\times \Aut(K)$, per ogni
     $(g_1, g_2), (h_1, h_2)\in H\times K$ abbiamo 
     \begin{multline*}
        (\varphi_1\times\varphi_2)((g_1, g_2)(h_1, h_2)) = 
        (\varphi_1(g_1h_1), \varphi_2(g_2h_2)) = 
        (\varphi_1(g_1)\varphi_1(h_1), \varphi_2(g_2)\varphi_2(h_2)) = \\
        =(\varphi_1(g_1), \varphi_2(g_2))(\varphi_1(h_1),\varphi_2(h_2)) = 
        ((\varphi_1\times\varphi_2)(g_1, g_2))((\varphi_1\times\varphi_2)(h_1,h_2))
     \end{multline*}
     cioè $\varphi_1\times\varphi_2$ è un omomorfismo. Inoltre 
    \[
        \ker (\varphi_1\times\varphi_2) = \{(g_1, g_2) \in H\times K\mid 
        (\varphi_1(g_1), \varphi_2(g_2)) = (e_H, e_K)\} = \{(e_H, e_K)\}
    \]
    quindi $\varphi_1\times \varphi_2 \in \Aut(H\times K)$
    in quanto $H\times K$ è finito, pertanto $\iota$ è ben definita;
    \item per ogni $(\varphi_1, \varphi_2), (\psi_1, \psi_2) \in \Aut(H)\times \Aut(K)$,
    per ogni $(g_1, g_2) \in H\times K$ abbiamo 
    \begin{multline*}
        \iota((\varphi_1, \varphi_2)(\psi_1, \psi_2))(g_1, g_2) = 
        \iota(\varphi_1\psi_1, \varphi_2\psi_2)(g_1, g_2) = 
        (\varphi_1\psi_1\times\varphi_2\psi_2)(g_1, g_2) =\\
        = (\varphi_1(\psi_1(g_1)), \varphi_2(\psi_2(g_2))) = 
        (\varphi_1\times\varphi_2)(\psi_1(g_1), \psi_2(g_2)) = \\
        = ((\varphi_1\times\varphi_2)(\psi_1\times\psi_2))(g_1, g_2) = 
        (\iota(\varphi_1, \varphi_2)\iota(\psi_1,\psi_2))(g_1, g_2)
    \end{multline*}cioè $\iota((\varphi_1, \varphi_2)(\psi_1, \psi_2)) = 
    \iota(\varphi_1, \varphi_2)\iota(\psi_1, \psi_2)$, quindi $\iota$ è un 
    omomorfismo;
    \item$\iota$ è iniettiva, infatti \begin{multline*}
        \ker \iota = \{(\varphi_1, \varphi_2) \in \Aut(H)\times \Aut(K) \mid
        \iota(\varphi_1, \varphi_2) = id_{\Aut(H\times K)}\} = \\
        = \{(\varphi_1, \varphi_2) \in \Aut(H)\times \Aut(K)\mid 
        (\varphi_1(g_1), \varphi_2(g_2)) = (e_H, e_K)~\forall 
        (g_1, g_2) \in H\times K\}
    \end{multline*} Poiché gli unici elementi $\varphi_1 \in \Aut(H)$,
    $\varphi_2 \in \Aut(K)$ tali che $\varphi_1(H) = \{e_H\}$ e $\varphi_2(K) = \{e_K\}$
    sono rispettivamente $id_{\Aut(H)}$, $id_{\Aut(K)}$ abbiamo \[
        \ker\iota = \{(id_{\Aut(H)}, id_{\Aut(K)})\} = \{id_{\Aut(H \times K)}\}
    \] 

\end{itemize}

\begin{proposition}
    Dati due gruppi finiti $H, K$, $\Aut(H)\times \Aut(K)\cong \Aut(H\times K)$
    se e solo se $H\times \{e_K\}$ e $\{e_H\}\times K$ sono sottogruppi 
    caratteristici di $H\times K$.
\end{proposition}

\begin{proof}
    Sia $\iota$ l'immersione da $\Aut(H)\times \Aut(K)$ in $\Aut(H\times K)$ 
    definita come sopra, se $\iota$ è surgettiva allora ogni elemento di 
    $\Aut(H\times K)$ può essere scritto come $\varphi_1\times\varphi_2$ con
    $\varphi_1 \in \Aut(H)$ e $\varphi_2 \in \Aut(K)$. Allora abbiamo 
    \[
        (\varphi_1\times\varphi_2)(H\times\{e_K\}) = 
        (\varphi_1(H), \varphi_2(\{e_K\})) = H\times\{e_K\}
    \]
    \[
        (\varphi_1\times \varphi_2)(\{e_H\}\times K) = 
        (\varphi_1(\{e_H\}), \varphi_2(K)) = \{e_H\}\times K
    \]cioè $H\times\{e_K\}$ e $\{e_H\}\times K$ sono caratteristici in
    $H\times K$. Viceversa, se i due sottogruppi sono caratteristici, dato
    $\varphi \in \Aut(H\times K)$ poniamo $\varphi_1 \in \Aut(H)$ tale che 
    $\varphi(g_1, e_K) = (\varphi_1(g_1), e_K)$ e $\varphi_2 \in \Aut(K)$ 
    tale che $\varphi(e_H, g_2) = (e_H, \varphi_2(g_2))$ per ogni $g_1 \in H$,
    per ogni $g_2 \in K$ (questo possiamo farlo in quanto $H\times\{e_K\}$ 
    e $\{e_H\}\times K$ sono caratteristici). Allora abbiamo 
    \begin{multline*}
        \varphi(g_1, g_2) = \varphi((g_1, e_K)(e_H, g_2)) = 
        \varphi(g_1, e_K)\varphi(e_H, g_2) = \\
        = (\varphi_1(g_1), e_K)(e_H, \varphi_2(g_2)) = 
        (\varphi_1(g_1), \varphi_2(g_2)) = (\varphi_1\times\varphi_2)(g_1, g_2)
    \end{multline*}cioè $\iota$ è surgettiva e quindi un isomorfismo tra
    $\Aut(H)\times \Aut(K)$ e $\Aut(H\times K)$.
\end{proof}

\begin{example}
    Consideriamo il gruppo $G = \ZZ \times \Zn$, osserviamo che il sottogruppo 
    $\{0\}\times \Zn$ è caratteristico in quanto un automorfismo $\varphi$ di $G$ deve
    preservare gli ordini degli elementi, in particolare quello di un generatore,
    quindi l'immagine di un generatore è un altro generatore del sottogruppo.
    Poiché gli elementi di $G$ di ordine finito sono tutti della forma $(0, d)$
    abbiamo che $\varphi(\{0\}\times\Zn) = \{0\}\times\Zn$. Viceversa, l'immagine
    di $\varphi$ su un generatore di $\ZZ\times\{0\}$, ad esempio 
    $\varphi(1, 0)$, è della forma $(a, b)$,
    e questo implica che $\ZZ\times \{0\}$ non è caratteristico. Se $\varphi$ è
    surgettivo, necessariamente esiste $(x, y) \in \ZZ\times \Zn$ tale che
    $\varphi(x, y) = (\pm 1, 0)$, da cui, posti $\varphi(1, 0) = (a, b)$ e
    $\varphi(0, 1) = (0, d)$ con $n$ e $d$ coprimi, abbiamo
    \begin{multline*}
        \varphi(x, y) = \varphi(x(1, 0) + y(0, 1)) = x\varphi(1, 0) + y\varphi(0, 1) = \\
        = x(a, b) + y(0, d) = (xa, xb + yd) = (\pm 1, 0) \iff a = \pm 1
    \end{multline*}
    Viceversa, se $a = \pm 1$ allora $\varphi$ è surgettiva, infatti 
    per ogni $(x_0, y_0) \in \ZZ\times\Zn$, scegliendo
    $x = x_0a$ e $y \equiv d^{-1}(y_0 - x_0ab)\pmod n$ abbiamo \[
        \varphi(x, y) = (x_0a^2, x_0ab + d d^{-1}(y_0 - x_0ab)) = (x_0, y_0)
    \]e questo ci permette di concludere che $\ZZ\times\{0\}$ non è un sottogruppo
    caratteristico. In questo caso abbiamo solo un'immersione del gruppo
    $\Aut(\ZZ) \times \Aut(\Zn)$ dentro a $\Aut(\ZZ\times \Zn)$, in quanto 
    gli automorfismi che mandano $(\pm 1, 0)$ in $(a, b)$ con $a = \pm 1$ e 
    $b \neq 0$ non possono essere ristretti ad automorfismi di $\ZZ\times \{0\}$.
\end{example}

È utile riuscire a determinare se i sottogruppi $H\times\{e_K\}$, $\{e_H\}\times K$
sono caratteristici in $H\times K$, da cui il seguente risultato:

\begin{proposition}
    Dati due gruppi finiti $H, K$, se $(|H|, |K|) = 1$ allora $H\times\{e_K\}$
    e $\{e_H\}\times K$ sono sottogruppi caratteristici di $H\times K$.
\end{proposition}

\begin{proof}
    Posti $n = |H|$, $m = |K|$, consideriamo l'insieme
    \[
        S = \{(g_1, g_2) \in H\times K\mid (g_1, g_2)^n = (e_H, e_K)\}\]
    Osserviamo che $H\times \{e_K\} = S$, infatti 
    $H\times \{e_K\} \subseteq S$ in quanto tutti gli elementi di $H\times{e_K}$
    hanno ordine che divide $n$. D'altra parte dato $(g_1, g_2) \in S$, se
    $\ord(g_1, g_2) \mid n$ allora $\ord(g_1)\mid n$ e $\ord(g_2)\mid n$, ma 
    $\ord(g_2) \mid m$ per il Teorema di Lagrange, quindi $\ord(g_2) = 1$ e
    $S \subseteq H\times\{e_K\}$, da cui l'uguaglianza. Con un ragionamento
    analogo possiamo caratterizzare $\{e_H\} \times K$ come 
    \[
        \{e_H\} \times K = \{(g_1, g_2) \in H\times K\mid (g_1, g_2)^m = (e_H, e_K)\}
    \] Poiché un automorfismo di $H\times K$ deve preservare gli ordini degli
    elementi, per la caratterizzazione data abbiamo che i due sottogruppi sono
    caratteristici.
\end{proof}

\begin{corollary}
    Se $m, n \geqslant 2$ sono interi coprimi allora
    \[
        \Aut(\Zn\times\Zm) \cong \Aut(\Zn)\times \Aut(\Zm)
    \]
\end{corollary}

\newpage

\subsection{Gruppo derivato}

\begin{definition}
    Dati un gruppo $G$ e $x, y$ elementi di $G$, chiamiamo \vocab{commutatore}
    di $x$ e $y$ l'elemento $[x, y] = xyx^{-1}y^{-1}$. Chiamiamo \vocab{sottogruppo
    derivato} di $G$, oppure \vocab{sottogruppo dei commutatori} di $G$
     il sottogruppo 
    \[
        G' = \langle\{[x, y]\mid x, y \in G\}\rangle
    \]
\end{definition}

\begin{remark}
    $[x, y] = e$ se e solo se $x$ e $y$ commutano.
\end{remark}

\begin{proposition}
    Dato un gruppo $G$, valgono i seguenti fatti:
    \begin{enumerate}[(1)]
        \item $G'$ è un sottogruppo caratteristico di $G$;
        \item $\faktor{G}{G'}$ è un gruppo abeliano;
        \item dato $A$ un gruppo abeliano e $\varphi \in \Hom(G, A)$,
        allora $G' \subseteq \ker\varphi$.
    \end{enumerate}
\end{proposition}

\begin{proof}
    Mostriamo le affermazioni singolarmente:
    \begin{enumerate}[(1)]
        \item consideriamo $\varphi \in \Aut(G)$, poiché $\varphi$ preserva la struttura
        di gruppo è sufficiente descrivere come $\varphi$ agisce sui 
        generatori di $G'$ per determinare $\varphi(G')$. 
        Fissati $x, y \in\nolinebreak G$ abbiamo 
        \[
            \varphi([x, y]) = \varphi(xyx^{-1}y^{-1}) = \varphi(x)\varphi(y)
            \varphi(x)^{-1}\varphi(y)^{-1} \in G'
        \]pertanto $\varphi(G') \subseteq G'$, da cui l'uguaglianza in quanto 
        $\varphi$ è bigettiva;
        \item dati $x, y \in G$, $xG'\cdot yG' = yG'\cdot xG'$ se e solo se 
        $xyG' = yxG'$, che è equivalente a richiedere $xyx^{-1}y^{-1} \in G'$. 
        Dato che effettivamente $xyx^{-1}y^{-1} = [x, y]$ è un elemento di $G'$
        abbiamo che $\faktor{G}{G'}$ è abeliano;
        \item dati $x, y \in G$, abbiamo 
        \[
            \varphi([x, y]) = \varphi(xyx^{-1}y^{-1}) = 
        \varphi(x)\varphi(y)\varphi(x)^{-1}\varphi(y)^{-1}
        \]
        e questo coincide con
        l'identità di $A$ in quanto $A$ è abeliano. Poiché l'immagine di $\varphi$
        è un sottogruppo di $A$ allora $G' \subseteq \ker\varphi$, in quanto
        il commutatore di ogni coppia di elementi di $G$ è contenuto in $\ker \varphi$.
    \end{enumerate}
\end{proof}

\begin{remark}
    Come conseguenza del Primo Teorema di Omomorfismo abbiamo che $\faktor{G}{G'}$ è 
    il "più grande" quoziente abeliano di $G$, o analogamente che 
    $G'$ è il "più piccolo" sottogruppo di $G$ che produce un quoziente abeliano.
    In questo senso, $G'$ misura quanto è abeliano il gruppo $G$.
\end{remark}

\begin{remark}
    Dato $A$ un gruppo abeliano, il Primo Teorema di Omomorfismo produce una bigezione naturale tra 
    $\Hom(G, A)$ e $\Hom\left(\faktor{G}{G'}, A\right)$. Consideriamo infatti $\varphi \in \Hom(G, A)$,
    $\pi_{G'}: G \longrightarrow \faktor{G}{G'}$ la proiezione al quoziente e 
    $\overline{\varphi} : \faktor{G}{G'}\longrightarrow A$, il Teorema
    fornisce un'unico omomorfismo $\overline{\varphi}: \faktor{G}{G'}\longrightarrow A$
    che rende commutativo il diagramma
    \begin{center}
        \begin{tikzcd}[column sep = small, row sep = small]
            G\arrow[rrr, "\varphi"]\arrow[ddd, "\pi_{G'}"', two heads]& & &A \\
            {}\arrow[rr, "\circlearrowleft", phantom]& & {}& \\
            & & & \\
            \faktor{G}{G'}\arrow[uuurrr, "\overline{\varphi}"', hook]& & &
        \end{tikzcd}
    \end{center}
    Viceversa, dato un omomorfismo $\overline{\varphi}:\faktor{G}{G'}\longrightarrow A$
    otteniamo un'unico omomorfismo $\varphi:G\longrightarrow A$ con la 
    composizione $\pi_{G'}\circ\overline{\varphi}$.
\end{remark}

\begin{example}
    Consideriamo il gruppo $S_3$, chiaramente $(S_3)' \neq \{id\}$ in quanto
    $\faktor{S_3}{\langle id\rangle} \cong S_3$ che non è abeliano, pertanto 
    abbiamo due possibilità: $(S_3)' = S_3$ oppure $(S_3)' = \langle\cycle{1, 2, 3}\rangle$\footnote{
        Gli unici sottogruppi normali di $S_3$ sono $\{id\}$, 
        $\langle\cycle{1, 2, 3}\rangle$, $S_3$.}. D'altra parte 
        $\faktor{S_3}{\langle\cycle{1, 2, 3}\rangle}$ è isomorfo a $\Z2$, che
        è abeliano, pertanto $(S_3)'$ è contenuto in $\langle\cycle{1, 2, 3}\rangle$,
        da cui necessariamente $(S_3)' = \langle\cycle{1, 2, 3}\rangle$.
        Più in generale vedremo che $(S_n)' = \mathcal{A}_n$, dove $\mathcal{A}_n$ è il sottogruppo
        di $S_n$ delle permutazioni pari (sappiamo già che $(S_n)' \subseteq
        \mathcal{A}_n$ in quanto $\faktor{S_n}{\mathcal{A}_n}\cong\Z2$).
\end{example}

\newpage

\subsection{Azioni di gruppo}

\subsubsection{Azioni transitive}

\begin{definition}
    Siano $G$ un gruppo e $X$ un insieme, un'azione \[
        \varphi:G\longrightarrow S(X) :g \longmapsto \varphi_g
    \]si dice \vocab{transitiva} se per ogni $x, y \in X$ esiste $g \in G$
    tale che $\varphi_g(x) = y$, equivalentemente se $\Orb(x) = G$ per ogni 
    $x \in X$. Diciamo anche che G \vocab{agisce transitivamente} su $X$ 
    tramite $\varphi$.
\end{definition}

\begin{lemma}
    \label{lemma2.0}
    Dato $G$ un gruppo finito e $H \lneq G$ un suo sottogruppo proprio, allora \[
        G \neq \bigcup_{g \in G}gHg^{-1}
    \]
\end{lemma}

\begin{proof}
    Poniamo $K = \displaystyle\bigcup_{g \in G}gHg^{-1}$, osserviamo che gli
    elementi della forma $xHx^{-1}$ con $x \in N_G(H)$ contribuiscono una
    sola volta all'unione, in quanto $xHx^{-1} = H$, pertanto $K$
    è unione di $[G:N_G(H)] = \displaystyle\frac{|G|}{|N_G(H)|}$ elementi distinti\footnote
    {Infatti, se $X = \{N\mid N\leqslant G\}$ e $\varphi$ è l'azione di coniugio
    su $X$, per ogni $N \in X$ abbiamo $\St(N) = N_G(N)$ e $\Orb(N) = \Cl(N) =
    \{gNg^{-1}\mid g \in G\}$. Vale quindi la relazione $|G| = |\Cl(N)|\cdot|N_G(N)|$.}. Poiché
    $H \subseteq N_G(H)$ e $|gHg^{-1}| = |H|$ per ogni $g \in G$, possiamo stimare 
    la cardinalità di $K$ nel seguente modo 
    \[
        |K| \leq\frac{|G|}{|N_G(H)|}|H| \leq\frac{|G|}{|H|}|H| = |G|.
    \]D'altra parte, per il Principio di Inclusione-Esclusione abbiamo che $|K|$ 
    è somma delle cardinalità dei singoli termini dell'unione se e solo se 
    l'unione è disgiunta, ma questo è falso in quanto ogni classe di coniugio
    di $H$ contiene l'identità del gruppo, quindi $|K| < |G|$, cioè $G \neq K$.
\end{proof}

\begin{proposition}
    \label{prop1.0}
    Dati un gruppo $G$ e un insieme $X$, se 
    \[
        \varphi:G\longrightarrow S(X) :g\longmapsto \varphi_g
    \]è un'azione transitiva valgono i seguenti fatti:
    \begin{enumerate}[(1)]
        \item per ogni $x, y \in X$ esiste $g \in G$ tale che $g\St(x)g^{-1} = \St(y)$;
        \item se $|X|\geqslant 2$ allora esiste $g \in G$ che agisce su $X$ senza
        punti fissi, cioè tale che $\varphi_g(x) \neq x$ per ogni $x \in X$.
    \end{enumerate}
\end{proposition}

\begin{proof}
    Mostriamo i due fatti singolarmente:
    \begin{enumerate}[(1)]
        \item sia $g \in G$ tale che $\varphi_g(x) = y$, dato 
        $h \in g\St(x)g^{-1}$ esiste $w \in \St(x)$ tale che $h = gwg^{-1}$. 
        Allora
        \[
            \varphi_h(y) = \varphi_{gwg^{-1}}(y) = 
            \varphi_g(\varphi_w(\varphi_g^{-1}(y))) = \varphi_g(\varphi_w(x)) =
            \varphi_g(x) = y
        \]pertanto $g\St(x)g^{-1} \subseteq \St(y)$. Osservando che 
        $\varphi_{g^{-1}}(y) = x$ e ragionando in modo simmetrico otteniamo
        l'inclusione $g^{-1}\St(y)g \subseteq \St(x)$, da cui $g\St(x)g^{-1} = \St(y)$;
        \item un elemento $g \in G$ con tali proprietà non può essere contenuto 
        nello stabilizzatore di nessun elemento di $X$, cioè cerchiamo $g \in G$
        tale che
        \[
            g \in \bigcap_{x \in X}\St(x)^{\mathcal{C}}
        \]
        che è equivalente a
        \[
            g \notin \bigcup_{x \in X} \St(x) = \bigcup_{h \in G}h\St(x_0)h^{-1}
        \]per il fatto precedente, fissato $x_0 \in G$. Osserviamo che 
        $\St(x_0) \neq G$, infatti se fosse $\St(x_0) = G$ avremmo 
        \[
            |\Orb(x_0)| = \frac{|G|}{|\St(x_0)|} = 1
        \]ma questo è assurdo in quanto $\Orb(x_0) = X$ per la transitività di 
        $\varphi$ e $|X|\geqslant 2$. Allora per il \hyperref[lemma2.0]{Lemma 6.2}
        abbiamo 
        \[
            G \neq \bigcup_{h \in G}h\St(x_0)h^{-1}
        \]pertanto esiste almeno un elemento $g\in G$ con la proprietà voluta.
    \end{enumerate}
\end{proof}

\begin{remark}
    Se $\varphi$ è l'azione di un gruppo $G$ su un insieme $X$, restringendo
    $\varphi$ all'orbita di un elemento $x \in X$ otteniamo per definizione
    un'azione transitiva su $\Orb(x)$. Pertanto gli stabilizzatori degli elementi 
    di $\Orb(x)$ sono tra loro coniugati.
\end{remark}

\begin{proposition}
    Dato $G$ un gruppo finito e $H \lneq G$ un sottogruppo proprio, se $[G:H] = p$
    con $p$ il più piccolo primo che divide l'ordine di $G$ allora $H$ è normale
    in $G$.
\end{proposition}

\begin{proof}
    Consideriamo l'azione di $G$ sull'insieme quoziente $\faktor{G}{H}$ 
    \[
        \psi: G\longrightarrow S\left(\faktor{G}{H}\right) : g \longmapsto \psi_g
    \]con 
    \[
        \psi_g : \faktor{G}{H}\longrightarrow\faktor{G}{H} : g'H\longmapsto gg'H
    \]Poiché l'immagine di $\psi$ è un sottogruppo di $S\left(\faktor{G}{H}\right)$,
    che è isomorfo a $S_p$, abbiamo che $|\mathrm{Im}\psi| \mid p!$, inoltre 
    $|\mathrm{Im}\psi| = \displaystyle\frac{|G|}{|\ker \psi|}$ come conseguenza
    del Primo Teorema di Omomorfismo. Pertanto $|\mathrm{Im}\psi| \mid (p!, |G|) = p$,
    in quanto $p$ è il più piccolo primo che divide $|G|$, quindi $|\mathrm{Im}\psi| \in \{1, p\}$.
    D'altra parte osserviamo che $\psi$ è un'azione transitiva, infatti per 
    ogni $g_1, g_2 \in G$ abbiamo $\psi_{g_2^{}g_1^{-1}}(g_1H) = g_2g_1^{-1}g_1H = g_2H$,
    pertanto non è possibile $\mathrm{Im}\psi = \{id\}$, da cui $|\mathrm{Im}\psi| = p$
    e $[G:\ker\psi] = p$. Consideriamo il nucleo di $\psi$
    \[
        \ker\psi = \{g\in G\mid gg'H = g'H~\forall g' \in G\}
    \]
    nel caso particolare $g' = e$ otteniamo l'inclusione
    \[
        \ker\psi \subseteq \{g \in G\mid gH = H\} = H
    \]
    in quanto stiamo indebolendo la condizione di appartenenza all'insieme.
    Poiché $[G:\ker \psi] = [G : H] = p$ e $G$ è un gruppo finito abbiamo
    che effettivamente $\ker\psi = H$, cioè $H$ è normale in $G$.
\end{proof}


\subsubsection{Teorema di Cauchy e Piccolo Teorema di Fermat}

Vediamo una dimostrazione alternativa del Teorema di Cauchy e del Piccolo
Teorema di Fermat, di cui ricordiamo gli enunciati, che fa uso del concetto
 di azione. 

\begin{theorem}
    [Teorema di Cauchy]
    \label{teorema1.0}
    Dato un gruppo $G$ e un numero primo $p$, se $p\mid |G|$ allora esiste 
    $g \in G$ tale che $\ord(g) = p$.
\end{theorem}

\begin{theorem}
    [Piccolo Teorema di Fermat]
    \label{teorema2.0}
    Dato un numero primo $p$, se $n \in \ZZ$ è coprimo con $p$ allora 
    $n^{p - 1} \equiv 1 \pmod p$.
\end{theorem}

Dati un gruppo $G$ e un numero primo $p$, consideriamo l'insieme 
\[
    X = \{(g_1, \ldots, g_p) \in G^p\mid g_1\ldots g_p = e\}
\]osserviamo che $|X| = |G|^{p - 1}$, possiamo infatti scegliere liberamente
i primi $p - 1$ elementi di ogni $p$-upla, che ne determinano l'ultimo in 
modo univoco (per unicità dell'inverso). Definiamo un'azione di $\Zp$ su $X$
nel seguente modo:
\[
    \psi: \Zp \longrightarrow S(X) : a \longmapsto \psi_a
\]
con
\[
    \psi_g:X\longrightarrow X : (g_1, \ldots, g_p)\longmapsto (g_{1 + a}, \ldots, g_p, g_1, \ldots, g_a)
\]

Fissato $x \in X$, poiché la cardinalità di $\Orb(x)$ divide l'ordine di $\Zp$
abbiamo che $|\Orb(x)| \in \{1, p\}$, in particolare le orbite di cardinalità
1 sono date dalle $p$-uple della forma $(g, \ldots, g)$ con $g^p = e$.
Poniamo $S = \{g\in G \mid \ord(g) = p\}$ e $\mathcal{R}$ un insieme di 
rappresentanti per la relazione di equivalenza indotta da $\psi$, poiché 
le orbite degli elementi di $X$ formano una partizione dell'insieme abbiamo
\[
    |G|^{p - 1} = |X| = \sum_{x \in \mathcal{R}} |\Orb(x)| = 1 + |S| + \sum_{x \in \mathcal{R}\setminus S}|\Orb(x)|
\]dove l'ultimo termine della somma è divisibile per $p$. Distinguiamo 
quindi due casi:
\begin{itemize}
    \item se $p\mid |G|$, riducendo modulo $p$ la formula sopra otteniamo
    $|S| \equiv -1 \pmod p$, in particolare esiste almeno un elemento di
    ordine $p$ (\hyperref[teorema1.0]{Teorema di Cauchy});
    \item se $G = \Zn$ con $p$ e $n$ coprimi, $\Zn$ non contiene elementi
    di ordine $p$, pertanto riducendo modulo $p$ la formula sopra otteniamo
    $n^{p - 1} \equiv 1 \pmod p$ (\hyperref[teorema2.0]{Piccolo Teorema di Fermat}).
\end{itemize}

\begin{exercise}
    Mostrare che i gruppi di ordine $15$ sono ciclici.
\end{exercise}

\begin{soln}
    Sia $G$ un gruppo di ordine 15, poiché 5 è un primo che divide $|G|$
    esiste $h \in G$ tale che $\ord(h) = 15$ per il \hyperref[teorema1.0]{Teorema di Cauchy}.
    Inoltre, posto $H = \langle h\rangle$, abbiamo che $[G:H] = 3$ e quindi,
    dato che 3 è il più piccolo primo che divide $|G|$, $H$ è un sottogruppo 
    normale di $G$. Mostriamo che $H \subseteq Z(G)$, questo è equivalente a 
    richiedere che l'omomorfismo \[
        \varphi: G\longrightarrow \Aut(H) :g\longmapsto \varphi_{g\mid H}
    \]
    dove $\varphi_g$ è il coniugio per $g$, abbia come unico elemento dell'immagine
    l'applicazione
    \[
        id_H:H\longrightarrow H: h \longmapsto h
    \]
    Poiché $H \cong \Z5$, abbiamo $\Aut(H)\cong (\Z5)^* \cong \Z4$, d'altra 
    parte $|\mathrm{Im}\varphi_{\mid H}|$ divide $(|G|, |H|) = 1$, pertanto
    $|\mathrm{Im}\varphi = 1$ e l'omomorfismo è banale, cioè $H \subseteq Z(G)$.
    Diamo adesso due modi per concludere l'esercizio:
    \begin{enumerate}[(1)]
        \item osserviamo che se $G$ è un gruppo abeliano, cioè se $Z(G) = G$,
        allora abbiamo che $G$ è ciclico. Infatti posto $k \in G$ un elemento di 
        ordine 3 (che esiste in virtù del \hyperref[teorema1.0]{Teorema di Cauchy}),
        abbiamo che $\ord(hk) = \ord(h)\ord(k) = 15$ in quanto i due elementi hanno
        ordine coprimo. D'altra parte, se $G$ non fosse abeliano allora avremmo 
        necessariamente $Z(G) = H$, quindi $\faktor{G}{Z(G)}$ sarebbe ciclico 
        in quanto di ordine 3, pertanto $G$ sarebbe un gruppo abeliano, da cui 
        la tesi per quanto appena detto;
        \item sia $k \in G$ un elemento di ordine 3, consideriamo il centralizzatore
        di $k$
        \[
            Z_G(k) = \{x \in G\mid xk = kx\}
        \]Osserviamo che $k \in Z_G(k)$ e $Z(G) \subseteq Z_G(k)$, pertanto $h$ è un elemento 
        di $Z_G(k)$. Abbiamo quindi che $\ord(h)\mid |Z_G(k)|$ e $\ord(k)\mid |Z_G(k)|$, 
        da cui $|Z_G(k)| = 15$. Abbiamo che tutti gli elementi di ordine 3
        sono contenuti nel centro di $G$, che quindi coincide con $G$. Allora $G$
        è ciclico in quanto abeliano e contenente un elemento di ordine 3 e uno
        di ordine 5, quindi uno di ordine 15.
    \end{enumerate}
\end{soln}

\begin{remark}
    In generale dati $x, y\in G$, se $x$ e $y$ commutano allora 
    $\ord(xy) = [\ord(x), \ord(y)]$ anche se $G$ non è un gruppo abeliano.
\end{remark}

\begin{exercise}
    \label{ex1.0}
    Dato $d$ un numero dispari, mostrare che ogni gruppo di ordine $2d$ ammette
    un sottogruppo normale di indice 2.
\end{exercise}

\begin{soln}
    Consideriamo la rappresentazione regolare a sinistra di $G$
    \[
        \lambda: G \longrightarrow S(G) : g\longmapsto \lambda_g
    \]
    con
    \[
        \lambda_g : G\longrightarrow G : x\longmapsto gx
    \]
    Fissato un isomorfismo $\psi: S(G) \longrightarrow S_{2d}$ poniamo
    $\varphi = \psi\circ\lambda :G\longrightarrow S_{2d}$, $\varphi$ è 
    un omomorfismo iniettivo (infatti nella dimostrazione del Teorema di Cayley
    abbiamo visto che $\lambda$ è un omomorfismo iniettivo). Consideriamo 
    il sottogruppo $\varphi^{-1}(\mathcal{A}_{2d})$, mostriamo che il suo 
    indice in $G$ è al più 2:
    posta $\pi_{\mathcal{A}_{2d}}$ la proiezione al quoziente
    \[
        \pi_{\mathcal{A}_{2d}}:G\longrightarrow \faktor{S_{2d}}{\mathcal{A}_{2d}}\cong \Z2
    \]
    possiamo caratterizzare $\varphi^{-1}(\mathcal{A}_{2d})$ come
    \[
        \varphi^{-1}(\mathcal{A}_{2d}) = \{g \in G \mid \varphi(g) \in \mathcal{A}_{2d}\}
        = \ker (\pi_{\mathcal{A}_{2d}}\circ\varphi)
    \]
    pertanto $\varphi^{-1}(\mathcal{A}_{2d})\trianglelefteqslant G$. 
    Per il Primo Teorema di Omomorfismo abbiamo che esiste un omomorfismo
    iniettivo da $\faktor{G}{\ker(\pi_{\mathcal{A}_{2d}}\circ\varphi)}$ in
    $\Z2$, da cui $[G:\ker(\pi_{\mathcal{A}_{2d}})] \leq 2$. Tale 
    sottogruppo ha indice 1 se e solo se $G = \ker(\pi_{\mathcal{A}_{2d}}\circ\varphi)$,
    cioè $\varphi(G) \subseteq \mathcal{A}_{2d}$, mostriamo che in effetti esiste 
    un elemento di $G$ la cui immagine tramite $\varphi$ è una permutazione 
    dispari. Consideriamo $g \in G$ un elemento di ordine 2, poiché $\varphi$
    è un omomorfismo iniettivo abbiamo che $\ord(\varphi(g)) = \ord(g) = 2$,
    pertanto la permutazione $\varphi(g)$ ha una decomposizione in $d$ 2-cicli,
    cioè è dispari. Pertanto $G \neq \varphi^{-1}(\mathcal{A}_{2d})$,
    da cui $[G: \varphi^{-1}(\mathcal{A}_{2d})] = 2$,
\end{soln}

Possiamo generalizzare il ragionamento appena usato nel seguente risultato

\begin{proposition}
    \label{prop2.0}
    Dato un gruppo $G$ e $H\lneq G$ un sottogruppo tale che $[G:H] = 2$, se
    $K$ è un sottogruppo di $G$ allora $H\cap K$ ha indice 1 o 2 in $K$,
    cioè $[K:H\cap K] \in \{1, 2\}$.
\end{proposition}

\begin{proof}
    Distinguiamo due casi:
    \begin{itemize}
        \item se $K \subseteq H$ allora $H \cap K = K$, da cui $[K:H\cap K] = 1$;
        \item se $K \subsetneq H$ consideriamo la proiezione 
        \[
            \pi_H: G\longrightarrow \faktor{G}{H} :g \longmapsto gH
        \]
        Poiché $\faktor G H \cong \Z2$ abbiamo che gli unici sottogruppi 
        del quoziente sono $\{H\}$ e $\faktor G H$, pertanto 
        $\pi_H(K) = \faktor G H$. Osserviamo che $\ker\pi_{H\mid K}
        = \ker \pi_H \cap K$, per il Primo Teorema di Omomorfismo allora 
        $\faktor{K}{H\cap K} \cong \Z2$, cioè $[K:H\cap K] = 2$.
    \end{itemize}
\end{proof}

\subsubsection{Teorema di Poincaré}

Vediamo un risultato che sarà utile nel futuro, che permette di esibire,
se esistono, sottogruppi normali non banali di un gruppo finito.

\begin{theorem}
    [Teorema di Poincaré]
    \label{teorema3.0}
    Dato un gruppo $G$ finito e $H\leqslant G$ un suo sottogruppo, 
    se $[G:H] = n$ allora esiste un sottogruppo normale $N\triangleleft G$ 
    tale che:
    \begin{enumerate}[(1)]
        \item $N\leqslant H \leqslant G$;
        \item $n \mid [G:N] \mid n!$.
    \end{enumerate}
\end{theorem}

\begin{proof}
    Consideriamo l'azione di $G$ su $\faktor{G}{H}$
    \[
        \psi: G\longrightarrow S\left(\faktor{G}{H}\right)S:g\longmapsto \psi_g
    \]
    con
    \[
        \psi_g:\faktor{G}{H}\longrightarrow\faktor{G}{H} : g'H\longmapsto gg'H
    \]
    \begin{enumerate}[(1)]
        \item Consideriamo il nucleo di $\psi$
        \[
            \ker\psi = \{g \in G\mid gg'H = g'H~\forall g' \in G\}
        \]
        nel caso particolare $g' = e$ otteniamo l'inclusione
        \[
            \ker\psi \subseteq \{g \in G \mid gH = H\} = H
        \]
        in quanto stiamo indebolendo la condizione di appartenenza all'insieme,
        pertanto $\ker\psi \leqslant H$;
        \item poiché $\ker\psi \leqslant H$ abbiamo $[G:H]\mid [G:\ker\psi]$, cioè
        $n \mid [G:\ker\psi]$. Dal Primo Teorema di Omomorfismo abbiamo che
        $\faktor{G}{\ker\psi} \cong \mathrm{Im\psi}$, che è un sottogruppo
        di $S\left(\faktor{G}{H}\right)\cong S_n$, pertanto $[G:\ker\psi]\mid n!$.
    \end{enumerate}
    Poiché $\ker\psi$ è normale in $G$ abbiamo che $N = \ker\psi$ è un sottogruppo
    con le proprietà cercate.
\end{proof}

\begin{remark}
    In particolare, se $G$ ha un sottogruppo di indice $n$ e $n! < |G|$
    allora $G$ ammette sottogruppi normali non banali.
\end{remark}

\newpage

\subsection{Gruppo simmetrico}

\subsubsection{Generatori di $S_n$}

Esibiamo alcuni insiemi di generatori per $S_n$:

\begin{itemize}
    \item $\{\cycle{i, j} \mid i, j \in\{1, \ldots, n\}, i < j\}$, abbiamo visto 
    che ogni permutazione può essere scritta come prodotto di trasposizioni;
    \item $\{\cycle{1, j}\mid j \in \{2, \ldots, n\}\}$, infatti per ogni $i<j$ abbiamo
    \[
        \cycle{i, j} = \cycle{1, i}\cycle{1, j}\cycle{1, i}
    \]
    \item $\{\cycle{i, i + 1}\mid i \in\{1, \ldots, n - 1\}\}$,
    infatti per ogni $j$ abbiamo 
    \[
        \cycle{1, j} = \cycle{j - 1, j}\cycle{1, j - 1}\cycle{j - 1, j}
    \]
    \item $\{\cycle{1, 2}, \cycle{1, 2, \ldots, n}\}$, infatti per ogni
    $i$ abbiamo 
    \[
        \cycle{1, \ldots, n}^{i - 1}\cycle{1, 2}\cycle{1, \ldots, n}^{1 - i} = 
        \cycle{i, i + 1}
    \]
\end{itemize}

\begin{remark}
    Non è vero in generale che una trasposizione e un $n$-ciclo generano $S_n$,
    consideriamo ad esempio $\langle\sigma, \rho\rangle\leqslant S_4$ con
    $\sigma = \cycle{1, 2, 3, 4}$, $\rho = \cycle{2, 4}$. Abbiamo
    $\sigma^4 = \rho^2 = 1$ e $\rho\sigma\rho^{-1} = \cycle{1, 4, 3, 2} =
    \sigma^{-1}$, pertanto $\langle\sigma, \rho\rangle$ è isomorfo a un 
    quoziente di $D_4$. D'altra parte $\langle\sigma\rangle\cap \langle\rho\rangle = \{id\}$
    e $\rho \in N_{S_4}(\sigma)$, pertanto $\langle\sigma, \rho\rangle =
    \langle\sigma\rangle\langle\rho\rangle$ e $|\langle\sigma, \rho\rangle| = 8$,
    pertanto è isomorfo a $D_4$. 
\end{remark}

\subsubsection{Sottogruppi abeliani massimali di $S_n$}

Vogliamo studiare i sottogruppi abeliani di $S_n$, caratterizzando in particolare
i suoi sottogruppi abeliani massimali.

\begin{definition}
    Un sottogruppo $G\leqslant S_n$ si dice \vocab{transitivo} se l'azione
    \[
        \varphi: G\longrightarrow S_n :\sigma \longmapsto \sigma
    \]indotta da $G$ su $\{1, \ldots, n\}$ è transitiva, cioè se per ogni
    $i, j \in \{1, \ldots, n\}$ esiste $\sigma \in G$ tale che $\sigma(i) = j$.
\end{definition}

\begin{lemma}
    \label{lemma3.0}
    Dato $G$ un sottogruppo abeliano di $S_n$, se $G$ è transitivo allora $|G| = n$.
\end{lemma}

\begin{proof}
    Consideriamo l'azione di $G$ su $\{1, \ldots, n\}$
    \[
        \psi : G\longrightarrow S_n :\sigma \longmapsto \sigma
    \]
    poiché $G$ è transitivo, per la \hyperref[prop1.0]{Proposizione 6.3}
    gli stabilizzatori degli elementi di $\{1, \ldots, n\}$ sono tra loro coniugati.
    D'altra parte, poiché lo stabilizzatore è un sottogruppo di $G$, che
    è un gruppo abeliano, la restrizione del coniugio agli
    stabilizzatori coincide con l'applicazione identità, da cui $\St(i) = \St(j)$
    per ogni $i, j \in \{1, \ldots, n\}$. Osserviamo infine che 
    \[
        \bigcap_{i = 1}^n \St(i) = \{id_{S_n}\}
    \]in quanto $id_{S_n}$ è l'unica permutazione che fissa tutti gli elementi
    di $\{1, \ldots, n\}$, pertanto $\St(i) = \{id_{S_n}\}$ per ogni $i \in \{1, \ldots, n\}$.
    Fissato $i \in \{1, \ldots, n\}$, abbiamo 
    \[
        |G| = |\Orb(i)|\cdot|\St(i)| = |\Orb(i)| = n
    \]
    in quanto $G$ è transitivo.
\end{proof}

\begin{lemma}
    \label{lemma4.0}
    Se $a_1, \ldots, a_k$ sono interi positivi tali che $\displaystyle
    \sum_{i = 1}^k a_i = 3m$,
    con $m \geqslant k$ intero, allora $\displaystyle
        \prod_{i = 1}^k a_i \leq 3^m$, 
        e vale l'uguaglianza se e solo se $k = m$ e $a_i = 3$ per ogni $i \in \{1, \ldots, k\}$.
\end{lemma}

\begin{proof}
    Senza perdita di generalità, a meno di aumentare $k$ possiamo supporre
    $a_i \in \{1, 2, 3\}$ per ogni $i \in \{1, \ldots, k\}$, infatti se 
    uno degli $a_i$ è uguale a 4 possiamo sostituirlo con $2 + 2$, se uno degli
    $a_i$ è uguale a 5 possiamo sostituirlo con $2 + (a_i - 2)$ e così via
    (queste sostituzioni mantengono inalterato il valore della somma).
    In particolare abbiamo che $a_i \leqslant 3$ per ogni $i \in \{1, \ldots, n\}$,
    pertanto 
    \[
        \prod_{i = 1}^k a_i \leq 3^k \leq 3^m
    \]
    inoltre se $k = m$ e tutti gli $a_i$ sono uguali a 3 abbiamo chiaramente
    \[
        \prod_{i = 1}^k a_i = 3^k = 3^m
    \]
    Viceversa, se il prodotto degli $a_i$ è uguale a $3^m$ allora necessariamente
    $k = m$ e $a_i = 3$ per ogni $i \in \{1, \ldots, k\}$ in quanto possiamo
    supporre $a_i \in \{1, 2, 3\}$ senza perdita di generalità.
\end{proof}

\begin{lemma}
    \label{lemma5.0}
    Dati $\sigma, \tau \in S_n$, se $\sigma = \cycle{x_1, \ldots, x_k}$ è un 
    $k$-ciclo allora 
    \[
        \tau\sigma\tau^{-1} = \cycle{\tau(x_1),\ldots, \tau(x_k)}
    \]
\end{lemma}

\begin{proof}
    \[
        (\tau\sigma\tau^{-1})(\tau(x_i)) = (\tau\sigma)(x_i) = \tau(x_{i + 1})
    \]per ogni $i \in \{1, \ldots, k\}$, pertanto
    \[
        \tau\sigma\tau^{-1} = \cycle{\tau(x_1), \ldots, \tau(x_k)}
    \]
\end{proof}


\begin{exercise}
    Posto $n = 3m$, mostrare che la massima cardinalità di un sottogruppo
    abeliano di $S_n$ è $3^m$ e caratterizzare la sua classe di isomorfismo.
\end{exercise}

\begin{soln}
    Per prima cosa, osserviamo che $S_n$ contiene sottogruppi abeliani di
    cardinalità $3m$, ad esempio
    \[
        \langle\cycle{1, 2, 3}\rangle\cdot \langle\cycle{4, 5, 6}\rangle
        \cdot \ldots \cdot \langle\cycle{n - 2, n - 1, n}\rangle
    \]
    è un sottogruppo abeliano di $S_n$ di cardinalità $3^m$, essendo 
    isomorfo a
    \[
        \langle\cycle{1, 2, 3}\rangle\times\langle\cycle{4, 5, 6}\rangle
        \times\ldots\times\langle\cycle{n - 2, n - 1, n}\rangle
    \]
    Sia $G$ un sottogruppo abeliano di $S_n$ di ordine massimo, data
    \[
        \psi: G\longrightarrow S_n : \sigma \longmapsto \sigma
    \]
    l'azione naturale di $G$ su $\{1, \ldots, n\}$ chiamiamo $\Omega_1, \ldots, \Omega_k$
    le orbite. Consideriamo le mappe $\varphi_i : G \longrightarrow S(\Omega_i)$
    tali che, data $\sigma \in G$ e fissata $\rho_1\ldots\rho_k$ una sua decomposizione
    in cicli disgiunti, $\varphi_i(\sigma) = \rho_i$, poniamo $G_i = \mathrm{Im}\varphi_i=
    \mathrm{Im}\psi \cap S(\Omega_i)$. Possiamo quindi costruire l'omomorfismo
    \[
        \varphi: G \longrightarrow G_1\times \ldots \times G_k : g \longmapsto (\varphi_1(g), \ldots, \varphi_k(g))
    \]
    che è iniettivo in quanto 
    \[
        \varphi(\sigma) = id \iff \varphi_i(\sigma) = id_{S(\Omega_i)} \iff 
        \sigma_{\mid \Omega_i} = id_{S(\Omega_i)}
    \]
    per ogni $i \in \{1, \ldots, k\}$, che è equivalente a $\sigma = id_{S_n}$
    dato che le orbite ricoprono $\{1, \ldots, n\}$, da cui $\ker\varphi = 
    \{id_{S_n}\}$. Osserviamo adesso che ogni $G_i$ è un gruppo abeliano poiché
    immagine omomorfa di $G$, che è un gruppo abeliano, inoltre è transitivo
    sull'orbita $\Omega_i$ per costruzione, pertanto per il 
    \hyperref[lemma3.0]{Lemma 7.3} abbiamo $|G_i| = |\Omega_i|$ per ogni 
     $i \in \{1, \ldots, k\}$. Vale quindi la seguente disuguaglianza, data
    dall'iniettività di $\varphi$
    \[
        |G| \leqslant \prod_{i = 1}^k|G_i| = \prod_{i = 1}^k |\Omega_i|
    \]
    D'altra parte
    \[
        3m = \sum_{i = 1}^k|\Omega_i|
    \]pertanto per il \hyperref[lemma4.0]{Lemma 7.4} abbiamo $|G| \leq 3^m$, 
    ma questa è effettivamente un'uguaglianza in quanto $S_n$ contiene
    almeno un sottogruppo abeliano di ordine $3^m$ e $G$ ha ordine massimo.
    Sempre per il \hyperref[lemma4.0]{Lemma 7.4} allora $k = m$ e $|\Omega_i| = 3$
    per ogni $i \in \{1, \ldots, k\}$. Abbiamo quindi che $\varphi$ è un isomorfismo
    e che $G_1\times\ldots\times G_k$ è isomorfo a $(\Z3)^k$, pertanto
    $G$ è isomorfo a $(\Z3)^k$.
\end{soln}

\begin{remark}
    Se $a_1, \ldots, a_k$ sono interi tali che 
    \[
        3m + 2 = \sum_{i = 1}^k a_i
    \]
    ragionando come nella dimostrazione del \hyperref[lemma4.0]{Lemma 7.4}
    possiamo scrivere
    \[
        3m + 2 = 2 + \sum_{i = 1}^{k - 1}a_i
    \]
    da cui ricaviamo
    \[
        \prod_{i = 1}^k a_i \leqslant 2\cdot3^m
    \]
    Inoltre questa è un'uguaglianza se e solo se esiste $j \in \{1, \ldots, k\}$
    tale che $a_j = 2$, $a_i = 3$ per ogni $i \in \{1, \ldots, k\}\setminus\{j\}$
    e $k = m$. Ragionando come sopra otteniamo $|G| \leqslant 2\cdot 3^m$, 
    d'altra parte osserviamo che $S_n$ contiene un sottogruppo abeliano
    \[
        \langle\cycle{1, 2, 3}\rangle\cdot\ldots\cdot\langle\cycle{3m - 2, 3m - 1, 3m}\rangle\cdot\langle\cycle{3m + 1, 3m + 2}\rangle
    \]
    di ordine $2\cdot3^m$ poiché isomorfo a 
    \[
        \langle\cycle{1, 2, 3}\rangle\times\ldots\times\langle\cycle{3m - 2, 3m - 1, 3m}\rangle\times\langle\cycle{3m + 1, 3m + 2}\rangle
    \]
    pertanto $|G| = 2\cdot3^m$ e $G \cong (\Z3)^m \times \Z2$.
    Se $n = 3m + 1$, ragionando in modo simile abbiamo che la somma delle
    cardinalità delle orbite $\Omega_1, \ldots, \Omega_k$ è $3m + 1$ e il 
    loro prodotto è minore o uguale a $4\times 3^{m - 1}$, da cui $|G| \leq
    4\cdot3^{m - 1}$. D'altra parte $S_n$ contiene almeno due tipi di 
    sottogruppi abeliani di ordine $3m + 1$, uno isomorfo a 
    $(\Z3)^{m - 1} \times \Z4$ e uno isomorfo a $(\Z3)^{m - 1}\times V_4$,
    dove 
    \[
        V_4 = \{\cycle{1, 2}\cycle{3, 4}, \cycle{1, 3}\cycle{2, 4}, 
    \cycle{1, 4}\cycle{2, 3}, id\}
    \]
    è un sottogruppo abeliano non ciclico di $S_4$, chiamato 
    \vocab{gruppo di Klein} o \vocab{Klein 4-group}. Pertanto un sottogruppo
    abeliano di ordine massimo deve avere una di queste due forme.
\end{remark}

\begin{remark}
    I sottogruppi di $S_n$ di questo tipo sono tutti coniugati tra loro, infatti
    se 
    \[
        G = \langle\cycle{x_1, x_2, x_3}\rangle\cdot\ldots\cdot\langle\cycle{x_{n - 2}, x_{n - 1}, x_n}
    \]
    \[
        G' = \langle\cycle{y_1, y_2, y_3}\rangle\cdot\ldots\cdot\langle\cycle{y_{n - 2}, y_{n - 1}, y_n}\rangle
    \]
    sono due sottogruppi abeliani di $S_n$ di ordine massimo (per semplicità
    supponiamo $n = 3m$, gli altri due casi si studiano in modo analogo)
    consideriamo $\sigma \in S_n$ tale che $\sigma(y_i) = x_i$ per ogni 
    $i \in \{1, \ldots, n\}$, è sufficiente mostrare che i generatori delle componenti del
    prodotto sono tra loro coniugate. Infatti, per il \hyperref[lemma5.0]{Lemma 7.5}
    abbiamo 
    \[
        \sigma\cycle{x_i, x_{i + 1}, x_{i + 2}}\sigma^{-1} = 
        \cycle{\sigma(x_i), \sigma(x_{i + 1}), \sigma(x_{i + 2})} = 
        \cycle{y_i, y_{i + 1}, y_{i + 2}}
    \]
    per ogni $i \in \{1, \ldots, n - 2\}$, pertanto $G$ e $G'$ sono coniugati.
\end{remark}

\subsubsection{Classi di coniugio in $\mathcal{A}_n$}

Studiamo le classi di coniugio in $\mathcal{A}_n$. In particolare,
fissato $\sigma \in \mathcal{A}_n$, vogliamo determinare una relazione tra
$\Cl_{\mathcal{A}_n}(\sigma)$ e $\Cl_{S_n}(\sigma)$.
Poiché valgono $|\mathcal{A}_n| = |\Cl_{\mathcal{A}_n}(\sigma)|\cdot |Z_{\mathcal{A}_n(\sigma)}|$
e $Z_{\mathcal{A}_n}(\sigma) = Z_{S_n}(\sigma) \cap \mathcal{A}_n$, abbiamo
\[
    |\Cl_{\mathcal{A}_n}(\sigma)| = \frac{|\mathcal{A}_n|}{|Z_{\mathcal{A}_n}(\sigma)|} =
    \frac 1 2 \frac{|S_n|}{|Z_{S_n}(\sigma) \cap \mathcal{A}_n|}
\]
Dato che $[S_n:\mathcal{A}_n] = 2$, per la \hyperref[prop2.0]{Proposizione 6.10}
abbiamo $[Z_{S_n}(\sigma):Z_{S_n}(\sigma) \cap \mathcal{A}_n] \in \{1, 2\}$,
distinguiamo quindi due casi:
\begin{itemize}
    \item $|Z_{S_n}(\sigma) \cap \mathcal{A}_n| = \displaystyle\frac 1 2 |Z_{S_n}(\sigma)|$;
    \item $|Z_{S_n}(\sigma) \cap \mathcal{A}_n| = |Z_{S_n}(\sigma)|$.
\end{itemize}

Nel primo caso otteniamo 
\[
    |\Cl_{\mathcal{A}_n}| = \frac 1 2 \frac{|S_n|}{|Z_{S_n}(\sigma) \cap \mathcal{A}_n|} =
    \frac{|S_n|}{|Z_{S_n}(\sigma)|} = |\Cl_{S_n}(\sigma)|
\]
poiché $\Cl_{\mathcal{A}_n}(\sigma)\subseteq \Cl_{S_n}(\sigma)$ abbiamo che le
due classi coincidono. In particolare questo succede se $Z_{S_n}(\sigma)
\nsubseteq \mathcal{A}_n$.

Nel secondo caso invece, che si verifica se $Z_{S_n}(\sigma)\subseteq 
\mathcal{A}_n$,
\[
    |\Cl_{\mathcal{A}_n}(\sigma)| = \frac 1 2 \frac{|S_n|}{|Z_{S_n}(\sigma)\cap\mathcal{A}_n|}
    = \frac 1 2 \frac{|S_n|}{|Z_{S_n}(\sigma)|} = \frac 1 2 |\Cl_{S_n}(\sigma)|
\]
Più precisamente, abbiamo $\Cl_{S_n}(\sigma) = \Cl_{\mathcal{A}_n}(\sigma) \cup
\Cl_{\mathcal{A}_n}(\tau\sigma\tau^{-1})$ per ogni $\tau$ permutazione dispari. 
Infatti $\Cl_{\mathcal{A}_n}(\sigma) \cup \Cl_{\mathcal{A}_n}(\tau\sigma\tau^{-1})
\subseteq \Cl_{S_n}(\sigma)$ (i coniugati di $\tau\sigma\tau^{-1}$ sono anche
coniugati di $\sigma$), d'altra parte per ogni $\rho \in S_n$ abbiamo
$\rho\sigma\rho^{-1} \in \Cl_{\mathcal{A}_n}(\sigma)$ se $\rho$ è pari,
$\rho\sigma\rho^{-1} = (\rho\tau^{-1})(\tau\sigma\tau^{-1})(\rho\tau^{-1})^{-1}
\in \Cl_{\mathcal{A}_n}(\tau\sigma\tau)$ se $\rho$ è dispari, da cui l'uguaglianza.
Abbiamo altri due casi:
\begin{itemize}
    \item $|\Cl_{\mathcal{A}_n}(\tau\sigma\tau^{-1})| = |\Cl_{S_n}(\tau\sigma\tau^{-1})|$;
    \item $|\Cl_{\mathcal{A}_n}(\tau\sigma\tau^{-1})| = \displaystyle\frac 1 2
    |\Cl_{S_n}(\tau\sigma\tau^{-1})|$.
\end{itemize}
Tuttavia se fosse $|\Cl_{\mathcal{A}_n}(\tau\sigma\tau)| = |\Cl_{S_n}(\tau\sigma\tau)|$
avremmo $\Cl_{\mathcal{A}_n}(\sigma) = \Cl_{\mathcal{A}_n}(\tau\sigma\tau^{-1})$,
che è assurdo in quanto $\tau\sigma\tau^{-1} \notin \Cl_{\mathcal{A}_n}(\sigma)$,
pertanto \[
    |\Cl_{\mathcal{A}_n}(\tau\sigma\tau^{-1})| = \displaystyle\frac 1 2
    |\Cl_{S_n}(\tau\sigma\tau^{-1})| = \frac 1 2|\Cl_{S_n}(\sigma)| =
    |\Cl_{\mathcal{A}_n}(\sigma)|
\]
Poiché 
$|\Cl_{S_n}(\sigma)| = |\Cl_{\mathcal{A}_n}(\sigma)| + |\Cl_{\mathcal{A}_n}(\tau\sigma\tau^{-1})|$,
per il Principio di Inclusione-Esclusione abbiamo che l'unione è disgiunta,
cioè
\[
    \Cl_{S_n}(\sigma) = \Cl_{\mathcal{A}_n}(\sigma) \cupdot \Cl_{\mathcal{A}_n}(\tau\sigma\tau^{-1})
\]

\subsubsection{Studio di $S_5$}

%trovare un modo per incastrare meglio questo pezzo
Consideriamo gli elementi di $S_5$ $\sigma = \cycle{1, 2, 3, 4, 5}, 
\tau = \cycle{2, 5}\cycle{3, 4}$, studiamo il sottogruppo $H = \langle\sigma, \tau\rangle$,
in particolare siamo interessati a determinare una regola di commutazione
per $\sigma$ e $\tau$. Osserviamo che 
\[
    \tau\sigma\tau^{-1} = \cycle{\tau(1), \tau(2), \tau(3), \tau(4), \tau(5)} = 
    \cycle{1, 5, 4, 3, 2}
\]
e che questo coincide con $\sigma^{-1}$. Abbiamo quindi che $H$ è generato 
da un elemento $\tau$ di ordine 2 e da un elemento $\sigma$ di ordine 5 
che soddisfano la relazione $\tau\sigma\tau^{-1} = \sigma^{-1}$, pertanto 
$H$ è isomorfo a un sottogruppo del gruppo diedrale $D_5$. D'altra parte, 
da questa relazione ricaviamo che $\langle \tau\rangle \subseteq N_{S_5}(\langle\sigma\rangle)$,
pertanto possiamo scrivere $H = \langle\sigma\rangle\cdot\langle\tau\rangle$
in quanto $\langle\sigma\rangle\cdot\langle\tau\rangle$ è un sottogruppo di $H$
che ha la sua stessa cardinalità. In particolare otteniamo che $|H| = 10 = |D_5|$,
quindi $H \cong D_5$.\\

%%%

Abbiamo visto che le classi di coniugio in un gruppo simmetrico su $n$ elementi
sono parametrizzate dalle partizioni di $n$
\begingroup
\renewcommand{\arraystretch}{2}
\begin{center}
    \begin{tabular}{c|c}
        Partizioni di 5 & Cardinalità della classe di coniugio associata\\
        \hline
        5 & $\displaystyle\binom{5}{5}4! = 4! = 24$\\
        4 + 1 & $\displaystyle\binom{5}{4}3! = 30$\\
        3 + 2 & $\displaystyle\binom{5}{3}2!\binom{2}{2}1! = 20$\\
        3 + 1 + 1 & $\displaystyle\binom{5}{3}2! = 20$\\
        2 + 2 + 1 & $\displaystyle\frac 1 2\binom{5}{2}1!\binom{3}{2}1! = 15$\\
        2 + 1 + 1 + 1 & $\displaystyle\binom{5}{2}1! = 10$\\
        1 + 1 + 1 + 1 + 1 & $1$
    \end{tabular}
\end{center}
\endgroup
(Nel calcolo della cardinalità della classe associata alla partizione 2 + 2 + 1
dividiamo per 2 in quanto contiamo i cicli a meno dell'ordine, e
le coppie di trasposizioni che stiamo considerando commutano).
Di queste, le permutazioni che appartengono a $\mathcal{A}_5$ sono quelle
la cui classe di coniugio è associata alle partizioni 5, 3 + 1 + 1, 
2 + 2 + 1, 1 + 1 + 1 + 1 + 1, cioè le permutazioni $\sigma, \tau, \rho$
aventi una decomposizione in cicli disgiunti della forma
\[
    \sigma = \cycle{a_1, a_2, a_3, a_4, a_5}
\]
\[
    \tau = \cycle{b_1, b_2, b_3}
\]
\[
    \rho = \cycle{c_1, c_2}\cycle{d_1, d_2}
\]
e l'identità. Vediamo come sono fatte le loro classi di coniugio in $\mathcal{A}_5$.
Chiaramente $\Cl_{\mathcal{A}_n}(id) = \Cl_{S_n}(id) = \{id\}$, studiamo quindi
le classi di $\sigma, \tau, \rho$ fissate come sopra.
\begin{itemize}
    \item $Z_{S_5}(\sigma) = \langle\cycle{a_1, a_2, a_3, a_4, a_5}\rangle$,
    infatti 
    \[
        |Z_{S_5}(\sigma)| = \frac{|S_5|}{|\Cl_{S_5}(\sigma)|} = \frac{5!}{4!} = 5
    \]
    Allora $Z_{S_5}(\sigma)$ contiene solo permutazioni pari, fissata $\psi$
    una permutazione dispari la sua
    classe di coniugio in $S_5$ si scrive come
    \[
        \Cl_{S_5}(\sigma) = \Cl_{\mathcal{A}_5}(\sigma) \cupdot \Cl_{\mathcal{A}_5}(\psi\sigma\psi^{-1})
    \]
    \item $Z_{S_5}(\tau)$ non è contenuto in $\mathcal{A}_5$, infatti una
    trasposizione $\psi$ disgiunta da $\tau$ è una permutazione dispari che
    appartiene al centralizzatore. Pertanto
    \[
        \Cl_{S_5}(\tau) = \Cl_{\mathcal{A}_5}(\tau)
    \]
    \item $Z_{S_5}(\rho)$ non è contenuto in $\mathcal{A}_5$, infatti la 
    trasposizione $\cycle{c_1, c_2}$ è una permutazione dispari che commuta
    con $\rho$ (infatti $\cycle{c_1, c_2}$ e $\cycle{d_1, d_2}$ commutano in 
    quanto cicli disgiunti e $\cycle{c_1, c_2}$ commuta con se stessa).
    Pertanto 
    \[
        \Cl_{S_5}(\rho) = \Cl_{\mathcal{A}_5}(\rho)
    \]
\end{itemize}

\subsubsection{Sottogruppi normali di $\mathcal{A}_n$}

Esibiamo alcuni insiemi di generatori per $\mathcal{A}_n$:
\begin{itemize}
    \item $\{\cycle{i, j}\cycle{k, l}\mid i\neq j, k \neq l\}$, infatti
    ogni elemento di $\mathcal{A}_n$ può essere scritto come prodotto di 
    coppie di trasposizioni in quanto permutazione pari;
    \item $\{\cycle{i, j, k}\mid i, j, k\text{ distinti}\}$. Infatti se 
    $\{i, j\} = \{k, l\}$ allora $\cycle{i, j}\cycle{k, l} = id$ è un elemento
    generato dall'insieme, se invece $|\{i, j\}\cap\{k, l\}| = 1$, ad esempio $j = k$,
    abbiamo $\cycle{i, j}\cycle{k, l} = \cycle{i, j}\cycle{j, l} = \cycle{i, j, l}$,
    che è un elemento generato dall'insieme. Nel caso $\{i, j\}\cap \{k, l\} = \emptyset$ abbiamo
    $\cycle{i, j}\cycle{k, l} = \cycle{i, j}\cycle{j, k}\cycle{j, k}\cycle{k, l}=
    \cycle{i, j, k}\cycle{j, k, l}$, che è un elemento generato dall'insieme.
    Possiamo quindi ottenere il precedente insieme di generatori a partire
    da questo;
    %\item $\{\cycle{1, 2, k}\mid k \in \{3, \ldots, n\}\}$, infatti 
\end{itemize}

\begin{definition}
    Un gruppo non banale $G$ si dice \vocab{semplice} se i suoi unici sottogruppi
    normali sono $\{e\}$ e $G$.
\end{definition}


\begin{proposition}
    \label{prop3.0}
    $\mathcal{A}_5$ è un gruppo semplice.
\end{proposition}

\begin{proof}
    Ricordiamo le cardinalità delle classi di coniugio in $\mathcal{A}_5$:


    \begingroup
    \renewcommand{\arraystretch}{1.5}
    \begin{center}
    \begin{tabular}{c|c}
        Rappresentante della classe & Cardinalità della classe\\
        \hline
        $\cycle{1, 2, 3, 4, 5}$ & 12\\
        $\cycle{2, 1, 3, 4, 5}$ & 12\\
        $\cycle{1, 2}\cycle{3, 4}$ & 15\\
        $\cycle{1, 2, 3}$ & 20\\
        $id$ & 1 
    \end{tabular}
    \end{center}
    \endgroup
    In generale, un sottogruppo è normale se e solo se è unione disgiunta
    delle classi di coniugio dei suoi elementi, quindi la cardinalità
    di $N \trianglelefteqslant \mathcal{A}_5$ deve essere somma di alcuni 
    termini nella seconda colonna, compreso 1. D'altra parte $|N| \mid \mathcal{A}_5 = 60$,
    da cui $|N| = 1$ oppure $|N| = 60$. Pertanto $\mathcal{A}_5$ è semplice.
\end{proof}

\begin{lemma}
    \label{lemma6.0}
    Dati un gruppo $G$ e $N\trianglelefteqslant G$ un sottogruppo normale di 
    indice finito, $N$ contiene ogni elemento di $G$ il cui ordine è coprimo
    con $[G:N]$.
\end{lemma}

\begin{proof}
    Sia $g \in G$ tale che $(\ord(g), [G:N]) = 1$, consideriamo la proiezione
    \[
        \pi_N:G \longrightarrow \faktor{G}{N} (x \longmapsto xN)
    \]
    Poiché $\pi_N$ è un omomorfismo abbiamo $\ord(\pi_N(g)) \mid (\ord(g), [G:N]) = 1$,
    pertanto $\pi_N(g) = N$, cioè $g \in N$.
\end{proof}

Diamo adesso una dimostrazione alternativa della semplicità di $\mathcal{A}_5$.

\begin{proof}
    Consideriamo un sottogruppo normale $N\trianglelefteqslant \mathcal{A}_5$.
    Distinguiamo tre casi:
    \begin{itemize}
        \item se $2 \mid [\mathcal{A}_5:N]$, per il \hyperref[lemma6.0]{Lemma7.11}
        $N$ contiene tutti gli elementi di 
        $\mathcal{A}_5$ di ordine $2$, cioè le permutazioni della forma $\cycle{a, b}\cycle{c, d}$
        con $a\neq b$ e $c \neq d$, da cui $N = \mathcal{A}_5$ in quanto contiene 
        un suo insieme di generatori;
        \item se $3\nmid [\mathcal{A}_5:N]$, per il \hyperref[lemma6.0]{Lemma7.11}
        $N$ contiene tutti gli elementi di 
        $\mathcal{A}_5$ di ordine 3, cioè i 3-cicli, da cui $N = \mathcal{A}_5$
        in quanto contiene un suo insieme di generatori;
        \item se $6 \mid [\mathcal{A}_5:N]$ allora $|N| \mid 10$, ma l'unica
        classe di coniugio di $\mathcal{A}_5$ di cardinalità minore di 10 è
        $\{id\}$, pertanto $N = \{id\}$.
    \end{itemize}
    Quindi $\mathcal{A}_5$ è semplice.
\end{proof}

In effetti vale un risultato più generale

\begin{proposition}
    $\mathcal{A}_n$ è un gruppo semplice per $n \geq 5$.
\end{proposition}

\begin{proof}
    Procediamo per induzione su $n$, per $n = 5$ la tesi è garantita dalla 
    \hyperref[prop3.0]{Proposizione 7.10}, supponiamo quindi che $\mathcal{A}_n$
    sia un gruppo semplice e mostriamo che anche $\mathcal{A}_{n + 1}$ lo è.
    Consideriamo un sottogruppo normale $N \trianglelefteqslant \mathcal{A}_{n + 1}$
    e i sottogruppi 
    \[
        H_i = \{\sigma \in \mathcal{A}_{n + 1}\mid \sigma(i) = i\},~
        i \in \{1, \ldots, n + 1\}
    \]
    questi sono tutti isomorfi a $\mathcal{A}_n$ (infatti
    gli elementi di $H_i$ sono tutte e sole le permutazioni pari su $n + 1$
    elementi che fissano l'$i$-esimo, cioè sono permutazioni pari su $n$ elementi).
    Notiamo che l'azione naturale di $\mathcal{A}_{n + 1}$ su $\{1, \ldots, n + 1\}$
    \[
        \psi:\mathcal{A}_{n + 1} \longrightarrow S_{n + 1} :\sigma \longmapsto \sigma
    \]
    è transitiva, infatti per $i, j \in \{1, \ldots, n + 1\}$ distinti 
    la permutazione pari $\rho =\nolinebreak \cycle{i, j}\cycle{h, k}$, con $\cycle{i, j}$
    disgiunta da $\cycle{h, k}$, è tale che $\rho(i) = j$. Per costruzione 
    vale $\St(i) = H_i$ per ogni $i \in \{1, \ldots, n + 1\}$, pertanto
    per la \hyperref[prop1.0]{Proposizione 6.3} abbiamo che gli $H_i$ sono 
    tutti coniugati.\newline
    Fissato $i \in \{1, \ldots, n + 1\}$, consideriamo $N \cap H_i$: questo
    è un sottogruppo normale di $H_i$, infatti per ogni $h \in H_i$ si ha 
    $h(N\cap H_i)h^{-1} = N\cap H_i$ in quanto $N$ è normale in $\mathcal{A}_{n + 1}$
    e $h \in H_i$, d'altra parte $H_i \cong \mathcal{A}_n$ è un gruppo semplice
    per ipotesi induttiva, pertanto $N\cap H_i$ coincide con $\{id\}$ oppure 
    con $H_i$. \newline
    Se $N \cap H_i = H_i$ allora $H_i \subseteq N$, pertanto $N$ 
    contiene almeno un 3-ciclo $\cycle{i, j, k}$ e tutti i suoi coniugati in 
    $\mathcal{A}_{n + 1}$. Notiamo che una trasposizione $\cycle{a, b}$
    disgiunta da $\cycle{i, j, k}$ (che esiste in quanto $n \geq 5$) è una
    permutazione dispari in $Z_{S_{n + 1}}(\cycle{i, j, k})$, pertanto 
    $\Cl_{\mathcal{A}_{n + 1}}(\cycle{i, j, k}) = \Cl_{S_{n + 1}}(\cycle{i, j, k})$
    e $N$ contiene l'insieme dei 3-cicli di $S_{n + 1}$, quindi $N = 
    \mathcal{A}_{n + 1}$ dal momento che contiene un suo insieme di generatori.
    \newline
    Altrimenti $N\cap H_i = \{id\}$ per ogni $i \in \{1, \ldots, n + 1\}$,
    cioè l'unico elemento di $N$ avente almeno un punto fisso è l'identità,
    vogliamo mostrare che in effetti $N = \{id\}$. Osserviamo che
    $\sigma \in N$ ha una decomposizione in cicli disgiunti
    della forma
    \[
        \sigma = \cycle{x_1^{(1)}, \ldots, x_{l_1}^{(1)}}\ldots\cycle{x_1^{(k)}, \ldots, x_{l_k}^{(k)}}
    \]
    con $l_1\leqslant l_2\leqslant \ldots\leqslant l_k$, allora i suoi cicli 
    hanno tutti la stessa lunghezza, cioè $l_i = l_j$ per ogni $i \neq j$.
    Infatti, posto $r = \min\{l_i\mid 1\leqslant i \leqslant k\} = l_1$, 
    abbiamo
    \[
        \sigma^{l_1} = id \cdot \cycle{x_1^{(2)}, \ldots, x_{l_2}^{(2)}}^{l_1}
        \ldots\cycle{x_1^{(k)}, \ldots, x_{l_k}^{(k)}}^{l_1}
    \]
    da cui $\sigma^{l_1} = id$ in quanto ha almeno un punto fisso e quindi 
    $l_1 = l_2 = \ldots = l_k$. Fissata $\sigma \in N$ possiamo quindi scrivere
    $\sigma = \sigma_1\ldots\sigma_k$,
    dove $\sigma_i$ sono $l$-cicli disgiunti con $l = \displaystyle\frac{n + 1}{k}$.
    Supponiamo per assurdo $N\cap H_i \neq \{id\}$, distinguiamo tre casi:
    \begin{itemize}
        \item se $k = 1$ abbiamo $l = n + 1$, cioè $\sigma$ è un $n + 1$-ciclo.
        Scriviamo $\sigma = \cycle{a_1, \ldots, a_l}$ e consideriamo la permutazione
        pari $\tau = \cycle{a_1, a_2}\cycle{a_3, a_4}$, poiché $N$ è normale in 
        $\mathcal{A}_{n + 1}$ contiene
        \[
            \tau\sigma\tau^{-1} = \cycle{a_2, a_1, a_4, a_3, a_5, a_6, \ldots, a_l}
        \]
        Consideriamo $\rho = (\tau\sigma\tau^{-1})\sigma \in N$, notiamo che
        $\rho \neq id$ in quanto
        \[
            \rho(a_4) = (\tau\sigma\tau^{-1})(\sigma(a_4)) = (\tau\sigma\tau^{-1})(a_5) = 
            a_6 \neq a_4
        \]
        d'altra parte $a_1$ è un punto fisso per $\rho$, che è assurdo;
        \item se $k > 1$ e $l > 2$, poiché $\sigma_1^{-1}$ è un $l$-ciclo
        disgiunto da $\sigma_2, \ldots, sigma_k$ la permutazione 
        $\rho = \sigma^{-1}\sigma_2\ldots\sigma_k$ è un elemento di $N$. 
        Consideriamo $\alpha = \rho\sigma \in N$, osserviamo che 
        \[
            \alpha = \sigma_2^2\ldots\sigma_k^2 \neq id
        \]
        in quanto $ord(\sigma_i) = l > 2$ per ogni $i \in \{1, \ldots, k\}$,
        tuttavia $a_1$ è un punto fisso per $\alpha$, che è assurdo;
        \item se $k > 1$ e $l = 2$, scriviamo $\sigma$ come prodotto di $k$
        trasposizioni disgiunte
        \[
            \sigma = \cycle{a_1, b_1}\cycle{a_k, b_k}
        \]
        Consideriamo la permutazione pari $\tau = \cycle{a_1, a_2, b_1}$,
        poiché $N$ è normale in $\mathcal{A}_{n + 1}$ contiene
        \[
            \rho = \tau\sigma\tau^{-1} = \cycle{a_2, a_1}\cycle{b_1, b_2}
            \cycle{a_3, b_3}\ldots\cycle{a_k, b_k}
        \]
        e anche la permutazione 
        \[
            \alpha = \rho\sigma = (\cycle{a_2, a_1}\cycle{b_1, b_2})
            (\cycle{a_1, b_1}\cycle{a_2, b_2}) = \cycle{a_1, b_2}\cycle{a_2, b_1} \neq id
        \]
        ma $a_3$ è un punto fisso per $\alpha$, che è assurdo.
    \end{itemize}
    Pertanto $N\cap H_i = \{id\}$, cioè $\mathcal{A}_{n + 1}$ è un gruppo semplice.
\end{proof}

\begin{corollary}
    L'insieme $X = \{\sigma \in S_n\mid \sigma \text{ è un 5-ciclo}\}$ genera
    $\mathcal{A}_n$ per $n \geqslant 5$.
\end{corollary}

\begin{proof}
    Sia $\sigma \in X$ un 5-ciclo, per ogni $\tau \in \mathcal{A}_n$
    abbiamo che $\tau\sigma\tau^{-1}$ è ancora un elemento di $X$, pertanto
    $\langle X\rangle$ è un sottogruppo normale di $\mathcal{A}_n$, da cui 
    $\langle X \rangle = \mathcal{A}_n$ in quanto diverso da $\{id\}$.
\end{proof}

\newpage

\subsection{Prodotti semidiretti}

\subsubsection{Descrizione di $S_4$ come prodotto semidiretto}

Per ogni $n \geq 2$ vale in generale la relazione 
\[
    S_n \cong \mathcal{A}_n \rtimes \langle\cycle{a, b}\rangle
\]
dove $\cycle{a, b}$ è una trasposizione di $S_n$, vogliamo però dare una
decomposizione di $S_4$ più specifica. \newline
Consideriamo il sottogruppo di Klein $V_4 = \{id, \cycle{1, 2}\cycle{3, 4}, \cycle{1, 3}\cycle{2, 4},
\cycle{1, 4}\cycle{2, 3}\}$ e $H = \{\sigma \in S_4\mid \sigma(4) = 4\}$ lo 
stabilizzatore di $4$ secondo l'azione naturale di $S_4$ su $\{1, 2, 3, 4\}$,
osserviamo che $V_4$ è normale in $S_4$ in quanto
unione delle classi di coniugio di ogni suo elemento\footnote{
    Le classe di coniugio in $S_4$ di $\cycle{1, 2}\cycle{3, 4}$ è 
    $\{\cycle{1, 2}\cycle{3, 4}, \cycle{1, 3}\cycle{2, 4}, \cycle{1, 4}\cycle{2, 3}\}$.
}
e che $H$ è isomorfo a $S_3$ (in effetti gli elementi di $H$ sono tutte e 
sole le permutazioni di 3 elementi). Dato che l'unica permutazione di $V_4$
che fissa 4 è l'identità abbiamo $V_4 \cap H = \{id\}$, inoltre $V_4H = S_4$
in quanto i due insiemi hanno la stessa cardinalità. Possiamo quindi scrivere
\[
    S_4 \cong V_4 \rtimes H \cong (\Z2 \times \Z2) \rtimes_{\varphi} S_3
\]

con

\[
    \varphi: S_3 \longrightarrow \Aut(\Z2\times\Z2)
\]

Specifichiamo come agisce la mappa $\varphi$\footnote{Se descriviamo $S_4$ 
    come prodotto semidiretto di due sottogruppi questo non è 
    necessario, in quanto tale mappa è sempre il coniugio.
}:
consideriamo gli isomorfismi
\[
    \alpha: V_4\longrightarrow \Z2\times\Z2: \cycle{1, 2}\cycle{3, 4}\longmapsto (1, 0),
    \cycle{1, 3}\cycle{2, 4} \longmapsto (0, 1)
\]
\[
    \beta: H \longrightarrow S_3: \sigma \longmapsto \sigma_{\mid\{1, 2, 3\}}
\]
le immagini di $\varphi$ in $\Aut(\Z2\times\Z2)$ corrispondono tramite $\alpha$
e $\beta$ ai coniugi su $V_4$ per elementi di $H$. Vediamo quindi come
i generatori $\cycle{1, 2, 3}$, $\cycle{1, 2}$ di $H$ agiscono per coniugio
sui generatori $\cycle{1, 2}\cycle{3, 4}$, $\cycle{1, 3}\cycle{2, 4}$ di $V_4$:
\[
    \cycle{1, 2, 3}(\cycle{1, 2}\cycle{3, 4})\cycle{1, 3, 2} = \cycle{1, 4}\cycle{2, 3}
\]
\[
    \cycle{1, 2, 3}(\cycle{1, 3}\cycle{2, 4})\cycle{1, 3, 2} = \cycle{1, 2}\cycle{3, 4}
\]
\[
    \cycle{1, 2}(\cycle{1, 2}\cycle{3, 4})\cycle{1, 2} = \cycle{1, 2}\cycle{3, 4}
\]
\[
    \cycle{1, 2}(\cycle{1, 3}\cycle{2, 4})\cycle{1, 2} = \cycle{1, 4}\cycle{1, 3}
\]
Pertanto $\varphi(\cycle{1, 2, 3}) = f$ e $\varphi(\cycle{1, 2}) = g$,
dove $f$ e $g$ sono gli automorfismi di $\Z2\times\Z2$ tali che
\[
    f: (1, 0)\longmapsto (1, 1), (0, 1) \longmapsto (1, 0)
\]
\[
    g: (1, 0)\longmapsto (1, 0), (0, 1) \longmapsto (1, 1)
\]

\subsubsection{Prodotti semidiretti isomorfi}

Dati due gruppi, può succedere che il loro prodotto diretto sia isomorfo a 
un loro prodotto semidiretto non banale.\newline
Consideriamo il gruppo $GL_3(\RR)$ e $N = SL_3(\RR) = \{M \in GL_3(\RR)
\mid \det M = 1\}$, $N$ è un sottogruppo normale di $GL_3(\RR)$ in quanto è il nucleo
dell'omomorfismo
\[
    \det: GL_3(\RR) \longrightarrow \RR^*
\]
mostriamo che $GL_3(\RR) \cong SL_3(\RR)\times\RR^*$. Consideriamo il sottogruppo
\[
    H = \left\{\begin{pmatrix}
    \lambda & 0 & 0\\
    0 & \lambda & 0\\
    0 & 0 & \lambda
    \end{pmatrix}\Bigm| \lambda \in \RR^* \right\}
\]isomorfo a $\RR^*$, abbiamo che:

\begin{itemize}
    \item $N \cap H = \{Id\}$ in quanto $M = \lambda Id \in N\cap H$ è tale
    che $\det M = \lambda^3 = 1$, cioè $\lambda = 1$ e quindi $M = Id$;
    \item $H$ è un sottogruppo normale di $GL_3(\RR)$, in quanto tutti i suoi elementi
    sono mutlipli scalari della matrice identità e quindi commutano con gli
    elementi di $GL_3(\RR)$;
    \item $GL_3(\RR) = NH$, infatti per ogni $M \in GL_3(\RR)$ possiamo scrivere $M = S(\lambda Id)$,
    dove $\lambda = \nolinebreak(\det M)^{\frac 1 3}$ e $S = (\det M)^{-\frac 1 3} M \in N$.
\end{itemize}

Possiamo quindi scrivere 
\[
    GL_3(\RR) \cong SL_3(\RR) \times H \cong SL_3(\RR)\times \RR^*
\]
Consideriamo adesso il sottogruppo di $GL_3(\RR)$

\[
    K = \left\{\begin{pmatrix}
        \lambda & 0 & 0\\
        0 & 1 & 0\\
        0 & 0 & 1
    \end{pmatrix}\Bigm| \lambda \in \RR^* \right\}
\]
anch'esso isomorfo a $\RR^*$. Ragionando in modo analogo abbiamo $N \cap H = \{Id\}$, 
inoltre $GL_3(\RR) = NK$ in quanto per ogni $M \in GL_3(\RR)$ possiamo scrivere
$M = (MA^{-1})A$ con 
\[
    A = \begin{pmatrix}
        \det M & 0 & 0\\
        0 & 1 & 0\\
        0 & 0 & 1
    \end{pmatrix} \in K, \quad MA^{-1} \in N
\]
Possiamo quindi scrivere 
\[
    GL_3(\RR) \cong SL_3(\RR)\rtimes K
\]
Notiamo che l'azione di coniugio di $K$ su $SL_3(\RR)$ non è banale, in 
quanto
\[
    \begin{pmatrix}
        \lambda & 0 & 0\\
        0 & 1 & 1\\
        0 & 0 & 1
    \end{pmatrix}
    \begin{pmatrix}
        1 & 1 & 0\\
        0 & 1 & 0\\
        0 & 0 & 1
    \end{pmatrix}
    \begin{pmatrix}
        \lambda^{-1}& 0 & 0\\
        0 & 1 & 0\\
        0 & 0 & 1
    \end{pmatrix} = 
    \begin{pmatrix}
        1 & \lambda & 0\\
        0 & 1 & 0\\
        0 & 0 & 1
    \end{pmatrix} \neq \begin{pmatrix}
        1 & 1 & 0\\
        0 & 1 & 0\\
        0 & 0 & 1
    \end{pmatrix},\quad \lambda \neq 0, 1
\]
quindi il prodotto non è diretto.

Vediamo adesso un criterio che stabilisce una condizione sufficiente affinché 
i prodotti semidiretti di due gruppi siano isomorfi.

\begin{proposition}
    \label{prop4.0}
    Siano $H, N$ due gruppi e $\varphi: H \longrightarrow \Aut(N)$ un omomorfismo.
    Dato $f \in \Aut(N)$ allora $N\rtimes_{\varphi} H \cong N\rtimes_{\varphi\circ f} H$.
\end{proposition}

\begin{proof}
    Consideriamo l'applicazione
    \[
        \psi: N\rtimes_{\varphi}H \longrightarrow N\rtimes_{\varphi\circ f}H:
        (n, h) \longmapsto (n, f^{-1}(h))
    \]
    $\psi$ è una bigezione tra i due insiemi in quanto $f$ è bigettiva,
    mostriamo che è anche un omomorfismo di gruppi. Per ogni $(n_1, h_1)$, $(n_2, h_2)
    \in N\rtimes_{\varphi}H$ abbiamo
    \begin{multline*}
        \psi((n_1, h_1)(n_2, h_2)) = \psi(n_1\cdot\varphi(h_1)(n_2), h_1h_2) = \\
        = (n_1\cdot\varphi(h_1)(n_2), f^{-1}(h_1h_2)) = (n_1\cdot\varphi(h_1)(n_2), f^{-1}(h_1)f^{-1}(h_2))
    \end{multline*}
    d'altra parte
    \begin{multline*}
        \psi(n_1, h_1)\psi(n_2, h_2) = (n_1, f^{-1}(h_1))(n_2, f^{-1}(h_2)) =\\
        = (n_1\cdot(\varphi\circ f)(f^{-1}(h_1))(n_2), f^{-1}(h_1)f^{-1}(h_2)) = 
        (n_1\cdot \varphi(h_1)(n_2), f^{-1}(h_1)f^{-1}(h_2))
    \end{multline*}
    cioè $\psi$ è un omomorfismo, quindi i due gruppi sono isomorfi.
\end{proof}

\begin{example}
    Abbiamo visto che i prodotti semidiretti della forma $\Zp \rtimes\Z{q}$
    con $p$, $q$ primi tali che $q \mid p - 1$ si suddividono in esattamente
    due classi di isomorfismo, utilizziamo il risultato appena mostrato 
    per verificare che tutti i prodotti semidiretti non banali sono tra loro 
    isomorfi. Consideriamo un omomorfismo
    \[
        \varphi_a: \Z{q}\longrightarrow \Z{(p - 1)} : 1 \longmapsto a
    \]
    con $\ord(a) = q$ (poiché $\Aut(\Zp) \cong \Z{(p - 1)}$
    questo è equivalente a fissare un omomorfismo tra $\Z{q}$ e $\Aut(\Zp))$,
    possiamo scrivere 
    \[
        a = k\frac{p - 1}{q}\quad k \in \{1, \ldots, q - 1\}
    \]
    Posto $f_k \in \Aut(\Z{q})$ tale che
    \[
        f_k :\Z{q}\longrightarrow\Z{q}: x \longmapsto kx
    \]
    con $(k, q) = 1$, possiamo scrivere 
    $\varphi_a = \varphi_{\frac{p - 1}{q}}\circ f_k$. Allora i prodotti
    semidiretti non banali $\Zp\rtimes_{\varphi_a}\Z{q}$ sono tutti isomorfi
    a tra loro per la \hyperref[prop4.0]{Proposizione 1.65}.
\end{example}

\newpage

\subsection{Classificazione dei gruppi semplici di ordine al più 100}

In questa sezione vogliamo determinare quali sono i sottogruppi semplici di 
ordine minore o uguale a 100. Facciamo prima una serie di osservazioni che 
ci permetterà di ridurre lo studio a pochi casi interessanti.

\begin{itemize}
    \item Gli unici gruppi abeliani semplici sono i gruppi $\Zp$ con $p$ primo,
    in quanto i loro sottogruppi sono solo quelli banali e tutti i sottogruppi
    di un gruppo abeliano sono normali;
    \item i gruppi $G$ di ordine $p^k$ con $p$ primo e $k > 1$ non sono semplici
    in quanto hanno centro non banale e il centro è un sottogruppo caratteristico,
    in particolare normale (alternativamente, dal Teorema di Sylow abbiamo che 
    $G$ contiene un sottogruppo proprio di ordine $p^{k - 1}$, che è normale in quanto
    il suo indice è $p$, il più piccolo primo che divide $|G|$);
    \item i gruppi di ordine $2d$ con $d$ dispari non sono semplici in quanto
    contengono un sottogruppo di indice $2$, che è normale e non banale, per 
    l'\hyperref[ex1.0]{Esercizio 1.48};
    \item i gruppi di ordine $pq$ con $q > p$ primi non sono semplici, in
    quanto possiamo scriverli come prodotto semidiretto dei loro sottogruppi
    di Sylow, pertanto almeno uno di questi è normale e non banale;
    \item $\mathcal{A}_5$ è un gruppo semplice di ordine $60$.
\end{itemize}

Ci riduciamo quindi a studiare i gruppi di ordine 56, 60, 72, 80, 96.
\newline\newline
\underline{$|G| = 56 = 2^3\cdot 7$}: poiché $n_7 \equiv 1 \pmod 7$ e $n_7 \mid 56$ abbiamo
$n_7 \in \{1, 8\}$. Se $n_7 = 1$ allora $G$ contiene un unico 7-Sylow, chè 
è quindi un sottogruppo proprio normale di $G$. Se $n_7 = 8$ allora $G$ contiene
$6\cdot8 = 48$ elementi di ordine 7 (dato che i 7-Sylow di $G$ sono isomorfi a $\Z7$)
pertanto i restanti $7$ elementi non banali devono essere contenuti in un 
unico 2-Sylow, che è quindi normale. In entrambi i casi $G$ non è semplice.
\newline\newline
\underline{$|G| = 96 = 2^5\cdot 3$}: sia $P_2$ un 2-Sylow di $G$, poiché 
$[G:P_2] = 3$ per il \hyperref[teorema3.0]{Teorema 1.50} esiste un sottogruppo
$N \triangleleft G$ tale che $N \subseteq P_2$ e $[G:N] \mid 3!$, da cui
$N \neq G$ e $N \neq \{e\}$ in quanto $[G:\{e\}] = |G|$. Pertanto $G$ non 
è semplice.
\newline\newline
\underline{$|G| = 72 = 2^3 \cdot 3^2$}: dalle condizioni 
\[
    \begin{cases}
        n_2 \equiv 1 \pmod 2\\
        n_2 \mid 72
    \end{cases}\quad
    \begin{cases}
        n_3 \equiv 1 \pmod 3\\
        n_3 \mid 72
    \end{cases}
\]
otteniamo $n_2 \in \{1, 3, 9\}$ e $n_3 \in \{1, 4\}$, distinguiamo quindi
due casi.
\begin{itemize}
    \item Se $n_3 = 1$ allora $G$ contiene un unico 3-Sylow, che è quindi un
    sottogruppo normale non banale di $G$, cioè $G$ non è semplice;
    \item se $n_3= 4$, siano $Q_1$, $Q_2$, $Q_3$, $Q_4$ i 3-Sylow di $G$ e 
    $X = \{Q_1, Q_2, Q_3, Q_4\}$, consideriamo l'azione di coniugio di $G$ su $X$
    \[
        \varphi: G\longrightarrow S(X) \cong S_4
    \]
    poiché i 3-Sylow di $G$ sono tutti coniugati tale azione è transitiva.
    Mostriamo che $\ker \varphi$ è un sottogruppo di $G$ non banale. Se $\ker\varphi
    = \{e\}$ allora $\varphi$ sarebbe un omomorfismo iniettivo, che è assurdo
    in quanto l'ordine di $G$ non divide l'ordine di $S(X) \cong S_4$. D'altra
    parte se fosse $\ker\varphi = G$ allora $\varphi$ sarebbe l'azione banale,
    che è assurdo in quanto $\varphi$ è transitiva e $|X| > 1$ (alternativamente,
    se $\varphi$ fosse l'azione banale allora i 3-Sylow di $G$ sarebbero tutti
    normali). Pertanto $\ker\varphi$ è un sottogruppo normale non banale di $G$,
    cioè $G$ non è semplice.
\end{itemize}
\underline{$|G| = 80 = 2^4\cdot 5$}: dalle condizioni
\[
    \begin{cases}
        n_2 \equiv 1 \pmod 2\\
        n_2 \mid 80
    \end{cases}\quad
    \begin{cases}
        n_5 \equiv 1 \pmod 5\\
        n_5 \mid 80
    \end{cases}
\]
otteniamo $n_2 \in \{1, 5\}$ e $n_5 \in \{1, 16\}$, distinguiamo quindi due
casi.
\begin{itemize}
    \item Se $n_5 = 1$ allora $G$ contiene un unico 5-Sylow, che è quindi un
    sottogruppo normale non banale di $G$, cioè $G$ non è semplice;
    \item se $n_5 = 16$ allora $G$ contiene $4\cdot 16 = 64$ elementi di ordine
    $5$ (dato che i 5-Sylow di $G$ sono isomorfi a $\Z5$), pertanto i restanti
    15 elementi devono esserre contenuti in un unico 2-Sylow, che è quindi normale.
    Allora $G$ non è semplice. \newline Alternativamente, consideriamo $P_2$ un 2-Sylow
    e l'azione di moltiplicazione a sinistra di $G$ sull'insieme quoziente $\faktor{G}{P_2}$
    \[
        \varphi: G\longrightarrow S\left(\faktor{G}{P_2}\right) \cong S_5
    \]
    Poiché $|G| \nmid |S_5|$ abbiamo $\ker\varphi\neq\{e\}$, d'altra parte
    $\ker\varphi \neq G$ in quanto $\varphi$ è un'azione transitiva (per 
    ogni $x, y \in G$ vale $\varphi(xy^{-1})(yP_2) = xy^{-1}yP_2 = xP_2)$.
    Quindi $\ker\varphi$ è un sottogruppo normale di $G$ non banale, cioè $G$
    non è semplice. 
\end{itemize}

Rimangono da studiare i gruppi di ordine $60$, vogliamo dimostrare che 
$\mathcal{A}_5$ è l'unico sottogruppo semplice di tale ordine (a meno di isomorfismo).

\begin{proposition}
    Se $G$ è un gruppo semplice di ordine $60$ allora è isomorfo a $\mathcal{A}_5$.
\end{proposition}



\end{document}