\documentclass[11pt]{scrartcl}
\usepackage[italian]{babel}
\usepackage[sexy]{evan}

\begin{document}
\title{Complementi di Algebra 1}
\maketitle
\newpage

\tableofcontents
\eject

\newpage

\section{Insiemi di generatori}

\begin{definition}
    Dati un gruppo $G$ e $x_1, \dots, x_n$ elementi di $G$, chiamiamo \vocab{sottogruppo 
    generato} da $x_1, \dots, x_n$ il più piccolo sottogruppo $\left< x_1, \dots x_n
    \right>$ di $G$ contenente $x_1, \dots, x_n$, cioè \[\left<x_1, \dots, x_n\right> =
    \bigcap_{\substack{H\leq G\\ \{x_1, \dots, x_n\} \subseteq H}} H\] 
\end{definition}

\begin{remark}
    La definizione è ben posta, infatti l'intersezione avviene su una 
    famiglia non vuota di insiemi dal momento che $G$ è un sottogruppo di 
    $G$ contenente $x_1, \dots, x_n$. Inoltre l'intersezione non è vuota in 
    quanto contiene almeno l'identità e gli elementi $x_1, \dots, x_n$.
\end{remark}

La definizione data non dà informazioni su come sono fatti gli elementi di 
$\left<x_1, dots, x_n\right>$, cerchiamo quindi di caratterizzare in modo
diverso tale sottogruppo. In quanto sottogruppo, $\left<x_1, \dots, x_n\right>$
deve contenere tutti i prodotti finiti, in qualsiasi ordine, delle potenze di
$x_1, \dots, x_n$, cioè deve contenere l'insieme 
\[\{g_1^{\pm 1}, \dots, g_r^{\pm 1}\mid r \in \NN, g_i \in \{x_1, \dots, x_n\}
~\forall i \in \{1, \dots, r\}\}\]

\begin{proposition}
\label{prop1.0}
Dati un gruppo $G$ e $x_1, \dots, x_n$ elementi di $G$, allora \[
    \left<x_1, \dots, x_n\right> = \{g_1^{\pm 1}, \dots, g_r^{\pm 1}\mid r 
    \in \NN, g_i \in \{x_1, \dots, x_n\}~\forall i \in \{1, \dots, r\}\}.
    \]
\end{proposition}

\begin{proof}
Poniamo $S = \{g_1^{\pm 1}, \dots, g_r^{\pm 1}\mid r \in \NN, g_i \in \{x_1, \dots, x_n\}
~\forall i \in \{1, \dots, r\}\}$, mostriamo che $S$ è un sottogruppo di $G$. 
Effettivamente $e \in S$ in quanto è prodotto nessuna potenza di $x_1, \dots, x_n$, 
il prodotto di due elementi di $S$ è ancora un elemento di $S$ in quanto
prodotto finito di potenze di $x_1, \dots, x_n$ e l'inverso di un elemento
$g_1^{\pm 1}\dots g_r^{\pm 1} \in\nolinebreak S$ è $(g_1^{\pm 1}\dots 
g_r^{\pm 1})^{-1} = g_r^{\mp 1}\dots g_1^{\mp 1}$, che è un elemento di $S$.
Abbiamo quindi che $S$ è un sottogruppo di $G$ contenente $x_1, \dots, x_n$,
pertanto $\left<x_1, \dots, x_n\right>\subseteq S$ per minimalità di $\left<x_1,
\dots, x_n\right>$. D'altra parte, per quanto osservato sopra abbiamo che
tutti gli elementi della forma $g_1^{\pm 1}\dots g_r^{\pm 1}$ con $r \in \NN$, 
$g_i \in \{x_1, \dots, x_n\}$ per ogni $i \in \{1, \dots, r\}$ devono essere
contenuti in $\left<x_1, \dots, x_n\right>$, pertanto i due sottogruppi
coincidono.
\end{proof}

\begin{remark}
    Se $G$ è un gruppo ciclico abbiamo che esiste $x \in G$ tale che 
    $\left<x\right> = G$, cioè tutti gli elementi di $G$ sono potenze di $x$.
\end{remark}

Diciamo che $x_1, \dots, x_n \in G$ sono \vocab{generatori} per $G$, o che 
l'insieme $\{x_1, \dots, x_n\}$ \vocab{genera} $G$ se $\left<x_1, \dots, x_n\right> = G$.



\section{Gruppo diedrale}

\subsection{Elementi del gruppo}

\begin{definition}
    Dato $n \geq 2$ un naturale, consideriamo un poligono regolare di $n$ vertici,
    definiamo il \vocab{gruppo diedrale} su $n$ vertici $D_n$ come l'insieme 
    delle isometrie del piano che mandano i vertici in se stessi, cioè che 
    fissano il poligono (per $n = 2$ consideriamo le isometrie che mandano un 
    segmento si se stesso).
\end{definition}

\begin{remark}
    $D_n$ è effettivamente un gruppo, in quanto l'applicazione identità che 
    fissa tutti i vertici è un'isometria dal poligono in se stesso, la 
    composizione di isometrie è un'isometria e un'isometria ammette sempre 
    un'inversa, che è anch'essa un'isometria.
\end{remark}

\begin{remark}
    Una rotazione di angolo $\displaystyle\frac{2\pi}{5}$ è un elemento di $D_n$,
    così come una simmetria rispetto a un asse.
\end{remark}

Proseguendo con questa intuizione geometrica, indicheremo con $r$ una rotazione
di angolo $\displaystyle \frac{2\pi}{n}$ e con $s$ una simmetria rispetto a
un qualsiasi asse, notiamo che $\ord(r) = n$ e $\ord(s) = 2$ (per convenzione, 
indichiamo con un angolo positivo una rotazione in senso antiorario e con un 
angolo negativo una rotazione in senso orario).

\begin{definition}
    Data $r \in D_n$ una rotazione di ordine $n$, indichiamo con $\mathcal{R}$ il
    \vocab{sottogruppo delle rotazioni} $\left<r\right>$.
\end{definition}

\begin{remark}
    Il sottogruppo $\mathcal{R}$ contiene tutte le rotazioni di $D_n$, infatti
    se $r'$ è una rotazione di angolo $\displaystyle\frac{2k\pi}{n}$, $k \in \ZZ$,
    allora $r^k = r'$ in quanto anche $r^k$ è una rotazione di angolo 
    $\displaystyle\frac{2k\pi}{n}$.
\end{remark}

Per determinare come sono fatti gli elementi di $D_n$, studiamo il sottogruppo
$\left<r, s\right>$. Sicuramente $\left<r, s\right>$ contiene il sottogruppo $\mathcal{R}$
e tutti gli elementi della forma $sr^k$, $sr^ks$, $sr^ksr^h$ e così via, vogliamo
mostrare che in effetti $D_n$ è generato da $r$ e $s$

\begin{remark}
    Gli elementi della forma $r^k$ e $sr^h$ sono distinti per ogni $h, k \in \ZZ$. 
    Infatti sappiamo dall'algebra lineare che il determinante di una simmetria
    è $-1$ mentre il determinante di una rotazione è $1$, per la moltiplicatività
    del determinante abbiamo quindi $\det (r^k) = (\det r)^k = 1$ e
    $\det (sr^h) = (\det s)(\det r)^h = -1$, cioè $r^k \neq sr^h$.
\end{remark}

\begin{lemma}
    \label{lemma1.0}
    Per ogni rotazione $r \in D_n$ e per ogni simmetria $s \in D_n$ vale
    \[srs^{-1} = r^{-1}.\]
\end{lemma}

\begin{proof}
    $srs^{-1} = r^{-1} \iff sr = r^{-1}s = (s^{-1}r)^{-1}$. Si conclude
    osservando che $s^2 = 1$, pertanto $s^{-1} = s$ e $(s^{-1}r)^{-1} =
    (sr)^{-1} = r^{-1}s^{-1} = r^{-1}s$.
\end{proof}

\begin{proposition}
    \label{prop2.0}
    Se $n \geq 3$ allora $|D_n| = 2n$.
\end{proposition}

\begin{proof}
    Indicando con $1, \dots, n$ gli $n$ vertici di un poligono regolare, notiamo
    che un elemento $g \in D_n$ è univocamente determinato da $g(1), \dots, g(n)$.
    In particolare, fissato $g(1)$, per il quale abbiamo $n$ possibili scelte,
    abbiamo al massimo due valori per $g(2)$, cioè $g(2) \in \{g(1) + 1, g(1) - 1\}$
    (a meno di sommare $n$ se uno dei due elementi è negativo). Poiché $g(1)$
    e $g(2)$ determinano due vettori nel piano non allineati, che sono quindi
    linearmente indipendenti e determinano una base del piano. Una volta 
    determinati i valori di $g(1)$ e $g(2)$ abbiamo quindi determinato ogni
    elemento di $D_n$ in modo unico e, poiché possiamo farlo in al più $2n$ modi, 
    abbiamo che $|D_n| \leq 2n$. Ricordiamo adesso che $D_n$ contiene gli elementi
    della forma $r^k$, $sr^h$ per $h, k \in \ZZ$, mostriamo che questi sono 
    infatti $2n$: gli elementi $r^k$ appartengono al gruppo ciclico $\mathcal{R}$
    di ordine $r$, pertanto sono $n$ elementi distinti. Inoltre $sr^i = sr^j
    \iff r^i = r^j\iff i \equiv j \mod n$, pertanto anche questi sono $n$
    elementi distinti. Allora $|D_n| = 2n$.
\end{proof}

\end{document}