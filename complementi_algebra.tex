\documentclass[11pt]{scrartcl}
\usepackage[italian]{babel}
\usepackage[sexy]{evan}

\begin{document}
\title{Complementi di Algebra 1}
\maketitle
\newpage

\tableofcontents
\eject

\newpage

\section{Insiemi di generatori}

\begin{definition}
    Dati un gruppo $G$ e $x_1, \dots, x_n$ elementi di $G$, chiamiamo \vocab{sottogruppo 
    generato} da $x_1, \dots, x_n$ il più piccolo sottogruppo $\langle x_1, \dots x_n
    \rangle$ di $G$ contenente $x_1, \dots, x_n$, cioè \[\langle x_1, \dots, x_n\rangle =
    \bigcap_{\substack{H\leq G\\ \{x_1, \dots, x_n\} \subseteq H}} H\] 
\end{definition}

\begin{remark}
    La definizione è ben posta, infatti l'intersezione avviene su una 
    famiglia non vuota di insiemi dal momento che $G$ è un sottogruppo di 
    $G$ contenente $x_1, \dots, x_n$. Inoltre l'intersezione non è vuota in 
    quanto contiene almeno l'identità e gli elementi $x_1, \dots, x_n$.
\end{remark}

La definizione data non dà informazioni su come sono fatti gli elementi di 
$\langle <x_1, dots, x_n\rangle$, cerchiamo quindi di caratterizzare in modo
diverso tale sottogruppo. In quanto sottogruppo, $\langle x_1, \dots, x_n\rangle$
deve contenere tutti i prodotti finiti, in qualsiasi ordine, delle potenze di
$x_1, \dots, x_n$, cioè deve contenere l'insieme 
\[\{g_1^{\pm 1}, \dots, g_r^{\pm 1}\mid r \in \NN, g_i \in \{x_1, \dots, x_n\}
~\forall i \in \{1, \dots, r\}\}\]

\begin{proposition}
\label{prop1.0}
Dati un gruppo $G$ e $x_1, \dots, x_n$ elementi di $G$, allora \[
    \langle <x_1, \dots, x_n\rangle = \{g_1^{\pm 1}, \dots, g_r^{\pm 1}\mid r 
    \in \NN, g_i \in \{x_1, \dots, x_n\}~\forall i \in \{1, \dots, r\}\}.
    \]
\end{proposition}

\begin{proof}
Poniamo $S = \{g_1^{\pm 1}, \dots, g_r^{\pm 1}\mid r \in \NN, g_i \in \{x_1, \dots, x_n\}
~\forall i \in \{1, \dots, r\}\}$, mostriamo che $S$ è un sottogruppo di $G$. 
Effettivamente $e \in S$ in quanto è prodotto nessuna potenza di $x_1, \dots, x_n$, 
il prodotto di due elementi di $S$ è ancora un elemento di $S$ in quanto
prodotto finito di potenze di $x_1, \dots, x_n$ e l'inverso di un elemento
$g_1^{\pm 1}\dots g_r^{\pm 1} \in\nolinebreak S$ è $(g_1^{\pm 1}\dots 
g_r^{\pm 1})^{-1} = g_r^{\mp 1}\dots g_1^{\mp 1}$, che è un elemento di $S$.
Abbiamo quindi che $S$ è un sottogruppo di $G$ contenente $x_1, \dots, x_n$,
pertanto $\langle x_1, \dots, x_n\rangle\subseteq S$ per minimalità di $\langle <x_1,
\dots, x_n\rangle$. D'altra parte, per quanto osservato sopra abbiamo che
tutti gli elementi della forma $g_1^{\pm 1}\dots g_r^{\pm 1}$ con $r \in \NN$, 
$g_i \in \{x_1, \dots, x_n\}$ per ogni $i \in \{1, \dots, r\}$ devono essere
contenuti in $\langle <x_1, \dots, x_n\rangle$, pertanto i due sottogruppi
coincidono.
\end{proof}

\begin{remark}
    Se $G$ è un gruppo ciclico abbiamo che esiste $x \in G$ tale che 
    $\langle x\rangle = G$, cioè tutti gli elementi di $G$ sono potenze di $x$.
\end{remark}

Diciamo che $x_1, \dots, x_n \in G$ sono \vocab{generatori} per $G$, o che 
l'insieme $\{x_1, \dots, x_n\}$ \vocab{genera} $G$ se $\langle x_1, \dots, x_n\rangle = G$.



\section{Gruppo diedrale}

\subsection{Elementi del gruppo}

\begin{definition}
    Dato $n \geq 2$ un naturale, consideriamo un poligono regolare di $n$ vertici,
    definiamo il \vocab{gruppo diedrale} su $n$ vertici $D_n$ come l'insieme 
    delle isometrie del piano che mandano i vertici in se stessi, cioè che 
    fissano il poligono (per $n = 2$ consideriamo le isometrie che mandano un 
    segmento si se stesso).
\end{definition}

\begin{remark}
    $D_n$ è effettivamente un gruppo, in quanto l'applicazione identità che 
    fissa tutti i vertici è un'isometria dal poligono in se stesso, la 
    composizione di isometrie è un'isometria e un'isometria ammette sempre 
    un'inversa, che è anch'essa un'isometria.
\end{remark}

\begin{remark}
    Una rotazione di angolo $\displaystyle\frac{2\pi}{5}$ è un elemento di $D_n$,
    così come una simmetria rispetto a un asse.
\end{remark}

Proseguendo con questa intuizione geometrica, indicheremo con $r$ una rotazione
di angolo $\displaystyle \frac{2\pi}{n}$ e con $s$ una simmetria rispetto a
un qualsiasi asse, notiamo che $\ord(r) = n$ e $\ord(s) = 2$ (per convenzione, 
indichiamo con un angolo positivo una rotazione in senso antiorario e con un 
angolo negativo una rotazione in senso orario).

\begin{definition}
    Data $r \in D_n$ una rotazione di ordine $n$, indichiamo con $\mathcal{R}$ il
    \vocab{sottogruppo delle rotazioni} $\langle r\rangle$.
\end{definition}

\begin{remark}
    Il sottogruppo $\mathcal{R}$ contiene tutte le rotazioni di $D_n$, infatti
    se $r'$ è una rotazione di angolo $\displaystyle\frac{2k\pi}{n}$, $k \in \ZZ$,
    allora $r^k = r'$ in quanto anche $r^k$ è una rotazione di angolo 
    $\displaystyle\frac{2k\pi}{n}$.
\end{remark}

Per determinare come sono fatti gli elementi di $D_n$, studiamo il sottogruppo
$\langle r, s\rangle$. Sicuramente $\langle r, s\rangle$ contiene il sottogruppo $\mathcal{R}$
e tutti gli elementi della forma $sr^k$, $sr^ks$, $sr^ksr^h$ e così via, vogliamo
mostrare che in effetti $D_n$ è generato da $r$ e $s$

\begin{remark}
    Gli elementi della forma $r^k$ e $sr^h$ sono distinti per ogni $h, k \in \ZZ$. 
    Infatti sappiamo dall'algebra lineare che il determinante di una simmetria
    è $-1$ mentre il determinante di una rotazione è $1$, per la moltiplicatività
    del determinante abbiamo quindi $\det (r^k) = (\det r)^k = 1$ e
    $\det (sr^h) = (\det s)(\det r)^h = -1$, cioè $r^k \neq sr^h$.
\end{remark}

\begin{lemma}
    Per ogni rotazione $r \in D_n$ e per ogni simmetria $s \in D_n$ vale
    \[srs^{-1} = r^{-1}.\]
\end{lemma}

\begin{proof}
    $srs^{-1} = r^{-1} \iff sr = r^{-1}s = (s^{-1}r)^{-1}$. Si conclude
    osservando che $s^2 = 1$, pertanto $s^{-1} = s$ e $(s^{-1}r)^{-1} =
    (sr)^{-1} = r^{-1}s^{-1} = r^{-1}s$.
\end{proof}

\begin{proposition}
    \label{prop2.0}
    Se $n \geq 3$ allora $|D_n| = 2n$.
\end{proposition}

\begin{proof}
    Indicando con $1, \dots, n$ gli $n$ vertici di un poligono regolare, notiamo
    che un elemento $g \in D_n$ è univocamente determinato da $g(1), \dots, g(n)$.
    In particolare, fissato $g(1)$, per il quale abbiamo $n$ possibili scelte,
    abbiamo al massimo due valori per $g(2)$, cioè $g(2) \in \{g(1) + 1, g(1) - 1\}$
    (a meno di sommare $n$ se uno dei due elementi è negativo). Poiché $g(1)$
    e $g(2)$ determinano due vettori nel piano non allineati, che sono quindi
    linearmente indipendenti e determinano una base del piano. Una volta 
    determinati i valori di $g(1)$ e $g(2)$ abbiamo quindi determinato ogni
    elemento di $D_n$ in modo unico e, poiché possiamo farlo in al più $2n$ modi, 
    abbiamo che $|D_n| \leq 2n$. Ricordiamo adesso che $D_n$ contiene gli elementi
    della forma $r^k$, $sr^h$ per $h, k \in \ZZ$, mostriamo che questi sono 
    infatti $2n$: gli elementi $r^k$ appartengono al gruppo ciclico $\mathcal{R}$
    di ordine $r$, pertanto sono $n$ elementi distinti. Inoltre $sr^i = sr^j
    \iff r^i = r^j\iff i \equiv j \mod n$, pertanto anche questi sono $n$
    elementi distinti. Allora $|D_n| = 2n$.
\end{proof}

\begin{remark}
    Abbiamo mostrato che effettivamente $D_n = \langle r, s\rangle$, quindi i
    suoi elementi sono tutti della forma $r^k$, $sr^h$. 
\end{remark}


\subsection{Sottogruppi}

Consideriamo un sottogruppo $H\leq D_n$, abbiamo due casi distinti: 
$H \subseteq \mathcal{R}$ oppure $H \nsubseteq \mathcal{R}$. Nel primo caso
abbiamo che $|H|\mid n$, ed è l'unico sottogruppo di $\mathcal{R}$ con questa 
proprietà in quanto $\mathcal{R}$ è ciclico, in particolare $H$ è ciclico 
della forma $\langle ^{\frac n d}\rangle$, con $d \mid n$. \newline
Studiamo quindi il caso $H \nsubseteq \mathcal{R}$. Osserviamo che 
$\mathcal{R}\trianglelefteqslant D_n$ in quanto $[D_n : \mathcal{R}] = 2$,
pertanto il gruppo $\faktor{D_n}{\mathcal{R}}$ è ben definito e risulta essere 
isomorfo a $\Z2$. \newline
Consideriamo la proiezione al quoziente \[
    \pi_{\mathcal{R}}: D_n \longrightarrow \faktor{D_n}{\mathcal{R}} : g \mapsto [g],
\]poiché $H \nsubseteq \mathcal{R}$ abbiamo che esiste $h \in H$ tale che 
$h \notin \mathcal{R}$, pertanto $\pi_{\mathcal{R}}(h) \notin [\mathcal{R}]$ e
in particolare $\pi_{\mathcal{R}}(H) \nsubseteq [\mathcal{R}]$. Dato che i 
sottogruppi di $\faktor{D_n}{\mathcal{R}}$ sono solo $\{[\mathcal{R}]\}$ e
$\faktor{D_n}{\mathcal{R}}$ abbiamo quindi $\pi_{\mathcal{R}}(H) = 
\faktor{D_n}{\mathcal{R}}$. Osserviamo che $\ker \pi_{\mid H} = 
\ker \pi \cap H = \mathcal{R}\cap H$, per il Primo Teorema di Omomorfismo
allora $\faktor{H}{H\cap \mathcal{R}} \cong \Z2$, quindi 
$|H\cap\mathcal{R}| = \displaystyle\frac 1 2 |H|$. Dato che $R\cap H \subseteq
\mathcal{R}$, esiste $k \in \ZZ$ tale che $H\cap\mathcal{R} = \langle r^k\rangle$
in particolare $\langle r^k\rangle$ e $\langle sr^h\rangle$, $h \in \ZZ$, sono
contenuti in $H$. 

\begin{proposition}
    \label{prop2.0}
    Dati $H\leq D_n$ un sottogruppo tale che $H\nsubseteq \mathcal{R}$, se
    $r \in \mathcal{R}$ è tale che $H\cap\mathcal{R} = \langle r^k\rangle$ 
    e $s$ è una simmetria allora \[
    H = \langle r^k\rangle\cdot\langle sr^h\rangle = \{xy \mid x \in \langle r^k\rangle,
    y \in \langle sr^h\rangle\}, h, k \in \ZZ.    
    \]
\end{proposition}

\begin{proof}
    Per quanto visto sopra, abbiamo che $|\langle r^k\rangle| = 
    \displaystyle\frac 1 2|H|$, osserviamo inoltre che $(sr^h)^2 = sr^hsr^h = 
    (srs)^hr^h = (srs^{-1})^hr^h = r^{-h}r^h = e$, pertanto 
    $\langle sr^h\rangle \cong \Z2$. Ricordiamo che, se $K, N$ sono sottogruppi
    di un gruppo $G$, se vale almeno una delle inclusioni $K \subseteq N_G(N)$,
    $N \subseteq N_G(K)$. Nel nostro caso abbiamo che $\langle sr^h\rangle
    \subseteq N_{D_n}(\langle r^k\rangle)$, infatti per ogni $m \in \ZZ$ abbiamo\[
        (sr^h)r^{mk}(sr^h)^{-1} = sr^{h + mk} sr^h = r^{-h-mk}r^h = r^{-mk}
        \in \langle r^k \rangle,
    \]cioè $\langle sr^h\rangle \subseteq N_{D_n}(\langle r^k\rangle)$, quindi
    $\langle r^k\rangle\cdot\langle sr^h\rangle$ è un sottogruppo di $D_n$.
    Poiché $\langle r^k\rangle$ e $\langle sr^h\rangle$ sono contenuti in $H$
    abbiamo che $\langle r^k\rangle\cdot\langle sr^h\rangle \subseteq H$, inoltre
    \[|\langle r^k\rangle\cdot\langle sr^h\rangle| = \displaystyle\frac 1 2 |H|\cdot 2 = |H|\]
    in quanto $\langle r^k\rangle\cap\langle sr^h\rangle = \{e\}$, quindi 
    i due sottogruppi coincidono.
\end{proof}

\begin{remark}
    Per $k \mid n$ e $0\leq h < k$, i sottogruppi $H_{k, h} = \langle r^k, sr^h\rangle$
    e $H = \langle r^k\rangle\cdot\langle sr^h\rangle$ coincidono. Infatti 
    $H_{k, h}\subseteq H$ in quanto $r^k, sr^h$ sono elementi di $H$, 
    d'altra parte $H \subseteq H_{k, h}$ in quanto $H_{h, k}$ contiene tutti i 
    prodotti finiti delle potenze di $r^k$ e $sr^h$, in particolare gli elementi di $H$.
\end{remark}

\begin{remark}
    Per $k \mid n$ e $0\leq h < k$, $\langle r^k, sr^h\rangle = 
    \langle r^k, sr^{h + k}\rangle$, infatti $\langle r^k, sr^h\rangle \subseteq
    \langle r^k, sr^{h + k}\rangle$ in quanto $sr^h = (sr^{h + k})r^{-k}$ è
    un elemento del secondo gruppo, simmetricamente $\langle r^k, sr^{h + k}\rangle
    \subseteq \langle r^k, sr^h\rangle$ in quanto $sr^{h + k} = (sr^h)r^k$ è un
    elemento del primo gruppo.
\end{remark}

Abbiamo quindi finito la classificazione dei sottogruppi di $D_n$.

\begin{theorem}
    [Classificazione dei sottogruppi di $D_n$]
I sottogruppi di $D_n$ sono della forma \begin{itemize}
    \item[(1)] $\langle r^k\rangle$ con $k\mid n$;
    \item[(2)] $\langle r^k, sr^h\rangle$ con $k \mid n$, $0\leq h < k$, 
\end{itemize}
con $r \in \mathcal{R}$ e $s$ una simmetria. Inoltre tali sottogruppi sono
tutti distinti.
\end{theorem}

\begin{proof}
    Abbiamo già visto che i sottogruppi di $D_n$ hanno una di queste forme, 
    mostriamo quindi che sono tutti distinti. A meno di cambiare $k$, possiamo
    supporre che $r$ generi $\mathcal{R}$, cioè $\ord(r) = n$. 
    Consideriamo $H, K\leq D_n$ due sottogruppi, distinguiamo tre casi 
    \begin{itemize}
        \item se $H = \langle r^k\rangle$ e $K = \langle r^m\rangle$, $m \in \ZZ$,
        allora $H = K\iff k = m$ in quanto entrambi sottogruppi di un gruppo 
        ciclico, pertanto esiste un unico sottogruppo della forma $\langle r^k\rangle$
        per ogni $k \mid n$;
        \item se $H = \langle r^k\rangle$ e $K = \langle sr^h\rangle$ allora $H \neq K$
        in quanto $H$ è ciclico e $K$ no;
        \item se $H = \langle r^k, sr^h\rangle$ e $K = \langle r^m, sr^l\rangle$, 
        con $m \mid n$ e $0\leq l < m$, considerando le intersezioni con $\mathcal{R}$ 
        $H \cap \mathcal{R} = \langle r^k\rangle$ e $K \cap \mathcal{R} = \langle r^m\rangle$ 
        abbiamo \[
        H \cap \mathcal{R} = K\cap\mathcal{R} \iff \langle r^k\rangle = \langle r^m\rangle
        \iff k = m.
        \] Inoltre, se $sr^h \in \langle r^m, sr^l\rangle = \langle r^m\rangle
        \cdot \langle sr^l\rangle$, allora esiste $t \in \ZZ$ tale che \[
        sr^h = (r^m)^t sr^l \iff sr^h = s^2r^{mt}sr^l \iff r^h = r^{-mt + l}
        \iff h \equiv l - mt \mod n,
        \]da cui ricaviamo $h \equiv l \mod m$. Ma allora $h =l$ in quanto 
        $0 \leq h < k$, $0\leq l < m$.
    \end{itemize}
\end{proof}

\begin{lemma}
    \label{lemma1}
    Dati un gruppo $G$ e $A, B$ due sottogruppi tali che $A \leq B \leq G$,
    se $B\trianglelefteqslant G$ e $A$ è caratteristico in $B$ allora 
    $A \trianglelefteqslant G$.
\end{lemma}

\begin{proof}
    Fissato $g \in G$, consideriamo l'omomorfismo di coniugio 
    \[
        \varphi_g : G\longrightarrow G : x\longmapsto gxg^{-1},
    \] poiché 
    $B\trianglelefteqslant G$ è ben definita la restrizione \[
        \varphi_{g\mid B} : B\longrightarrow B :  b \longmapsto gbg^{-1},
    \]in particolare $\varphi_{g\mid B} \in Aut(G)$. Dal momento che $A$ è
    un sottogruppo caratteristico di $B$ abbiamo che $\varphi_{g\mid B}(A) = A$,
    pertanto $A \trianglelefteqslant G$.
\end{proof}


\begin{corollary}
    Ogni sottogruppo di $\mathcal{R}$ è normale in $D_n$.
\end{corollary}

\begin{proof}
    Siano $\langle r^k\rangle$ un sottogruppo di $\mathcal{R}$ e $\varphi
    \in Aut(\mathcal{R})$, allora $\varphi(\langle r^k\rangle) = \langle r^k\rangle$
    in quanto $\varphi$ preserva l'ordine del sottogruppo e $\langle r^k\rangle$
    è l'unico sottogruppo di $\mathcal{R}$ di tale ordine, pertanto $\langle r^k\rangle$
    è caratteristico in $\mathcal{R}$. Allora per il \hyperref[lemma1]{Lemma 2.14}
    abbiamo che $\langle r^k\rangle\trianglelefteqslant D_n$.
\end{proof}



\end{document}