\documentclass[11pt]{scrartcl}
\usepackage[italian]{babel}
\usepackage[sexy]{evan}

\begin{document}
\title{Complementi di Algebra 1}
\subtitle{\large\normalfont\rmfamily\scshape APPUNTI DEL CORSO DI ALGEBRA 1 TENUTO\\ DALLA PROF. DEL CORSO E DAL PROF. LOMBARDO}
\author{Leonardo Migliorini \\ \textnormal{\href{l.migliorini@studenti.unipi.it}{l.migliorini@studenti.unipi.it}}}
\date{Anno Accademico 2022-23}
\maketitle
\newpage

\tableofcontents
\eject
\newpage

\section{Insiemi di generatori}

\begin{definition}
    Dati un gruppo $G$ e $x_1, \ldots, x_n$ elementi di $G$, chiamiamo \vocab{sottogruppo 
    generato} da $x_1, \ldots, x_n$ il più piccolo sottogruppo $\langle x_1, \ldots x_n
    \rangle$ di $G$ contenente $x_1, \ldots, x_n$, cioè \[\langle x_1, \ldots, x_n\rangle =
    \bigcap_{\substack{H\leqslant G\\ \{x_1, \ldots, x_n\} \subseteq H}} H\] 
\end{definition}

\begin{remark}
    La definizione è ben posta, infatti l'intersezione avviene su una 
    famiglia non vuota di insiemi dal momento che $G$ è un sottogruppo di 
    se stesso contenente $x_1, \ldots, x_n$. Inoltre l'intersezione non è vuota in 
    quanto contiene almeno l'identità e gli elementi $x_1, \ldots, x_n$.
\end{remark}

La definizione data non dà informazioni su come sono fatti gli elementi di 
$\langle x_1, \ldots, x_n\rangle$, cerchiamo quindi di caratterizzare in modo
diverso tale sottogruppo. Poiché chiuso per l'operazione indotta da $G$, $\langle x_1, \ldots, x_n\rangle$
deve contenere tutti i prodotti finiti, in qualsiasi ordine, delle potenze di
$x_1, \ldots, x_n$, cioè deve contenere l'insieme 
\[\{g_1^{\pm 1}, \ldots, g_r^{\pm 1}\mid r \in \NN, g_i \in \{x_1, \ldots, x_n\}
~\forall i \in \{1, \ldots, r\}\}\]

\begin{proposition}
\label{prop1.0}
Dati un gruppo $G$ e $x_1, \ldots, x_n$ elementi di $G$, allora \[
    \langle x_1, \ldots, x_n\rangle = \{g_1^{\pm 1}, \ldots, g_r^{\pm 1}\mid r 
    \in \NN, g_i \in \{x_1, \ldots, x_n\}~\forall i \in \{1, \ldots, r\}\}
    \]
\end{proposition}

\begin{proof}
Poniamo $S = \{g_1^{\pm 1}, \ldots, g_r^{\pm 1}\mid r \in \NN, g_i \in \{x_1, \ldots, x_n\}
~\forall i \in \{1, \ldots, r\}\}$, mostriamo che $S$ è un sottogruppo di $G$. 
Effettivamente $e \in S$ in quanto è prodotto nessuna potenza di $x_1, \ldots, x_n$, 
il prodotto di due elementi di $S$ è ancora un elemento di $S$ in quanto
prodotto finito di potenze di $x_1, \ldots, x_n$ e l'inverso di un elemento
$g_1^{\pm 1}\ldots g_r^{\pm 1} \in\nolinebreak S$ è $(g_1^{\pm 1}\ldots 
g_r^{\pm 1})^{-1} = g_r^{\mp 1}\ldots g_1^{\mp 1}$, che è un elemento di $S$.
Abbiamo quindi che $S$ è un sottogruppo di $G$ contenente $x_1, \ldots, x_n$,
pertanto $\langle x_1, \ldots, x_n\rangle\subseteq S$ per minimalità di $\langle x_1,
\ldots, x_n\rangle$. D'altra parte, per quanto osservato sopra abbiamo che
tutti gli elementi della forma $g_1^{\pm 1}\ldots g_r^{\pm 1}$ con $r \in \NN$, 
$g_i \in \{x_1, \ldots, x_n\}$ per ogni $i \in \{1, \ldots, r\}$ devono essere
contenuti in $\langle x_1, \ldots, x_n\rangle$, pertanto i due sottogruppi
coincidono.
\end{proof}

\begin{remark}
    Se $G$ è un gruppo ciclico abbiamo che esiste $x \in G$ tale che 
    $\langle x\rangle = G$, cioè tutti gli elementi di $G$ sono potenze di $x$.
\end{remark}

Diciamo che $x_1, \ldots, x_n \in G$ sono \vocab{generatori} per $G$, o che 
l'insieme $\{x_1, \ldots, x_n\}$ \vocab{genera} $G$ se $\langle x_1, \ldots, x_n\rangle = G$.

\newpage

\section{Automorfismi di $(\Zp)^n$}
\label{aut sp vet}

Dato $p$ un primo, vogliamo determinare quanti sono gli automorfismi di 
$(\Zp)^n$, per fare ciò è conveniente definire una struttura di spazio vettoriale,
quindi un prodotto per scalari\[
    \cdot:\Zp\times (\Zp)^n \longrightarrow (\Zp)^n : 
    (\overline{\lambda}, v)\longmapsto \overline{\lambda}v
\]
con $\overline{\lambda}v = \underset{\tilde{\lambda}\text{ volte}}{\underbrace{v + \ldots + v}}$
e $\tilde{\lambda}$ un qualsiasi rappresentante di $\overline{\lambda}$.
Tale prodotto è ben definito, infatti se $\lambda, \lambda' \in \ZZ$ sono tali che
$\overline{\lambda} = \overline{\lambda'}$, cioè esiste $k \in \ZZ$ tale che
$\lambda = \lambda' + kp$, allora \[
    \overline{\lambda'} v = \underset{\lambda'\text{ volte}}{\underbrace{v + \ldots + v}} = 
    \underset{\lambda + kp\text{ volte}}{\underbrace{v + \ldots + v}} = 
    \underset{\lambda\text{ volte}}{\underbrace{v + \ldots + v}}
\]in quanto $\underset{kp\text{ volte}}{\underbrace{v + \ldots + v}} = 0$. 
Si verifica che $((\Zp)^n, +, \cdot)$ è effettivamente uno spazio vettoriale
sul campo $\FF_p = \Zp$ (dove $\cdot$ è il prodotto per scalari appena definito).
Per come abbiamo definito il prodotto per scalari, abbiamo che per ogni
$\varphi \in \Aut((\Zp)^n)$ vale $\varphi(\lambda v) = \lambda\varphi(v)$ 
per ogni $\lambda \in \FF_p$, pertanto
\[
    \Aut((\Zp)^n) = GL((\FF_p)^n) = \{\varphi: (\FF_p)^n\longrightarrow(\FF_p)^n
    \mid\varphi\text{ isomorfismo di spazi vettoriali}\}.
\]

Poiché $GL((\FF_p)^n) \cong GL_n(\FF_p) = \{M \in M_{n\times n}(\FF_p)\mid \det M \neq 0\}$
possiamo rappresentare ogni automorfismo di $(\Zp)^n$ con una matrice invertibile
di taglia $n\times n$ a coefficienti in $\FF_p$.

\begin{proposition}
Dato $p$ un primo, allora \[
    |\Aut((\Zp)^n)| = \prod_{i = 0}^{n - 1} (p^n - p^i)
    \]
\end{proposition}

\begin{proof}
    Osserviamo che un elemento di $\Aut((\Zp)^n)$ deve necessariamente mandare 
    una base di $(\Zp)^n$ in un'altra base, e si dermina univocamente in questo 
    modo. Sia $\{v_1, \ldots, v_n\}$ una base di $(\Zp)^n$ e $\varphi \in 
    \Aut((\Zp)^n)$, consideriamo $\varphi(v_1)$: $\varphi(1)$ può assumere
    qualsiasi valore non nullo, pertanto abbiamo $(p^n - 1)$ possibilità per 
    l'immagine del primo vettore. Per quanto riguarda $v_2$, $\varphi(v_2)$
    può assumere qualsiasi valore non nullo che non sia multiplo di $\varphi(v_1)$,
    che sono $p^n - p$, analogamente $\varphi(v_3)$ può assumere qualsiasi
    valore non nullo che non sia combinazione lineare di $v_1$ e $v_2$, che sono
    $p^n - p^2$, e così via. Reiteriamo questo ragionamento fino a $\varphi(v_n)$,
    che può essere scelto in $p^n - p^{n - 1}$ modi, da cui \[
        |\Aut((\Zp)^n)| = \prod_{i = 0}^{n - 1}(p^n - p^i)
    \]
\end{proof}

\newpage

\section{Gruppo diedrale}

\subsection{Elementi del gruppo}

\begin{definition}
    Dato $n \geqslant 2$ un naturale, consideriamo un poligono regolare di $n$ vertici,
    definiamo il \vocab{gruppo diedrale} su $n$ vertici $D_n$ come l'insieme 
    delle isometrie del piano che mandano i vertici in se stessi, cioè che 
    fissano il poligono (per $n = 2$ consideriamo le isometrie che mandano un 
    segmento in se stesso).
\end{definition}

\begin{remark}
    $D_n$ è un gruppo, in quanto l'applicazione identità che 
    fissa tutti i vertici è un'isometria dal poligono in se stesso, la 
    composizione di isometrie è un'isometria e un'isometria ammette sempre 
    un'inversa, che è anch'essa un'isometria.
\end{remark}

\begin{remark}
    Una rotazione di angolo $\displaystyle\frac{2\pi}{n}$ è un elemento di $D_n$,
    così come una simmetria rispetto a un asse.
\end{remark}

Proseguendo con questa intuizione geometrica, indicheremo con $r$ una rotazione
di angolo $\displaystyle \frac{2\pi}{n}$ e con $s$ una simmetria rispetto a
un qualsiasi asse. Notiamo che $\ord(r) = n$ e $\ord(s) = 2$ (per convenzione, 
indichiamo con un angolo positivo una rotazione in senso antiorario e con un 
angolo negativo una rotazione in senso orario).

\begin{definition}
    Data $r \in D_n$ una rotazione di ordine $n$, indichiamo con $\mathcal{R}$ il
    \vocab{sottogruppo delle rotazioni} $\langle r\rangle$.
\end{definition}

\begin{remark}
    Il sottogruppo $\mathcal{R}$ contiene tutte le rotazioni di $D_n$, infatti
    se $r'$ è una rotazione di angolo $\displaystyle\frac{2k\pi}{n}$, $k \in \ZZ$,
    allora $r^k = r'$ in quanto anche $r^k$ è una rotazione di angolo 
    $\displaystyle\frac{2k\pi}{n}$.
\end{remark}

Per determinare come sono fatti gli elementi di $D_n$, studiamo il sottogruppo
$\langle r, s\rangle$. Sicuramente $\langle r, s\rangle$ contiene il sottogruppo $\mathcal{R}$
e tutti gli elementi della forma $sr^k$, $sr^ks$, $sr^ksr^h$ e così via, vogliamo
mostrare che in effetti $D_n$ è generato da $r$ e $s$.

\begin{remark}
    Gli elementi della forma $r^k$ e $sr^h$ sono distinti per ogni $h, k \in \ZZ$. 
    Infatti sappiamo dall'algebra lineare che il determinante di una simmetria
    è $-1$ e che il determinante di una rotazione è $1$, per la moltiplicatività
    del determinante quindi $\det (r^k) = (\det r)^k = 1$ e
    $\det (sr^h) = (\det s)(\det r)^h = -1$, da cui $r^k \neq sr^h$.
\end{remark}

\begin{lemma}
    Per ogni rotazione $r \in D_n$ e per ogni simmetria $s \in D_n$ vale
    \[srs^{-1} = r^{-1}\]
\end{lemma}

\begin{proof}
    \[
        srs^{-1} = r^{-1} \iff sr = r^{-1}s = (s^{-1}r)^{-1}
    \] si conclude
    osservando che $s^2 = 1$, pertanto $s^{-1} = s$ e 
    \[
        (s^{-1}r)^{-1} = (sr)^{-1} = r^{-1}s^{-1} = r^{-1}s
    \]
\end{proof}

\begin{proposition}
    Se $n \geqslant 3$ allora $|D_n| = 2n$.
\end{proposition}

\begin{proof}
    Indicando con $1, \ldots, n$ gli $n$ vertici di un poligono regolare di $n$ lati, notiamo
    che un elemento $g \in D_n$ è univocamente determinato da $g(1), \ldots, g(n)$.
    In particolare, fissato $g(1)$, per il quale abbiamo $n$ possibili scelte,
    abbiamo al massimo due valori per $g(2)$, cioè $g(2) \in \{g(1) + 1, g(1) - 1\}$
    (a meno di sommare $n$ se uno dei due elementi è negativo). Poiché $g(1)$
    e $g(2)$ individuano due vettori nel piano non allineati, cioè
    linearmente indipendenti, ne costituiscono una base: fissati i valori di 
    $g(1)$ e $g(2)$ abbiamo quindi determinato ogni
    elemento di $D_n$ in modo unico e, poiché possiamo farlo in al più $2n$ modi, 
    $|D_n| \leqslant 2n$. Ricordiamo adesso che $D_n$ contiene gli elementi
    della forma $r^k$, $sr^h$ al variare di $h, k \in \ZZ$, mostriamo che questi sono 
    infatti $2n$. Gli elementi $r^k$ appartengono al gruppo ciclico $\mathcal{R}$
    di ordine $n$, pertanto sono $n$ elementi distinti, inoltre 
    \[
        sr^i = sr^j \iff r^i = r^j\iff i \equiv j \mod n
    \] pertanto anche questi sono $n$
    elementi distinti. Allora $|D_n| = 2n$.
\end{proof}

\begin{remark}
    Abbiamo mostrato che effettivamente $D_n = \langle r, s\rangle$, quindi i
    suoi elementi sono tutti della forma $r^k$, $sr^h$ al variare di $h, k \in \ZZ$. 
\end{remark}

\begin{remark}
    Il risultato è valido anche per $D_2$, ma con motivazioni diverse. 
    Se consideriamo un segmento nel piano $\RR^2$ giacente sulla retta $y = 0$, 
    le isometrie che possiamo applicare sono l'identità, la rotazione di 
    angolo $\pi$, la simmetria lungo la retta $y = 0$ e la simmetria lungo l'asse
    passante per il suo punto medio. $D_2$ contiene quindi quattro elementi,
    l'identità e tre elementi di ordine 2, pertanto è isomorfo a $\Z2\times\Z2$.
\end{remark}


\subsection{Sottogruppi}

Consideriamo un sottogruppo $H\leqslant D_n$, distinguiamo due possibilità: 
$H \subseteq \mathcal{R}$ oppure $H \nsubseteq \mathcal{R}$. Nel primo caso
abbiamo che $|H|\mid n$, ed è l'unico sottogruppo di $\mathcal{R}$ con questa 
proprietà in quanto $\mathcal{R}$ è ciclico, in particolare $H$ è ciclico 
della forma $ \displaystyle\left<\frac n d\right>$, con $d \mid n$. \newline
Studiamo quindi il caso $H \nsubseteq \mathcal{R}$: notiamo che 
$\mathcal{R}\trianglelefteqslant D_n$ in quanto $[D_n : \mathcal{R}] = 2$,
pertanto $\faktor{D_n}{\mathcal{R}}$ è un gruppo con l'operazione indotta da $D_n$
e risulta essere isomorfo a $\Z2$. \newline
Consideriamo la proiezione al quoziente \[
    \pi_{\mathcal{R}}: D_n \longrightarrow \faktor{D_n}{\mathcal{R}} : g \mapsto [g]
\]poiché $H \nsubseteq \mathcal{R}$ abbiamo che esiste $h \in H$ tale che 
$h \notin \mathcal{R}$, pertanto $\pi_{\mathcal{R}}(h) \notin [\mathcal{R}]$ e
in particolare $\pi_{\mathcal{R}}(H) \nsubseteq [\mathcal{R}]$. Dato che i 
sottogruppi di $\faktor{D_n}{\mathcal{R}}$ sono solo $\{[\mathcal{R}]\}$ e
$\faktor{D_n}{\mathcal{R}}$ abbiamo $\pi_{\mathcal{R}}(H) = 
\faktor{D_n}{\mathcal{R}}$. Osserviamo inoltre che $\ker \pi_{\mid H} = 
\ker \pi \cap H = \mathcal{R}\cap H$, per il Primo Teorema di Omomorfismo
allora $\faktor{H}{H\cap \mathcal{R}} \cong \Z2$, quindi 
$|H\cap\mathcal{R}| = \displaystyle\frac 1 2 |H|$. Dato che $R\cap H \subseteq
\mathcal{R}$, esiste $k \in \ZZ$ tale che $H\cap\mathcal{R} = \langle r^k\rangle$
in particolare $\langle r^k\rangle$ e $\langle sr^h\rangle$, $h \in \ZZ$, sono
contenuti in $H$. 

\begin{proposition}
    Dati $H\leqslant D_n$ un sottogruppo tale che $H\nsubseteq \mathcal{R}$, se
    $r$ è un generatore di $\mathcal{R}$ tale che $H\cap\mathcal{R} = \langle r^k\rangle$ 
    e $s$ è una simmetria allora \[
    H = \langle r^k\rangle\cdot\langle sr^h\rangle = \{xy \mid x \in \langle r^k\rangle,
    y \in \langle sr^h\rangle\}, h, k \in \ZZ
    \]
\end{proposition}

\begin{proof}
    Per quanto visto sopra abbiamo che $|\langle r^k\rangle| = 
    \displaystyle\frac 1 2|H|$, inoltre osserviamo che $\ord(sr^h) = 2$ in quanto
    \[
        (sr^h)^2 = sr^hsr^h = (srs)^hr^h = (srs^{-1})^hr^h = r^{-h}r^h = e
    \]
    pertanto $\langle sr^h\rangle \cong \Z2$. Da questo ricaviamo $\langle sr^h\rangle
    \subseteq N_{D_n}(\langle r^k\rangle)$, infatti per ogni $m \in \ZZ$ abbiamo\[
        (sr^h)r^{mk}(sr^h)^{-1} = sr^{h + mk} sr^h = r^{-h-mk}r^h = r^{-mk}
        \in \langle r^k \rangle
    \]cioè $\langle sr^h\rangle \subseteq N_{D_n}(\langle r^k\rangle)$ e quindi
    $\langle r^k\rangle\cdot\langle sr^h\rangle$ è un sottogruppo di $D_n$\footnote
    {Dati $K, N$ sottogruppi
    di un gruppo $G$, se vale almeno una delle inclusioni $K \subseteq N_G(N)$,
    $N \subseteq N_G(K)$ allora $HK = KH$, quindi $HK$ è un sottogruppo di $G$.}.
    Poiché $\langle r^k\rangle$ e $\langle sr^h\rangle$ sono contenuti in $H$
    abbiamo che $\langle r^k\rangle\cdot\langle sr^h\rangle \subseteq H$, inoltre
    \[|\langle r^k\rangle\cdot\langle sr^h\rangle| = \displaystyle\frac 1 2 |H|\cdot 2 = |H|\]
    in quanto $\langle r^k\rangle\cap\langle sr^h\rangle = \{e\}$\footnote{
        Se $H, K$ sono sottogruppi finiti di un gruppo $G$ e $HK\leqslant G$ allora
        vale $|HK| = \displaystyle\frac{|H|\cdot|K|}{|H\cap K|}$.
    }, pertanto 
    i due sottogruppi coincidono.
\end{proof}

\begin{remark}
    Per $k \mid n$ e $0\leqslant h < k$, i sottogruppi $H_{k, h} = \langle r^k, sr^h\rangle$
    e $H = \langle r^k\rangle\cdot\langle sr^h\rangle$ coincidono. Infatti 
    $H_{k, h}\subseteq H$ in quanto $r^k, sr^h$ sono elementi di $H$, 
    d'altra parte $H \subseteq H_{k, h}$ in quanto $H_{h, k}$ contiene tutti i 
    prodotti finiti delle potenze di $r^k$ e $sr^h$, in particolare gli elementi di $H$.
\end{remark}

\begin{remark}
    Per $k \mid n$ e $0\leqslant h < k$, $\langle r^k, sr^h\rangle = 
    \langle r^k, sr^{h + k}\rangle$. Infatti $\langle r^k, sr^h\rangle \subseteq
    \langle r^k, sr^{h + k}\rangle$ in quanto $sr^h = (sr^{h + k})r^{-k}$ è
    un elemento del secondo gruppo, simmetricamente $\langle r^k, sr^{h + k}\rangle
    \subseteq \langle r^k, sr^h\rangle$ in quanto $sr^{h + k} = (sr^h)r^k$ è un
    elemento del primo gruppo.
\end{remark}

\begin{theorem}
    [Classificazione dei sottogruppi di $D_n$]
I sottogruppi di $D_n$ sono della forma \begin{enumerate}[(1)]
    \item $\langle r^k\rangle$ con $k\mid n$;
    \item $\langle r^k, sr^h\rangle$ con $k \mid n$, $0\leqslant h < k$, 
\end{enumerate}
con $r \in \mathcal{R}$ e $s$ una simmetria. Inoltre tali sottogruppi sono
tutti distinti.
\end{theorem}

\begin{proof}
    Abbiamo già visto che i sottogruppi di $D_n$ sono di questo tipo, 
    mostriamo quindi che sono tutti distinti. A meno di cambiare $k$, possiamo
    supporre $\mathcal{R} = \langle r\rangle$, cioè $\ord(r) = n$. 
    Consideriamo $H, K\leqslant D_n$ due sottogruppi, abbiamo tre casi:
    \begin{itemize}
        \item se $H = \langle r^k\rangle$ e $K = \langle r^m\rangle$, $m \in \ZZ$,
        allora $H = K\iff k = m$ in quanto entrambi sottogruppi di $\mathcal{R}$, 
        pertanto esiste un unico sottogruppo della forma $\langle r^k\rangle$
        per $k \mid n$;
        \item se $H = \langle r^k\rangle$ e $K = \langle r^m, sr^h\rangle$, $m \mid n$,
        allora $H \neq K$ in quanto $H$ è ciclico e $K$ no;
        \item se $H = \langle r^k, sr^h\rangle$ e $K = \langle r^m, sr^l\rangle$, 
        con $m \mid n$ e $0\leqslant l < m$, considerando le intersezioni
        $H \cap \mathcal{R} = \langle r^k\rangle$ e $K \cap \mathcal{R} = \langle r^m\rangle$ 
        abbiamo \[
        H \cap \mathcal{R} = K\cap\mathcal{R} \iff \langle r^k\rangle = \langle r^m\rangle
        \iff k = m
        \] Inoltre, se $sr^h \in \langle r^m, sr^l\rangle = \langle r^m\rangle
        \cdot \langle sr^l\rangle$, allora esiste $t \in \ZZ$ tale che \[
        sr^h = (r^m)^t sr^l \iff sr^h = s^2r^{mt}sr^l \iff r^h = r^{-mt + l}
        \iff h \equiv l - mt \mod n
        \]da cui ricaviamo $h \equiv l \mod m$ in quanto $m \mid n$. Ma allora 
        $h =l$ dato che $0 \leqslant h < k$ e $0\leqslant l < m$.
    \end{itemize}
\end{proof}

\begin{lemma}
    \label{lemma1.0}
    Dati un gruppo $G$ e $A, B$ due sottogruppi tali che $A \leqslant B \leqslant G$,
    se $B\trianglelefteqslant G$ e $A$ è caratteristico in $B$ allora 
    $A \trianglelefteqslant G$.
\end{lemma}

\begin{proof}
    Fissato $g \in G$, consideriamo l'omomorfismo di coniugio 
    \[
        \varphi_g : G\longrightarrow G : x\longmapsto gxg^{-1}
    \] poiché 
    $B\trianglelefteqslant G$ è ben definita la restrizione $\varphi_{g\mid B} \in \Aut(B)$. 
    Dal momento che $A$ è
    un sottogruppo caratteristico di $B$ abbiamo che $\varphi_{g\mid B}(A) =
    \varphi_g(A) = A$,
    pertanto $A \trianglelefteqslant G$.
\end{proof}


\begin{corollary}
    Ogni sottogruppo di $\mathcal{R}$ è normale in $D_n$.
\end{corollary}

\begin{proof}
    Siano $\langle r^k\rangle$ un sottogruppo di $\mathcal{R}$ e $\varphi
    \in \Aut(\mathcal{R})$, allora $\varphi(\langle r^k\rangle) = \langle r^k\rangle$
    in quanto $\varphi$ preserva l'ordine del sottogruppo e $\langle r^k\rangle$
    è l'unico sottogruppo di $\mathcal{R}$ di tale ordine ($\mathcal{R}$ è ciclico),
    pertanto $\langle r^k\rangle$
    è caratteristico in $\mathcal{R}$. Poiché $\mathcal{R}$ è un sottogruppo
    normale di $D_n$, per il \hyperref[lemma1.0]{Lemma 2.15}
    abbiamo $\langle r^k\rangle\trianglelefteqslant D_n$.
\end{proof}

\begin{remark}
    $\mathcal{R}$ è caratteristico in $D_n$ per $n \geqslant 3$.
    Infatti se $\mathcal{R} = \langle r \rangle$, necessariamente 
    $\ord(r) = \ord(\varphi(r))$, da cui $|\langle\varphi(r)\rangle| = n$.
    Se fosse $\varphi(r) \notin \mathcal{R}$ avremmo $\ord(\varphi(r)) = 2$, 
    quindi $|\varphi(r)| = n = 2$, che è assurdo in quanto $|D_n| = 2n \geqslant 6$.
    Questo non è vero per $D_2$, che contiene una rotazione e due
    simmetrie. Poiché $\Aut(D_2) \cong S_3$, esiste un automorfismo che manda 
    la rotazione in una riflessione, quindi che non fissa $\mathcal{R}$.
\end{remark}

\begin{corollary}
    Per $k\mid n$ e $0\leqslant h < k$, il sottogruppo $H_{k, h} = \langle r^k, sr^h\rangle$
    è normale in $D_n$ se e solo se $r, s \in N_{D_n}(H_{k, h})$.
\end{corollary}

\begin{proof}~
    \begin{itemize}
        \item Se $H_{k, h}\trianglelefteqslant D_n$ allora $N_{D_n}(H_{k, h}) = D_n$, 
        in particolare $r, s \in N_{D_n}(H_{k, h})$;
        \item se $r, s \in N_{D_n}(H_{k, h})$, poiché il normalizzatore è un
        sottogruppo di $D_n$ abbiamo che $D_n = \langle r, s\rangle \subseteq
        N_{D_n}(H_{k, h})$, pertanto $H_{k, h} \trianglelefteqslant D_n$.
    \end{itemize}
\end{proof}

Vediamo effettivamente quali sono i sottogruppi normali della forma 
$\langle r^k, sr^h\rangle$. Consideriamo gli automorfismi di coniugio \[
    \varphi_s: D_n \longrightarrow D_n :x \longmapsto sxs^{-1}\qquad
    \varphi_r: D_n \longrightarrow D_n :x \longmapsto rxr^{-1}
\]e sia $x_1^{\pm 1}\ldots x_m^{\pm 1} \in H_{k, h} = \langle r^k, sr^h\rangle$, allora
\[
    \varphi_s(x_1^{\pm 1}\ldots x_m^{\pm 1}) = \varphi_s(x_1)^{\pm 1}\ldots \varphi_s(x_m)^{\pm 1}
    \in \langle srs, r^hs^{-1}\rangle = \langle sr^ks, r^hs^{-1}\rangle = \langle
    r^k, sr^{-h}\rangle
\]
\[
    \varphi_r(x_1^{\pm 1}\ldots x_m^{\pm 1}) = \varphi_r(x_1)^{\pm 1}\ldots \varphi_r(x_m)^{\pm 1}
    \in \langle r^k, rsr^{h - 1}\rangle = \langle r^k, sr^{h - 2}\rangle
\]

Pertanto $H_{k, h}\trianglelefteqslant D_n$ se e solo se $\langle r^k, sr^{h - 2}\rangle
= \langle r^k, sr^{-h}\rangle = \langle r^k, sr^h\rangle$, se e solo se 
$h \equiv h - 2 \mod k$, cioè $k \in \{1, 2\}$.\begin{itemize}
    \item Se $k = 1$ allora $H_{k, h} = \langle r, s\rangle = D_n$;
    \item se $k = 2$ (e $n$ pari) allora $H_{k, h} = \langle r^2, sr\rangle$ oppure 
    $H_{k, h} = \langle r^2, s\rangle$.
\end{itemize}

\begin{remark}
    Il secondo caso si presenta solo se $n$ è pari, questo corrisponde al fatto 
    che in un poligono con un numero pari di lati gli assi di simmetria sono 
    per metà passanti per i lati e metà passanti per i vertici opposti. In un poligono con un numero dispari
    di lati gli assi di simmetria sono tutti passanti per i lati.
\end{remark}


\subsection{Classi di coniugio}

Abbiamo visto che possiamo scrivere ogni elemento di $D_n$ nella forma
 $s^hr^k$, dove $s$ è una simmetria e $r$ è una rotazione che genera 
 $\mathcal{R}$, con $h \in \{0, 1\}$ e $k \in \{0, \ldots, n - 1\}$
in quanto $\ord(s)= 2$ e $\ord(r) = n$. Inoltre tutti gli elementi della
forma $sr^h$ hanno ordine $2$.

Consideriamo la classe di coniugio di $r$, $C_r = \{grg^{-1}\mid g \in D_n\}$,
fissato $g \in D_n$ abbiamo due possibili valori per $grg^{-1}$:
\begin{itemize}
    \item se $g \in \mathcal{R}$ allora $g$ è una potenza di $r$, pertanto i
    due elementi commutano e si ha $grg^{-1} = r$;
    \item se $g\notin\mathcal{R}$ allora $g = sr^h$ con $h \in \ZZ$, quindi
    \[
        (sr^h)r(sr^h)^{-1} = (sr^h)r(sr^h) = sr^{h + 1}sr^h = s^2r^{-1-h}r^h = r^{-1}
    \]
\end{itemize}
cioè $C_r = \{r, r^{-1}\}$. In modo analogo si mostra che $C_{r^k} = \{r^k, r^{-k}\}$
per ogni $k \in \ZZ$.

\begin{remark}
    Se $n$ è pari, scriviamo $n = 2m$ e consideriamo la classe di coniugio
    di $r^m$. Poiché $r^m \neq e$ e $r^{2m} = (r^m)^2 = e$ abbiamo che
    $\ord(r^m) = 2$, cioè $(r^m)^{-1} = r^m$. Allora $C_{r^m} = \{r^m\}$,
    pertanto abbiamo trovato un elemento del centro di $D_n$ (infatti se $G$
    è un gruppo e $x \in G$, allora $x \in Z(G)$ se e solo se $C_x = \{x\}$).
\end{remark}

Consideriamo adesso la classe di coniugio di $sr^h$, $C_{sr^h} = 
\{g(sr^h)g^{-1}\mid g\in D_n\}$, fissato $g \in D_n$ abbiamo due possibili
valori per $g(sr^h)g^{-1}$:
\begin{itemize}
    \item se $g \in \mathcal{R}$ allora $g = r^k$ con $k \in \ZZ$, pertanto
    \[
        r^k(sr^h)r^{-k} = sr^{-k}r^h r^{-k} = sr^{h - 2k}
    \]
    \item se $g \notin \mathcal{R}$ allora $g = sr^k$ con $k \in \ZZ$, pertanto
    \[
        (sr^k)(sr^h)(sr^k)^{-1} = (sr^k)(sr^h)(sr^k) = sr^{2k - h}
    \]
\end{itemize}
cioè $C_{sr^k} = \{sr^{h - 2k}, sr^{2k - h}\mid k \in \ZZ\}$. 

\begin{remark}
    La classe di coniugio di $sr^h$ contiene tutte le simmetrie in cui 
    l'esponente di $r$ ha la stessa parità di $h$. Se $n$ è 
    dispari tutte le simmetrie appartengono alla stessa classe,
    mentre se $n$ è pari abbiamo due classi distinte: quella 
    delle simmetrie rispetto agli assi passanti per i vertici opposti e quella 
    delle simmetrie rispetto agli assi passanti per i lati.
\end{remark}


\subsection{Legge di gruppo e omomorfismi}

Se $g$ è un elemento di $D_n$ possiamo scrivere $g$ in modo unico come $s^ar^b$ 
con $a \in \{0, 1\}$ e $b \in \{0, \ldots, n - 1\}$, utilizziamo questa 
proprietà per esplicitare la legge di gruppo di $D_n$. \newline
Fissati $g_1, g_2 \in D_n$, scriviamo $g_1 = s^{a_1}r^{b_1}$ e $g_2 = s^{a_2}r^{b_2}$
con $a_1, a_2 \in \{0, 1\}$ e $b \in \{0, \ldots, n - 1\}$, 
\[
    g_1g_2 = (s^{a_1}r^{b_1})(s^{a_2}r^{b_2}) = s^{a_1}s^{a_2}(s^{a_2}r^{b_1}s^{-a_2})r^{b_2} = 
    s^{a_1}s^{a_2}\varphi_{s^{a_2}}(r^{b_1})r^{b_2}
\]
dove $\varphi_{s^{a_2}}$ è l'automorfismo di coniugio per $s^{a_2}$
(ricordiamo che $s^{a_2} = s^{-a_2}$). Poiché
$\varphi_{s^{a_2}}$ è un omomorfismo e $\varphi_x\circ\varphi_y = \varphi_{xy}$ per ogni $x, y \in G$,
abbiamo $(\varphi_{s^{a_2}}(r^{b_1})) = (\varphi_s^{a_2}(r))^{b_1}$, quindi
\[
    g_1g_2 = s^{a_1}s^{a_2}(\varphi_s^{a_2}(r))^{b_1}r^{b_2} = 
    s^{a_1 + a_2}r^{(-1)^{a_2}b_1 + b_2}
\]
Per l'unicità della scrittura che stiamo usando (scegliendo 
$a \in \{0, 1\}$ e $b \in \{0, \ldots, n - 1\}$), possiamo identificare 
ogni elemento $g = s^ar^b \in D_n$ con la coppia $(a, b)$, la legge di gruppo
è quindi tale che \[
    (a_1, b_1)(a_2, b_2) = (a_1 + a_2, (-1)^{a_2}b_1 + b_2)
\]
\newline 
Usiamo il risultato appena ottenuto per descrivere gli omomorfismi da $D_n$ in 
un qualsiasi gruppo $G$. Poiché ogni elemento $g \in D_n$ si scrive come
$s^ar^b$, con $a, b \in \ZZ$, un omomorfismo $\varphi \in \Hom(D_, G)$ è univocamente
determinato da $\varphi(r)$ e $\varphi(s)$: infatti \[
    \varphi(g) = \varphi(s^ar^b) = \varphi(s)^a\varphi(r)^b
\]Poniamo $x = \varphi(s)$, $y = \varphi(r)$, necessariamente $\ord(x) \mid n$
e $\ord(y)\mid 2$, cioè $x^n = e_G$ e $y^n = e_G$, inoltre \[
    xyx^{-1} = \varphi(s)\varphi(r)\varphi(s)^{-1} = \varphi(srs^{-1}) = 
    \varphi(r^{-1}) = \varphi(r)^{-1} = y^{-1}
\]Mostriamo che effettivamente queste condizioni sono anche sufficienti:

\begin{proposition}
    Dati un gruppo $G$ e un'applicazione
    \[
        \varphi:D_n\longrightarrow G :s^ar^b \longmapsto x^ay^b
    \]dove $x = \varphi(s)$ e $y = \varphi(r)$, allora $\varphi$ è un omomorfismo
    se e solo se $x^2 = e_G$, $y^n = e_G$ e $xyx^{-1} = y^{-1}$.
\end{proposition}

\begin{proof}
    Mostriamo che tali condizioni sono sufficienti affinché $\varphi$ sia un 
    omomorfismo. Poiché $x^m = x^{-m}$ per ogni $m \in \ZZ$, fissati 
    $a_1, a_2, b_1, b_2 \in \ZZ$ abbiamo 
    \begin{multline*}
        (x^{a_1}y^{b_1})(x^{a_2}y^{b_2}) = x^{a_1}x^{a_2}(x^{a_2}y^{b_1}x^{-a_2})y^{b_2} = 
        x^{a_1 + a_2}\varphi_{x^{a_2}}(y^{b_1})y^{b_2} = \\
        = x^{a_1 + a_2} (\varphi_x^{a_2}(y))^{b_1}y^{b_2} =
        x^{a_1 + a_2}y^{(-1)^{a_2}b_1}y^{b_2} = x^{a_1 + a_2}y^{(-1)^{a_2}b_1 + b_2}
    \end{multline*}dove $\varphi_g$ è l'automorfismo di coniugio per $g \in G$.
    Allora abbiamo che $\varphi$ è un omomorfismo, infatti per ogni $h_1, h_2, k_1, k_2 \in \ZZ$
    \begin{multline*}
        \varphi((s^{h_1}r^{k_1})(s^{h_2}r^{k_2})) = \varphi(s^{h_1 + h_2}r^{(-1)^{h_2}k_1 + k_2}) =\\
        = x^{h_1 + h_2}y^{(-1)^{h_2}k_1 + k_2} = (x^{h_1}y^{k_1})(x^{h_2}y^{k_2}) = 
        \varphi(s^{h_1}r^{h_2})\varphi(s^{h_2}r^{h_2})
    \end{multline*}
\end{proof}


\subsection{Automorfismi}

Studiamo separatamente gli automorfismi di $D_n$ per $n \geqslant 3$ e di $D_2$.\newline
Per $n\geqslant 3$ consideriamo $\varphi \in \Aut(D_n)$, poiché $D_n = \langle
r, s\rangle$ è sufficiente studiare le immagini di $r, s$ per determinare $\varphi$.
Osserviamo che necessariamente $\varphi(r) = r^k$ con $(n, k) = 1$, infatti 
$\varphi$ deve preservare l'ordine di $r$ e la sua immagine deve essere un 
generatore di $\mathcal{R}$, in quanto $\mathcal{R}$ è caratteristico in $D_n$
è isomorfo a $\Zn$. Per quanto riguarda $\varphi(s)$, se $n$ è dispari allora le simmetrie
sono gli unici elementi di ordine 2, pertanto $\varphi(s) = sr^h$ con 
$0\leqslant h < n$. Se $n$ è pari abbiamo apparentemente due possibilità:
\begin{enumerate}[(1)]
    \item $\varphi(s) = sr^h$, con $0\leqslant h < n$;
    \item $\varphi(s) = r^{\frac n 2}$, se $n$ è pari.
\end{enumerate}

D'altra parte, se fosse $\varphi(s) = r^{\frac n 2}$ allora $\varphi$ non
sarebbe né iniettiva né surgettiva, pertanto $\varphi(s) = sr^h$ con 
$0\leqslant h \leq n$. Verifichiamo che $\varphi$ è un omomorfismo, per la 
caratterizzazione che abbiamo dato sopra è sufficiente verificare che
$\varphi(s)\varphi(r)\varphi(s)^{-1} = \nolinebreak\varphi(r)^{-1}$:
\[
    \varphi(s)\varphi(r)\varphi(s)^{-1} = (sr^h)r^k(sr^h)^{-1} = sr^{h + k}r^{-h}s =
    sr^k s^{-1} = r^{-k} = \varphi(r)^{-1}
\]
Inoltre $\varphi$ è surgettiva, infatti $r^k, sr^h \in \mathrm{Im}\varphi$,
cioè 
\[
    \langle r^k, sr^h\rangle = \langle r, sr^h\rangle = \langle s, r\rangle =
    D_n\subseteq \mathrm{Im}\varphi
\]da cui $\mathrm{Im}\varphi = D_n$. Poiché $D_n$ è finito abbiamo che $\varphi$
è un automorfismo. Gli automorfismi di $D_n = \langle r, s\rangle$ quindi sono
tutti e soli gli omomorfismi da $D_n$ in $D_n$ che mandano $r$ in un generatore
di $\mathcal{R}$, che sono $\phi(n)$, e $s$ in un'altra simmetria, che sono 
$n$, pertanto $|\Aut(D_n)| = n\phi(n)$.\newline

Per $n = 2$, sappiamo che $D_2 \cong (\Z2)^2$, pertanto 
\[
    \Aut(D_2) \cong \Aut((\Z2)^2) \cong S_3
\]
Alternativamente possiamo considerare $(\Z2)^2$ come spazio vettoriale su $\FF_2$,
pertanto abbiamo 
\[
    \Aut(D_2) \cong GL_2(\FF_2)
\]Per quanto visto nella sezione \hyperref[aut sp vet]{(2)}, $GL_2(\FF_2)$ 
contiene $(4 - 1)(4 - 2) = 6$ elementi, inoltre $GL_2$ non è un gruppo 
commutativo (con l'operazione di prodotto tra matrici), pertanto $GL_2(\FF_2) \cong S_3$.
In particolare, gli elementi di $GL_2(\FF_2)$ sono:
\begin{itemize}
    \item $\begin{pmatrix}
    1 & 0\\
    0 & 1
    \end{pmatrix}$, che è l'identità del gruppo;
    \item $\begin{pmatrix}
    0 & 1\\
    1 & 0
    \end{pmatrix}, \begin{pmatrix}
        1 & 0\\
        1 & 1
    \end{pmatrix}, \begin{pmatrix}
        1 & 1\\
        0 & 1
    \end{pmatrix}$, che sono gli elementi di ordine 2 corrispondenti alle 
    trasposizioni;
    \item $\begin{pmatrix}
    1 & 1\\
    1 & 0
    \end{pmatrix}, \begin{pmatrix}
        0 & 1\\
        1 & 1
    \end{pmatrix}$ che sono gli elementi di ordine $3$ corrispondenti ai $3$-cicli.
\end{itemize}

\newpage

\section{Automorfismi di un prodotto diretto}

Consideriamo due gruppi finiti $H, K$, studiamo il gruppo degli automorfismi 
di $H\times K$. Chiaramente esiste un'inclusione di $\Aut(H)\times \Aut(K)$ in 
$\Aut(H\times K)$ data dall'omomorfismo 
\[
    \iota: \Aut(H)\times \Aut(K)\longrightarrow \Aut(H\times K) :
    (\varphi_1, \varphi_2)\longmapsto \varphi_1\times \varphi_2
\]con 
\[
    \varphi_1\times\varphi_2: H\times K \longrightarrow H\times K:
    (g_1, g_2)\longmapsto (\varphi_1(g_1), \varphi_2(g_2))
\]
Mostriamo che $\iota$ è ben definita e che è un omomorfismo iniettivo:
\begin{itemize}
    \item per ogni $(\varphi_1, \varphi_2)\in \Aut(H)\times \Aut(K)$, per ogni
     $(g_1, g_2), (h_1, h_2)\in H\times K$ abbiamo 
     \begin{multline*}
        (\varphi_1\times\varphi_2)((g_1, g_2)(h_1, h_2)) = 
        (\varphi_1(g_1h_1), \varphi(g_2h_2)) = 
        (\varphi_1(g_1)\varphi_1(h_2), \varphi_2(g_2)\varphi_2(h_2)) = \\
        =(\varphi_1(g_1), \varphi_2(g_2))(\varphi_1(h_1),\varphi_2(h_2)) = 
        ((\varphi_1\times\varphi_2)(g_1, g_2))((\varphi_1\times\varphi_2)(h_1,h_2))
     \end{multline*}
     cioè $\varphi_1\times\varphi_2$ è un omomorfismo. Inoltre 
    \[
        \ker (\varphi_1\times\varphi_2) = \{(g_1, g_2) \in H\times K\mid 
        (\varphi_1(g_1), \varphi_2(g_2)) = (e_H, e_K)\} = \{(0, 0)\}
    \]
    quindi $\varphi_1\times \varphi_2 \in \Aut(H\times K)$
    in quanto $H\times K$ è finito, pertanto $\iota$ è ben definita;
    \item per ogni $(\varphi_1, \varphi_2), (\psi_1, \psi_2) \in \Aut(H)\times \Aut(K)$,
    per ogni $(g_1, g_2) \in H\times K$ abbiamo 
    \begin{multline*}
        \iota((\varphi_1, \varphi_2)(\psi_1, \psi_2))(g_1, g_2) = 
        \iota(\varphi_1\psi_1, \varphi_2\psi_2)(g_1, g_2) = 
        (\varphi_1\psi_1\times\varphi_2\psi_2)(g_1, g_2) =\\
        = (\varphi_1(\psi_1(g_1)), \varphi_2(\psi_2(g_2))) = 
        (\varphi_1\times\varphi_2)(\psi_1(g_1), \psi_2(g_2)) = \\
        = ((\varphi_1\times\varphi_2)(\psi_1\times\psi_2))(g_1, g_2) = 
        (\iota(\varphi_1, \varphi_2)\iota(\psi_1\psi_2))(g_1, g_2)
    \end{multline*}cioè $\iota((\varphi_1, \varphi_2)(\psi_1, \psi_2)) = 
    \iota(\varphi_1, \varphi_2)\iota(\psi_1, \psi_2)$, quindi $\iota$ è un 
    omomorfismo;
    \item$\iota$ è iniettiva, infatti \begin{multline*}
        \ker \iota = \{(\varphi_1, \varphi_2) \in \Aut(H)\times \Aut(K) \mid
        \iota(\varphi_1, \varphi_2) = e_{\Aut(H\times K)}\} = \\
        = \{(\varphi_1, \varphi_2) \in \Aut(H)\times \Aut(K)\mid 
        (\varphi_1(g_1), \varphi_2(g_2)) = (e_H, e_K)~\forall 
        (g_1, g_2) \in H\times K\}
    \end{multline*} Poiché gli unici elementi $\varphi_1 \in \Aut(H)$,
    $\varphi_2 \in \Aut(K)$ tali che $\varphi_1(H) = \{e_H\}$ e $\varphi_2(K) = \{e_K\}$
    sono rispettivamente $e_{\Aut(H)}$, $e_{\Aut(K)}$ abbiamo \[
        \ker\iota = \{(e_{\Aut(H)}, e_{\Aut(K)})\} = \{e_{\Aut(H \times K)}\}
    \] 

\end{itemize}

\begin{proposition}
    Dati due gruppi finiti $H, K$, $\Aut(H)\times \Aut(K)\cong \Aut(H\times K)$
    se e solo se $H\times \{e_K\}$ e $\{e_H\}\times K$ sono sottogruppi 
    caratteristici di $H\times K$.
\end{proposition}

\begin{proof}
    Sia $\iota$ l'immersione da $\Aut(H)\times \Aut(K)$ in $\Aut(H\times K)$ 
    definita come sopra, se $\iota$ è surgettiva allora ogni elemento di 
    $\\Aut(H\times K)$ può essere scritto come $\varphi_1\times\varphi_2$ con
    $\varphi_1 \in \Aut(H)$ e $\varphi_2 \in \Aut(K)$. Allora abbiamo 
    \[
        (\varphi_1\times\varphi_2)(H\times\{e_K\}) = 
        (\varphi_1(H), \varphi_2(\{e_K\})) = H\times\{e_K\}
    \]
    \[
        (\varphi_1\times \varphi_2)(\{e_K\}\times K) = 
        (\varphi_1(\{e_H\}), \varphi_2(K)) = \{e_H\}\times K
    \]cioè $H\times\{e_K\}$ e $\{e_H\}\times K$ sono caratteristici in
    $H\times K$. Viceversa, se i due sottogruppi sono caratteristici, dato
    $\varphi \in \Aut(H\times K)$ poniamo $\varphi_1 \in \Aut(H)$ tale che 
    $\varphi(g_1, e_K) = (\varphi_1(g_1), e_K)$ e $\varphi_2 \in \Aut(K)$ 
    tale che $\varphi(e_H, g_2) = (e_H, \varphi_2(g_2))$ per ogni $g_1 \in H$,
    per ogni $g_2 \in K$ (questo possiamo farlo in quanto $H\times\{e_K\}$ 
    e $\{e_H\}\times K$ sono caratteristici). Allora abbiamo 
    \begin{multline*}
        \varphi(g_1, g_2) = \varphi((g_1, e_K)(e_H, g_2)) = 
        \varphi(g_1, e_K)\varphi(e_H, g_2) = \\
        = (\varphi_1(g_1), e_K)(e_H, \varphi_2(g_2)) = 
        (\varphi_1(g_1), \varphi_2(g_2)) = (\varphi_1\times\varphi_2)(g_1, g_2)
    \end{multline*}cioè $\iota$ è surgettiva e quindi un isomorfismo tra
    $\Aut(H)\times \Aut(K)$ e $\Aut(H\times K)$.
\end{proof}

\begin{example}
    Consideriamo il gruppo $G = \ZZ \times \Zn$, osserviamo che il sottogruppo 
    $\{0\}\times \Zn$ è caratteristico in quanto un automorfismo $\varphi$ di $G$ deve
    preservare gli ordini degli elementi, in particolare quello di un generatore,
    quindi l'immagine di un generatore è un altro generatore del sottogruppo.
    Poiché gli elementi di $G$ di ordine finito sono tutti della forma $(0, d)$
    abbiamo che $\varphi(\{0\}\times\Zn) = \{0\}\times\Zn$. Viceversa, l'immagine
    di $\varphi$ su un generatore di $\ZZ\times\{0\}$, ad esempio 
    $\varphi(1, 0)$, è della forma $(a, b)$,
    e questo implica che $\ZZ\times \{0\}$ non è caratteristico. Se $\varphi$ è
    surgettivo, necessariamente esiste $(x, y) \in \ZZ\times \Zn$ tale che
    $\varphi(x, y) = (\pm 1, 0)$, da cui, posti $\varphi(1, 0) = (a, b)$ e
    $\varphi(0, 1) = (0, d)$ con $n$ e $d$ coprimi, abbiamo
    \begin{multline*}
        \varphi(x, y) = \varphi(x(1, 0) + y(0, 1)) = x\varphi(1, 0) + y\varphi(0, 1) = \\
        = x(a, b) + y(0, d) = (xa, xb + yd) = (\pm 1, 0) \iff a = \pm 1
    \end{multline*}
    Viceversa, se $a = \pm 1$ allora $\varphi$ è surgettiva, infatti 
    per ogni $(x_0, y_0) \in \ZZ\times\Zn$, scegliendo
    $x = x_0a$ e $y \equiv d^{-1}(y_0 - x_0ab)\mod n$ abbiamo \[
        \varphi(x, y) = (x_0a^2, x_0ab + d d^{-1}(y_0 - x_0ab)) = (x_0, y_0)
    \]e questo ci permette di concludere che $\ZZ\times\{0\}$ non è un sottogruppo
    caratteristico. In questo caso abbiamo solo un'immersione del gruppo
    $\Aut(\ZZ) \times \Aut(\Zn)$ dentro a $\Aut(\ZZ\times \Zn)$, in quanto 
    gli automorfismi che mandano $(\pm 1, 0)$ in $(a, b)$ con $a = \pm 1$ e 
    $b \neq 0$ non possono essere ristretti ad automorfismi di $\ZZ\times \{0\}$.
\end{example}

È utile riuscire a determinare se i sottogruppi $H\times\{e_K\}$, $\{e_H\}\times K$
sono caratteristici in $H\times K$, da cui il seguente risultato:

\begin{proposition}
    Dati due gruppi finiti $H, K$, se $(|H|, |K|) = 1$ allora $H\times\{e_K\}$
    e $\{e_H\}\times K$ sono sottogruppi caratteristici di $H\times K$.
\end{proposition}

\begin{proof}
    Poniamo $m = |H|$, $n = |K|$, $S = \{(g_1, g_2) \in H\times K\mid 
    (g_1, g_2)^n = (e_H, e_K)\}$, osserviamo che $H\times \{e_K\} = S$, infatti 
    $H\times \{e_K\} \subseteq S$ in quanto tutti gli elementi di $H\times{e_K}$
    hanno ordine che divide $n$. D'altra parte dato $(g_1, g_2) \in S$, se
    $\ord(g_1, g_2) \mid n$ allora $\ord(g_1)\mid n$ e $\ord(g_2)\mid n$, ma 
    $\ord(g_2) \mid m$ per il Teorema di Lagrange, quindi $\ord(g_2) = 1$ e
    $S \subseteq H\times\{e_K\}$, da cui l'uguaglianza. Con un ragionamento
    analogo possiamo caratterizzare $\{e_H\} \times K$ come 
    \[
        \{e_H\} \times K = \{(g_1, g_2) \in H\times K\mid (g_1, g_2)^m = (e_H, e_K)\}
    \] Poiché un automorfismo di $H\times K$ deve preservare gli ordini degli
    elementi, per la caratterizzazione data abbiamo che i due sottogruppi sono
    caratteristici.
\end{proof}

\begin{corollary}
    Se $m, n \geqslant 2$ sono interi coprimi allora
    \[
        \Aut(\Zn\times\Zm) \cong \Aut(\Zn)\times \Aut(\Zm)
    \]
\end{corollary}

\newpage

\section{Gruppo derivato}

\begin{definition}
    Dati un gruppo $G$ e $x, y$ elementi di $G$, chiamiamo \vocab{commutatore}
    di $x$ e $y$ l'elemento $[x, y] = xyx^{-1}y^{-1}$. Chiamiamo \vocab{sottogruppo
    derivato} di $G$, oppure \vocab{sottogruppo dei commutatori} di $G$
     il sottogruppo 
    \[
        G' = \langle\{[x, y]\mid x, y \in G\}\rangle
    \]
\end{definition}

\begin{remark}
    $[x, y] = e$ se e solo se $x$ e $y$ commutano.
\end{remark}

\begin{proposition}
    Dato un gruppo $G$, valgono i seguenti fatti:
    \begin{enumerate}[(1)]
        \item $G'$ è un sottogruppo caratteristico di $G$;
        \item $\faktor{G}{G'}$ è un gruppo abeliano;
        \item dato $A$ un gruppo abeliano e $\varphi \in \Hom(G, A)$,
        allora $G' \subseteq \ker\varphi$.
    \end{enumerate}
\end{proposition}

\begin{proof}
    Mostriamo le affermazioni singolarmente:
    \begin{enumerate}[(1)]
        \item consideriamo $\varphi \in \Aut(G)$, poiché $\varphi$ preserva la struttura
        di gruppo è sufficiente descrivere come $\varphi$ agisce sui 
        generatori di $G'$ per determinare $\varphi(G')$. 
        Fissati $x, y \in G$, abbiamo 
        \[
            \varphi([x, y]) = \varphi(xyx^{-1}y^{-1}) = \varphi(x)\varphi(y)
            \varphi(x)^{-1}\varphi(y)^{-1} \in G'
        \]pertanto $\varphi(G') \subseteq G'$, da cui l'uguaglianza in quanto 
        $\varphi$ è bigettiva;
        \item dati $x, y \in G$, $xG'\cdot yG' = yG'\cdot xG'$ se e solo se 
        $xyG' = yxG'$, che è equivalente a richiedere $xyx^{-1}y^{-1}$. 
        Dato che effettivamente $xyx^{-1}y^{-1} = [x, y]$ è un elemento di $G'$
        abbiamo che $\faktor{G}{G'}$ è abeliano;
        \item dati $x, y \in G$, abbiamo 
        \[
            \varphi([x, y]) = \varphi(xyx^{-1}y^{-1}) = 
        \varphi(x)\varphi(y)\varphi(x)^{-1}\varphi(y)^{-1}
        \]
        e questo coincide con
        l'identità di $A$ in quanto $A$ è abeliano. Poiché l'immagine di $\varphi$
        è un sottogruppo di $A$ allora $G' \subseteq \ker\varphi$, in quanto
        il commutatore di ogni coppia di elementi di $G$ è contenuto in $\ker \varphi$.
    \end{enumerate}
\end{proof}

\begin{remark}
    Come conseguenza del Primo Teorema di Omomorfismo abbiamo che $\faktor{G}{G'}$ è 
    isomorfo al "più grande" sottogruppo abeliano di $G$, o analogamente che 
    $G'$ è il "più piccolo" sottogruppo di $G$ che produce un quoziente abeliano.
    In questo senso, $G'$ misura quanto è abeliano il gruppo $G$.
\end{remark}

\begin{remark}
    Dato $A$ un gruppo abeliano, il Primo Teorema di Omomorfismo produce una bigezione naturale tra 
    $\Hom(G, A)$ e $\Hom\left(\faktor{G}{G'}, A\right)$. Consideriamo infatti $\varphi \in \Hom(G, A)$,
    $\pi_{G'}: G \longrightarrow \faktor{G}{G'}$ la proiezione al quoziente e 
    $\overline{\varphi} : \faktor{G}{G'}\longrightarrow A$, il Teorema
    fornisce un'unico omomorfismo $\overline{\varphi}: \faktor{G}{G'}\longrightarrow A$
    che rende commutativo il diagramma
    \begin{center}
        \begin{tikzcd}[column sep = small, row sep = small]
            G\arrow[rrr, "\varphi"]\arrow[ddd, "\pi_{G'}"', two heads]& & &A \\
            {}\arrow[rr, "\circlearrowleft", phantom]& & {}& \\
            & & & \\
            \faktor{G}{G'}\arrow[uuurrr, "\overline{\varphi}"', hook]& & &
        \end{tikzcd}
    \end{center}
    Viceversa, dato un omomorfismo $\overline{\varphi}:\faktor{G}{G'}\longrightarrow A$
    otteniamo un'unico omomorfismo $\varphi:G\longrightarrow A$ con la 
    composizione $\pi_{G'}\circ\overline{\varphi}$.
\end{remark}

\begin{example}
    Consideriamo il gruppo $S_3$, chiaramente $(S_3)' \neq \{id\}$ in quanto
    $\faktor{S_3}{\langle id\rangle} \cong S_3$ che non è abeliano, pertanto 
    abbiamo due possibilità: $(S_3)' = S_3$ oppure $(S_3)' = \langle\cycle{1, 2, 3}\rangle$\footnote{
        Gli unici sottogruppi normali di $S_3$ sono $\{id\}$, 
        $\langle\cycle{1, 2, 3}\rangle$, $S_3$.}. D'altra parte 
        $\faktor{S_3}{\langle\cycle{1, 2, 3}\rangle}$ è isomorfo a $\Z2$, che
        è abeliano, pertanto $(S_3)'$ è contenuto in $\langle\cycle{1, 2, 3}\rangle$,
        da cui necessariamente $(S_3)' = \langle\cycle{1, 2, 3}\rangle$.
        Più in generale vedremo che $(S_n)' = \mathcal{A}_n$, dove $\mathcal{A}_n$ è il sottogruppo
        di $S_n$ delle permutazioni pari (sappiamo già che $(S_n)' \subseteq
        \mathcal{A}_n$ in quanto $\faktor{S_n}{\mathcal{A}_n}\cong\Z2$).
\end{example}

\newpage

\section{Azioni di gruppo}

\subsection{Azioni transitive}

\begin{definition}
    Siano $G$ un gruppo e $X$ un insieme, un'azione \[
        \varphi:G\longrightarrow S(X) :g \longmapsto \varphi_g
    \]si dice \vocab{transitiva} se per ogni $x, y \in X$ esiste $g \in G$
    tale che $\varphi_g(x) = y$, equivalentemente se $\Orb(x) = G$ per ogni 
    $x \in X$. Diciamo anche che G \vocab{agisce transitivamente} su $X$ 
    tramite $\varphi$.
\end{definition}

\begin{lemma}
    \label{lemma2.0}
    Dato $G$ un gruppo finito e $H \lneq G$ un suo sottogruppo proprio, allora \[
        G \neq \bigcup_{g \in G}gHg^{-1}
    \]
\end{lemma}

\begin{proof}
    Poniamo $K = \displaystyle\bigcup_{g \in G}gHg^{-1}$, osserviamo che gli
    elementi della forma $xHx^{-1}$ con $x \in N_G(H)$ contribuiscono una
    sola volta all'unione, in quanto $xHx^{-1} = H$, pertanto $K$
    è unione di $[G:N_G(H)] = \displaystyle\frac{|G|}{|N_G(H)|}$ elementi distinti\footnote
    {Infatti, se $X = \{N\mid N\leqslant G\}$ e $\varphi$ è l'azione di coniugio
    su $X$, per ogni $N \in X$ abbiamo $\St(N) = N_G(N)$ e $\Orb(N) = C_N =
    \{gNg^{-1}\mid g \in G\}$. Vale quindi la relazione $|G| = |C_N|\cdot|N_G(N)|$.}. Poiché
    $H \subseteq N_G(H)$ e $|gHg^{-1}| = |H|$ per ogni $g \in G$, possiamo stimare 
    la cardinalità di $K$ nel seguente modo 
    \[
        |K| \leq \frac{|G|}{|N_G(H)|}|H| \leq \frac{|G|}{|H|}|H| = |G|.
    \]D'altra parte, per il Principio di Inclusione-Esclusione abbiamo che $|K|$ 
    è somma delle cardinalità dei singoli termini dell'unione se e solo se 
    l'unione è disgiunta, ma questo è falso in quanto ogni classe di coniugio
    di $H$ contiene l'identità del gruppo, quindi $|K| < |G|$, cioè $G \neq K$.
\end{proof}

\begin{proposition}
    Dati un gruppo $G$ e un insieme $G$, se 
    \[
        \varphi:G\longrightarrow S(X) :g\longmapsto \varphi_g
    \]è un'azione transitiva valgono i seguenti fatti:
    \begin{enumerate}[(1)]
        \item per ogni $x, y \in X$ esiste $g \in G$ tale che $g\St(x)g^{-1} = \St(y)$;
        \item se $|X|\geqslant 2$ allora esiste $g \in G$ che agisce su $X$ senza
        punti fissi, cioè tale che $\varphi_g(x) \neq x$ per ogni $x \in X$.
    \end{enumerate}
\end{proposition}

\begin{proof}
    Mostriamo i due fatti singolarmente:
    \begin{enumerate}[(1)]
        \item sia $g \in G$ tale che $\varphi_g(x) = y$, dato 
        $h \in g\St(x)g^{-1}$ esiste $w \in \St(x)$ tale che $h = gwg^{-1}$. 
        Allora
        \[
            \varphi_h(y) = \varphi_{gwg^{-1}}(y) = 
            \varphi_g(\varphi_w(\varphi_h^{-1}(y))) = \varphi_g(\varphi_w(x)) =
            \varphi_g(x) = y
        \]pertanto $g\St(x)g^{-1} \subseteq \St(y)$. Osservando che 
        $\varphi_{g^{-1}}(y) = x$ e ragionando in modo simmetrico otteniamo
        l'inclusione $g^{-1}\St(y)g \subseteq \St(x)$, da cui $g\St(x)g^{-1} = \St(y)$;
        \item un elemento $g \in G$ con tali proprietà non può essere contenuto 
        nello stabilizzatore di nessun elemento di $X$, cioè cerchiamo $g \in G$
        tale che
        \[
            g \in \bigcap_{x \in X}\St(x)^{\mathcal{C}}
        \]
        che è equivalente a
        \[
            g \notin \bigcup_{x \in X} \St(x) = \bigcup_{h \in G}h\St(x_0)h^{-1}
        \]per il fatto precedente, fissato $x_0 \in G$. Osserviamo che 
        $\St(x_0) \neq G$, infatti se fosse $\St(x_0) = G$ avremmo 
        \[
            |\Orb(x_0)| = \frac{|G|}{|\St(x_0)|} = 1
        \]ma questo è assurdo in quanto $\Orb(x_0) = X$ per la transitività di 
        $\varphi$ e $|X|\geqslant 2$. Allora per il \hyperref[lemma2.0]{Lemma 5.2}
        abbiamo 
        \[
            G \neq \bigcap_{h \in G}h\St(x_0)h^{-1}
        \]pertanto esiste almeno un elemento $g\in G$ con la proprietà voluta.
    \end{enumerate}
\end{proof}

\begin{proposition}
    Dato $G$ un gruppo finito e $H \lneq G$ un sottogruppo proprio, se $[G:H] = p$
    con $p$ il più piccolo primo che divide l'ordine di $G$ allora $H$ è normale
    in $G$.
\end{proposition}

\begin{proof}
    Consideriamo l'azione di $G$ sull'insieme quoziente $\faktor{G}{H}$ 
    \[
        \psi: G\longrightarrow S\left(\faktor{G}{H}\right) : g \longmapsto \psi_g
    \]con 
    \[
        \psi_g : \faktor{G}{H}\longrightarrow\faktor{G}{H} : g'H\longmapsto gg'H
    \]Poiché l'immagine di $\psi$ è un sottogruppo di $S\left(\faktor{G}{H}\right)$,
    che è isomorfo a $S_p$, abbiamo che $|\mathrm{Im}\psi| \mid p!$, inoltre 
    $|\mathrm{Im}\psi| = \displaystyle\frac{|G|}{|\ker \psi|}$ come conseguenza
    del Primo Teorema di Omomorfismo. Pertanto $|\mathrm{Im}\psi| \mid (p!, |G|) = p$,
    in quanto $p$ è il più piccolo primo che divide $|G|$, quindi $|\mathrm{Im}\psi| \in \{1, p\}$.
\end{proof}

\end{document}
