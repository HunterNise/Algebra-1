\documentclass[11pt]{scrartcl}
\usepackage[italian]{babel}
\usepackage[sexy]{evan}

\begin{document}
\title{Complementi di Algebra 1}
\maketitle
\newpage

\tableofcontents
\eject

\newpage

\section{Insiemi di generatori}

\begin{definition}
    Dati un gruppo $G$ e $x_1, \dots, x_n$ elementi di $G$, chiamiamo \vocab{sottogruppo 
    generato} da $x_1, \dots, x_n$ il più piccolo sottogruppo $\left< x_1, \dots x_n
    \right>$ di $G$ contenente $x_1, \dots, x_n$, cioè \[\left<x_1, \dots, x_n\right> =
    \bigcap_{\substack{H\leq G\\ \{x_1, \dots, x_n\} \subseteq H}} H\] 
\end{definition}

\begin{remark}
    La definizione è ben posta, infatti l'intersezione avviene su una 
    famiglia non vuota di insiemi dal momento che $G$ è un sottogruppo di 
    $G$ contenente $x_1, \dots, x_n$. Inoltre l'intersezione non è vuota in 
    quanto contiene almeno l'identità e gli elementi $x_1, \dots, x_n$.
\end{remark}

La definizione data non dà informazioni su come sono fatti gli elementi di 
$\left<x_1, dots, x_n\right>$, cerchiamo quindi di caratterizzare in modo
diverso tale sottogruppo. In quanto sottogruppo, $\left<x_1, \dots, x_n\right>$
deve contenere tutti i prodotti finiti, in qualsiasi ordine, delle potenze di
$x_1, \dots, x_n$, cioè deve contenere l'insieme 
\[\{g_1^{\pm 1}, \dots, g_r^{\pm 1}\mid r \in \N, g_i \in \{x_1, \dots, x_n\}
~\forall i \in \{1, \dots, r\}\}\]

\end{document}