\documentclass[11pt]{scrartcl}
\usepackage[italian]{babel}
\usepackage[sexy]{evan}

\begin{document}
\title{Complementi di Algebra 1}
\subtitle{\large\normalfont\rmfamily\scshape APPUNTI DEL CORSO DI ALGEBRA 1 TENUTO\\ DALLA PROF. DEL CORSO E DAL PROF. LOMBARDO}
\author{Leonardo Migliorini \\ \textnormal{\href{l.migliorini@studenti.unipi.it}{l.migliorini@studenti.unipi.it}}}
\date{Anno Accademico 2022-23}
\maketitle
\newpage

\tableofcontents
\eject

\newpage

\section{Insiemi di generatori}

\begin{definition}
    Dati un gruppo $G$ e $x_1, \ldots, x_n$ elementi di $G$, chiamiamo \vocab{sottogruppo 
    generato} da $x_1, \ldots, x_n$ il più piccolo sottogruppo $\langle x_1, \ldots x_n
    \rangle$ di $G$ contenente $x_1, \ldots, x_n$, cioè \[\langle x_1, \ldots, x_n\rangle =
    \bigcap_{\substack{H\leqslant G\\ \{x_1, \ldots, x_n\} \subseteq H}} H\] 
\end{definition}

\begin{remark}
    La definizione è ben posta, infatti l'intersezione avviene su una 
    famiglia non vuota di insiemi dal momento che $G$ è un sottogruppo di 
    se stesso contenente $x_1, \ldots, x_n$. Inoltre l'intersezione non è vuota in 
    quanto contiene almeno l'identità e gli elementi $x_1, \ldots, x_n$.
\end{remark}

La definizione data non dà informazioni su come sono fatti gli elementi di 
$\langle x_1, \ldots, x_n\rangle$, cerchiamo quindi di caratterizzare in modo
diverso tale sottogruppo. Poiché chiuso per l'operazione indotta da $G$, $\langle x_1, \ldots, x_n\rangle$
deve contenere tutti i prodotti finiti, in qualsiasi ordine, delle potenze di
$x_1, \ldots, x_n$, cioè deve contenere l'insieme 
\[\{g_1^{\pm 1}, \ldots, g_r^{\pm 1}\mid r \in \NN, g_i \in \{x_1, \ldots, x_n\}
~\forall i \in \{1, \ldots, r\}\}\]

\begin{proposition}
\label{prop1.0}
Dati un gruppo $G$ e $x_1, \ldots, x_n$ elementi di $G$, allora \[
    \langle x_1, \ldots, x_n\rangle = \{g_1^{\pm 1}, \ldots, g_r^{\pm 1}\mid r 
    \in \NN, g_i \in \{x_1, \ldots, x_n\}~\forall i \in \{1, \ldots, r\}\}.
    \]
\end{proposition}

\begin{proof}
Poniamo $S = \{g_1^{\pm 1}, \ldots, g_r^{\pm 1}\mid r \in \NN, g_i \in \{x_1, \ldots, x_n\}
~\forall i \in \{1, \ldots, r\}\}$, mostriamo che $S$ è un sottogruppo di $G$. 
Effettivamente $e \in S$ in quanto è prodotto nessuna potenza di $x_1, \ldots, x_n$, 
il prodotto di due elementi di $S$ è ancora un elemento di $S$ in quanto
prodotto finito di potenze di $x_1, \ldots, x_n$ e l'inverso di un elemento
$g_1^{\pm 1}\ldots g_r^{\pm 1} \in\nolinebreak S$ è $(g_1^{\pm 1}\ldots 
g_r^{\pm 1})^{-1} = g_r^{\mp 1}\ldots g_1^{\mp 1}$, che è un elemento di $S$.
Abbiamo quindi che $S$ è un sottogruppo di $G$ contenente $x_1, \ldots, x_n$,
pertanto $\langle x_1, \ldots, x_n\rangle\subseteq S$ per minimalità di $\langle x_1,
\ldots, x_n\rangle$. D'altra parte, per quanto osservato sopra abbiamo che
tutti gli elementi della forma $g_1^{\pm 1}\ldots g_r^{\pm 1}$ con $r \in \NN$, 
$g_i \in \{x_1, \ldots, x_n\}$ per ogni $i \in \{1, \ldots, r\}$ devono essere
contenuti in $\langle x_1, \ldots, x_n\rangle$, pertanto i due sottogruppi
coincidono.
\end{proof}

\begin{remark}
    Se $G$ è un gruppo ciclico abbiamo che esiste $x \in G$ tale che 
    $\langle x\rangle = G$, cioè tutti gli elementi di $G$ sono potenze di $x$.
\end{remark}

Diciamo che $x_1, \ldots, x_n \in G$ sono \vocab{generatori} per $G$, o che 
l'insieme $\{x_1, \ldots, x_n\}$ \vocab{genera} $G$ se $\langle x_1, \ldots, x_n\rangle = G$.



\section{Gruppo diedrale}

\subsection{Elementi del gruppo}

\begin{definition}
    Dato $n \geqslant 2$ un naturale, consideriamo un poligono regolare di $n$ vertici,
    definiamo il \vocab{gruppo diedrale} su $n$ vertici $D_n$ come l'insieme 
    delle isometrie del piano che mandano i vertici in se stessi, cioè che 
    fissano il poligono (per $n = 2$ consideriamo le isometrie che mandano un 
    segmento in se stesso).
\end{definition}

\begin{remark}
    $D_n$ è un gruppo, in quanto l'applicazione identità che 
    fissa tutti i vertici è un'isometria dal poligono in se stesso, la 
    composizione di isometrie è un'isometria e un'isometria ammette sempre 
    un'inversa, che è anch'essa un'isometria.
\end{remark}

\begin{remark}
    Una rotazione di angolo $\displaystyle\frac{2\pi}{n}$ è un elemento di $D_n$,
    così come una simmetria rispetto a un asse.
\end{remark}

Proseguendo con questa intuizione geometrica, indicheremo con $r$ una rotazione
di angolo $\displaystyle \frac{2\pi}{n}$ e con $s$ una simmetria rispetto a
un qualsiasi asse. Notiamo che $\ord(r) = n$ e $\ord(s) = 2$ (per convenzione, 
indichiamo con un angolo positivo una rotazione in senso antiorario e con un 
angolo negativo una rotazione in senso orario).

\begin{definition}
    Data $r \in D_n$ una rotazione di ordine $n$, indichiamo con $\mathcal{R}$ il
    \vocab{sottogruppo delle rotazioni} $\langle r\rangle$.
\end{definition}

\begin{remark}
    Il sottogruppo $\mathcal{R}$ contiene tutte le rotazioni di $D_n$, infatti
    se $r'$ è una rotazione di angolo $\displaystyle\frac{2k\pi}{n}$, $k \in \ZZ$,
    allora $r^k = r'$ in quanto anche $r^k$ è una rotazione di angolo 
    $\displaystyle\frac{2k\pi}{n}$.
\end{remark}

Per determinare come sono fatti gli elementi di $D_n$, studiamo il sottogruppo
$\langle r, s\rangle$. Sicuramente $\langle r, s\rangle$ contiene il sottogruppo $\mathcal{R}$
e tutti gli elementi della forma $sr^k$, $sr^ks$, $sr^ksr^h$ e così via, vogliamo
mostrare che in effetti $D_n$ è generato da $r$ e $s$.

\begin{remark}
    Gli elementi della forma $r^k$ e $sr^h$ sono distinti per ogni $h, k \in \ZZ$. 
    Infatti sappiamo dall'algebra lineare che il determinante di una simmetria
    è $-1$ e che il determinante di una rotazione è $1$, per la moltiplicatività
    del determinante quindi $\det (r^k) = (\det r)^k = 1$ e
    $\det (sr^h) = (\det s)(\det r)^h = -1$, da cui $r^k \neq sr^h$.
\end{remark}

\begin{lemma}
    Per ogni rotazione $r \in D_n$ e per ogni simmetria $s \in D_n$ vale
    \[srs^{-1} = r^{-1}.\]
\end{lemma}

\begin{proof}
    \[
        srs^{-1} = r^{-1} \iff sr = r^{-1}s = (s^{-1}r)^{-1}.
    \] Si conclude
    osservando che $s^2 = 1$, pertanto $s^{-1} = s$ e 
    \[
        (s^{-1}r)^{-1} = (sr)^{-1} = r^{-1}s^{-1} = r^{-1}s.
    \]
\end{proof}

\begin{proposition}
    Se $n \geqslant 3$ allora $|D_n| = 2n$.
\end{proposition}

\begin{proof}
    Indicando con $1, \ldots, n$ gli $n$ vertici di un poligono regolare di $n$ lati, notiamo
    che un elemento $g \in D_n$ è univocamente determinato da $g(1), \ldots, g(n)$.
    In particolare, fissato $g(1)$, per il quale abbiamo $n$ possibili scelte,
    abbiamo al massimo due valori per $g(2)$, cioè $g(2) \in \{g(1) + 1, g(1) - 1\}$
    (a meno di sommare $n$ se uno dei due elementi è negativo). Poiché $g(1)$
    e $g(2)$ individuano due vettori nel piano non allineati, cioè
    linearmente indipendenti, ne costituiscono una base: fissati i valori di 
    $g(1)$ e $g(2)$ abbiamo quindi determinato ogni
    elemento di $D_n$ in modo unico e, poiché possiamo farlo in al più $2n$ modi, 
    $|D_n| \leqslant 2n$. Ricordiamo adesso che $D_n$ contiene gli elementi
    della forma $r^k$, $sr^h$ al variare di $h, k \in \ZZ$, mostriamo che questi sono 
    infatti $2n$. Gli elementi $r^k$ appartengono al gruppo ciclico $\mathcal{R}$
    di ordine $r$, pertanto sono $n$ elementi distinti, inoltre 
    \[
        sr^i = sr^j \iff r^i = r^j\iff i \equiv j \mod n,
    \] pertanto anche questi sono $n$
    elementi distinti. Allora $|D_n| = 2n$.
\end{proof}

\begin{remark}
    Abbiamo mostrato che effettivamente $D_n = \langle r, s\rangle$, quindi i
    suoi elementi sono tutti della forma $r^k$, $sr^h$ al variare di $h, k \in \ZZ$. 
\end{remark}

\begin{remark}
    Il risultato è valido anche per $D_2$, ma con motivazioni diverse. 
    Se consideriamo un segmento nel piano $\RR^2$ giacente sulla retta $y = 0$, 
    le isometrie che possiamo applicare sono l'identità, la rotazione di 
    angolo $\pi$, la simmetria lungo la retta $y = 0$ e la simmetria lungo l'asse
    passante per il suo punto medio. $D_2$ contiene quindi quattro elementi,
    l'identità e tre elementi di ordine due, pertanto è isomorfo a $\Z2\times\Z2$.
\end{remark}


\subsection{Sottogruppi}

Consideriamo un sottogruppo $H\leqslant D_n$, distinguiamo due possibilità: 
$H \subseteq \mathcal{R}$ oppure $H \nsubseteq \mathcal{R}$. Nel primo caso
abbiamo che $|H|\mid n$, ed è l'unico sottogruppo di $\mathcal{R}$ con questa 
proprietà in quanto $\mathcal{R}$ è ciclico, in particolare $H$ è ciclico 
della forma $ \displaystyle\left<\frac n d\right>$, con $d \mid n$. \newline
Studiamo quindi il caso $H \nsubseteq \mathcal{R}$: notiamo che 
$\mathcal{R}\trianglelefteqslant D_n$ in quanto $[D_n : \mathcal{R}] = 2$,
pertanto $\faktor{D_n}{\mathcal{R}}$ è un gruppo con l'operazione indotta da $D_n$
e risulta essere isomorfo a $\Z2$. \newline
Consideriamo la proiezione al quoziente \[
    \pi_{\mathcal{R}}: D_n \longrightarrow \faktor{D_n}{\mathcal{R}} : g \mapsto [g],
\]poiché $H \nsubseteq \mathcal{R}$ abbiamo che esiste $h \in H$ tale che 
$h \notin \mathcal{R}$, pertanto $\pi_{\mathcal{R}}(h) \notin [\mathcal{R}]$ e
in particolare $\pi_{\mathcal{R}}(H) \nsubseteq [\mathcal{R}]$. Dato che i 
sottogruppi di $\faktor{D_n}{\mathcal{R}}$ sono solo $\{[\mathcal{R}]\}$ e
$\faktor{D_n}{\mathcal{R}}$ abbiamo $\pi_{\mathcal{R}}(H) = 
\faktor{D_n}{\mathcal{R}}$. Osserviamo inoltre che $\ker \pi_{\mid H} = 
\ker \pi \cap H = \mathcal{R}\cap H$, per il Primo Teorema di Omomorfismo
allora $\faktor{H}{H\cap \mathcal{R}} \cong \Z2$, quindi 
$|H\cap\mathcal{R}| = \displaystyle\frac 1 2 |H|$. Dato che $R\cap H \subseteq
\mathcal{R}$, esiste $k \in \ZZ$ tale che $H\cap\mathcal{R} = \langle r^k\rangle$
in particolare $\langle r^k\rangle$ e $\langle sr^h\rangle$, $h \in \ZZ$, sono
contenuti in $H$. 

\begin{proposition}
    Dati $H\leqslant D_n$ un sottogruppo tale che $H\nsubseteq \mathcal{R}$, se
    $r$ è un generatore di $\mathcal{R}$ tale che $H\cap\mathcal{R} = \langle r^k\rangle$ 
    e $s$ è una simmetria allora \[
    H = \langle r^k\rangle\cdot\langle sr^h\rangle = \{xy \mid x \in \langle r^k\rangle,
    y \in \langle sr^h\rangle\}, h, k \in \ZZ.    
    \]
\end{proposition}

\begin{proof}
    Per quanto visto sopra abbiamo che $|\langle r^k\rangle| = 
    \displaystyle\frac 1 2|H|$, inoltre osserviamo che $\ord(sr^h) = 2$ in quanto
    \[
        (sr^h)^2 = sr^hsr^h = (srs)^hr^h = (srs^{-1})^hr^h = r^{-h}r^h = e,
    \]
    pertanto $\langle sr^h\rangle \cong \Z2$. Ricordiamo che, dati $K, N$ sottogruppi
    di un gruppo $G$, se vale almeno una delle inclusioni $K \subseteq N_G(N)$,
    $N \subseteq N_G(K)$ allora $HK = KH$ e quindi $HK$ è un sottogruppo di $G$. 
    Nel nostro caso abbiamo che $\langle sr^h\rangle
    \subseteq N_{D_n}(\langle r^k\rangle)$, infatti per ogni $m \in \ZZ$ abbiamo\[
        (sr^h)r^{mk}(sr^h)^{-1} = sr^{h + mk} sr^h = r^{-h-mk}r^h = r^{-mk}
        \in \langle r^k \rangle,
    \]cioè $\langle sr^h\rangle \subseteq N_{D_n}(\langle r^k\rangle)$ e quindi
    $\langle r^k\rangle\cdot\langle sr^h\rangle$ è un sottogruppo di $D_n$.
    Poiché $\langle r^k\rangle$ e $\langle sr^h\rangle$ sono contenuti in $H$
    abbiamo che $\langle r^k\rangle\cdot\langle sr^h\rangle \subseteq H$, inoltre
    \[|\langle r^k\rangle\cdot\langle sr^h\rangle| = \displaystyle\frac 1 2 |H|\cdot 2 = |H|\]
    in quanto $\langle r^k\rangle\cap\langle sr^h\rangle = \{e\}$, pertanto 
    i due sottogruppi coincidono.
\end{proof}

\begin{remark}
    Per $k \mid n$ e $0\leqslant h < k$, i sottogruppi $H_{k, h} = \langle r^k, sr^h\rangle$
    e $H = \langle r^k\rangle\cdot\langle sr^h\rangle$ coincidono. Infatti 
    $H_{k, h}\subseteq H$ in quanto $r^k, sr^h$ sono elementi di $H$, 
    d'altra parte $H \subseteq H_{k, h}$ in quanto $H_{h, k}$ contiene tutti i 
    prodotti finiti delle potenze di $r^k$ e $sr^h$, in particolare gli elementi di $H$.
\end{remark}

\begin{remark}
    Per $k \mid n$ e $0\leqslant h < k$, $\langle r^k, sr^h\rangle = 
    \langle r^k, sr^{h + k}\rangle$. Infatti $\langle r^k, sr^h\rangle \subseteq
    \langle r^k, sr^{h + k}\rangle$ in quanto $sr^h = (sr^{h + k})r^{-k}$ è
    un elemento del secondo gruppo, simmetricamente $\langle r^k, sr^{h + k}\rangle
    \subseteq \langle r^k, sr^h\rangle$ in quanto $sr^{h + k} = (sr^h)r^k$ è un
    elemento del primo gruppo.
\end{remark}

Abbiamo finito la classificazione dei sottogruppi di $D_n$.

\begin{theorem}
    [Classificazione dei sottogruppi di $D_n$]
I sottogruppi di $D_n$ sono della forma \begin{itemize}
    \item[(1)] $\langle r^k\rangle$ con $k\mid n$;
    \item[(2)] $\langle r^k, sr^h\rangle$ con $k \mid n$, $0\leqslant h < k$, 
\end{itemize}
con $r \in \mathcal{R}$ e $s$ una simmetria. Inoltre tali sottogruppi sono
tutti distinti.
\end{theorem}

\begin{proof}
    Abbiamo già visto che i sottogruppi di $D_n$ hanno una di queste forme, 
    mostriamo quindi che sono tutti distinti. A meno di cambiare $k$, possiamo
    supporre che $r$ generi $\mathcal{R}$, cioè $\ord(r) = n$. 
    Consideriamo $H, K\leqslant D_n$ due sottogruppi, abbiamo tre casi 
    \begin{itemize}
        \item se $H = \langle r^k\rangle$ e $K = \langle r^m\rangle$, $m \in \ZZ$,
        allora $H = K\iff k = m$ in quanto entrambi sottogruppi di un gruppo 
        ciclico, pertanto esiste un unico sottogruppo della forma $\langle r^k\rangle$
        per ogni $k \mid n$;
        \item se $H = \langle r^k\rangle$ e $K = \langle r^m, sr^h\rangle$, con $m \mid n$,
        allora $H \neq K$ in quanto $H$ è ciclico e $K$ no;
        \item se $H = \langle r^k, sr^h\rangle$ e $K = \langle r^m, sr^l\rangle$, 
        con $m \mid n$ e $0\leqslant l < m$, considerando le intersezioni
        $H \cap \mathcal{R} = \langle r^k\rangle$ e $K \cap \mathcal{R} = \langle r^m\rangle$ 
        abbiamo \[
        H \cap \mathcal{R} = K\cap\mathcal{R} \iff \langle r^k\rangle = \langle r^m\rangle
        \iff k = m.
        \] Inoltre, se $sr^h \in \langle r^m, sr^l\rangle = \langle r^m\rangle
        \cdot \langle sr^l\rangle$, allora esiste $t \in \ZZ$ tale che \[
        sr^h = (r^m)^t sr^l \iff sr^h = s^2r^{mt}sr^l \iff r^h = r^{-mt + l}
        \iff h \equiv l - mt \mod n,
        \]da cui ricaviamo $h \equiv l \mod m$ in quanto $m \mid n$. Ma allora 
        $h =l$ in quanto $0 \leqslant h < k$ e $0\leqslant l < m$.
    \end{itemize}
\end{proof}

\begin{lemma}
    \label{lemma1}
    Dati un gruppo $G$ e $A, B$ due sottogruppi tali che $A \leqslant B \leqslant G$,
    se $B\trianglelefteqslant G$ e $A$ è caratteristico in $B$ allora 
    $A \trianglelefteqslant G$.
\end{lemma}

\begin{proof}
    Fissato $g \in G$, consideriamo l'omomorfismo di coniugio 
    \[
        \varphi_g : G\longrightarrow G : x\longmapsto gxg^{-1},
    \] poiché 
    $B\trianglelefteqslant G$ è ben definita la restrizione \[
        \varphi_{g\mid B} : B\longrightarrow B :  b \longmapsto gbg^{-1},
    \]che è un elemento di $Aut(B)$. Dal momento che $A$ è
    un sottogruppo caratteristico di $B$ abbiamo che $\varphi_{g\mid B}(A) =
    \varphi_g(A) = A$,
    pertanto $A \trianglelefteqslant G$.
\end{proof}


\begin{corollary}
    Ogni sottogruppo di $\mathcal{R}$ è normale in $D_n$.
\end{corollary}

\begin{proof}
    Siano $\langle r^k\rangle$ un sottogruppo di $\mathcal{R}$ e $\varphi
    \in Aut(\mathcal{R})$, allora $\varphi(\langle r^k\rangle) = \langle r^k\rangle$
    in quanto $\varphi$ preserva l'ordine del sottogruppo e $\langle r^k\rangle$
    è l'unico sottogruppo di $\mathcal{R}$ di tale ordine ($\mathcal{R}$ è ciclico),
    pertanto $\langle r^k\rangle$
    è caratteristico in $\mathcal{R}$. Poiché $\mathcal{R}$ è un sottogruppo
    normale di $D_n$, per il \hyperref[lemma1]{Lemma 2.15}
    abbiamo $\langle r^k\rangle\trianglelefteqslant D_n$.
\end{proof}

\begin{corollary}
    Per $k\mid n$ e $0\leqslant h < k$, il sottogruppo $H_{k, h} = \langle r^k, sr^h\rangle$
    è normale in $D_n$ se e solo se $r, s \in N_{D_n}(H_{k, h})$.
\end{corollary}

\begin{proof}~
    \begin{itemize}
        \item Se $H_{k, h}\trianglelefteqslant D_n$ allora $N_{D_n}(H_{k, h}) = D_n$, 
        in particolare $r, s \in N_{D_n}(H_{k, h})$;
        \item se $r, s \in N_{D_n}(H_{k, h})$, poiché il normalizzatore è un
        sottogruppo di $D_n$ abbiamo che $D_n = \langle r, s\rangle \subseteq
        N_{D_n}(H_{k, h})$, pertanto $H_{k, h} \trianglelefteqslant D_n$.
    \end{itemize}
\end{proof}

Vediamo effettivamente quali sono i sottogruppi normali della forma 
$\langle r^k, sr^h\rangle$. Consideriamo gli automorfismi di coniugio \[
    \varphi_s: D_n \longrightarrow D_n :x \longmapsto sxs^{-1}\qquad
    \varphi_r: D_n \longrightarrow D_n :x \longmapsto rxr^{-1}
\]e sia $x_1^{\pm 1}\ldots x_m^{\pm 1} \in H_{k, h} = \langle r^k, sr^h\rangle$, allora
\[
    \varphi_s(x_1^{\pm 1}\ldots x_m^{\pm 1}) = \varphi_s(x_1)^{\pm 1}\ldots \varphi_s(x_m)^{\pm 1}
    \in \langle srs, r^hs^{-1}\rangle = \langle sr^ks, r^hs^{-1}\rangle = \langle
    r^k, sr^{-h}\rangle,
\]
\[
    \varphi_r(x_1^{\pm 1}\ldots x_m^{\pm 1}) = \varphi_r(x_1)^{\pm 1}\ldots \varphi_r(x_m)^{\pm 1}
    \in \langle r^k, rsr^{h - 1}\rangle = \langle r^k, sr^{h - 2}\rangle.
\]

Pertanto $H_{k, h}\trianglelefteqslant D_n$ se e solo se $\langle r^k, sr^{h - 2}\rangle
= \langle r^k, sr^{-h}\rangle = \langle r^k, sr^h\rangle$, se e solo se 
$h \equiv h - 2 \mod k$, cioè $k \in \{1, 2\}$.\begin{itemize}
    \item Se $k = 1$ allora $H_{k, h} = \langle r, s\rangle = D_n$;
    \item se $k = 2$ (e $n$ pari) allora $H_{k, h} = \langle r^2, sr\rangle$ oppure 
    $H_{k, h} = \langle r^2, s\rangle$.
\end{itemize}

\begin{remark}
    Il secondo caso si presenta solo se $n$ è pari, questo corrisponde al fatto 
    che in un poligono con un numero pari di lati gli assi di simmetria sono 
    per metà passanti per i lati e metà passanti per i vertici opposti. In un poligono con un numero dispari
    di lati gli assi di simmetria sono tutti passanti per i lati.
\end{remark}


\subsection{Classi di coniugio}

Abbiamo visto che possiamo scrivere ogni elemento di $D_n$ nella forma
 $s^hr^k$, dove $s$ è una simmetria e $r$ è una rotazione che genera 
 $\mathcal{R}$, con $h \in \{0, 1\}$ e $k \in \{0, \ldots, n - 1\}$
in quanto $\ord(s)= 2$ e $\ord(r) = n$. Inoltre tutti gli elementi della
forma $sr^h$ hanno ordine $2$.

Consideriamo la classe di coniugio di $r$, $C_r = \{grg^{-1}\mid g \in D_n\}$,
fissato $g \in D_n$ abbiamo due possibili valori per $grg^{-1}$:
\begin{itemize}
    \item se $g \in \mathcal{R}$ allora $g$ è una potenza di $r$, pertanto i
    due elementi commutano e si ha $grg^{-1} = r$;
    \item se $g\notin\mathcal{R}$ allora $g = sr^h$ con $h \in \ZZ$, quindi
    \[
        (sr^h)r(sr^h)^{-1} = (sr^h)r(sr^h) = sr^{h + 1}sr^h = s^2r^{-1-h}r^h = r^{-1},
    \]
\end{itemize}
cioè $C_r = \{r, r^{-1}\}$. In modo analogo si mostra che $C_{r^k} = \{r^k, r^{-k}\}$
per ogni $k \in \ZZ$.

\begin{remark}
    Se $n$ è pari, scriviamo $n = 2m$ e consideriamo la classe di coniugio
    di $r^m$. Poiché $r^m \neq e$ e $r^{2m} = (r^m)^2 = e$ abbiamo che
    $\ord(r^m) = 2$, cioè $(r^m)^{-1} = r^m$. Allora $C_{r^m} = \{r^m\}$,
    pertanto abbiamo trovato un elemento del centro di $D_n$ (infatti se $G$
    è un gruppo e $x \in G$, allora $x \in Z(G)$ se e solo se $C_x = \{x\}$).
\end{remark}

Consideriamo adesso la classe di coniugio di $sr^h$, $C_{sr^h} = 
\{g(sr^h)g^{-1}\mid g\in D_n\}$, fissato $g \in D_n$ abbiamo due possibili
valori per $g(sr^h)g^{-1}$:
\begin{itemize}
    \item se $g \in \mathcal{R}$ allora $g = r^k$ con $k \in \ZZ$, pertanto
    \[
        r^k(sr^h)r^{-k} = sr^{-k}r^h r^{-k} = sr^{h - 2k};
    \]
    \item se $g \notin \mathcal{R}$ allora $g = sr^k$ con $k \in \ZZ$, pertanto
    \[
        (sr^k)(sr^h)(sr^k)^{-1} = (sr^k)(sr^h)(sr^k) = sr^{2k - h},
    \]
\end{itemize}
cioè $C_{sr^k} = \{sr^{h - 2k}, sr^{2k - h}\mid k \in \ZZ\}$. 

\begin{remark}
    La classe di coniugio di $sr^h$ contiene tutte le simmetrie in cui 
    l'esponente di $r$ ha la stessa parità di $h$. Se $n$ è 
    dispari tutte le simmetrie appartengono alla stessa classe,
    mentre se $n$ è pari abbiamo due classi distinte: quella 
    delle simmetrie rispetto agli assi passanti per i vertici opposti e quella 
    delle simmetrie rispetto agli assi passanti per i lati.
\end{remark}
\end{document}